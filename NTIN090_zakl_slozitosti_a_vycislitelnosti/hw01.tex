\documentclass{article}

\title{%
    Domácí úkol č. 1 - Turingovy stoje \\
    NTIN090
}
\author{Zdeněk Tomis}
\date{6. 10. 2024}

\usepackage[utf8]{inputenc}
\usepackage[czech]{babel}
\setlength{\parindent}{0pt}
\setlength{\parskip}{6pt} 

\usepackage[margin=0.7in]{geometry}
\hyphenpenalty=10000

\usepackage{amssymb}
\usepackage{mathrsfs}
\usepackage{gensymb}
\usepackage{amsmath}
\renewcommand{\familydefault}{\sfdefault}
\usepackage{graphicx}

\begin{document}


\maketitle


\section{ Jen dvě akce najednou}

Pro \( M = (\mathcal{Q}, \Sigma, \delta, q_0, F ) \) sestrojme \(M^\prime\) následovně:  

Vytvoříme stroj, který nikdy nebude najednou měnit stav pásky a zároveň hýbat hlavou, a to tak, že místo toho nejdříve zapíšeme na pásku, ale stav změníme na mezistav, který nám řekne pouze kam v dalším kroku pohnout hlavou a na jaký konečný stav nastavit stav.

Formálně (alespoň částečně):

\[ Q^\prime = Q \cup Q_{mezi} \]

kde 

\[ Q_{mezi} = Q \times \{ L,R,N \}  \]

Máme původní stavy a pak stavy složené, takové mezistavy, které nám říkají, kam v mezikroku pohnout páskou a na jaký stav se nastavit poté.

Abeceda, počáteční stav, přijimajíci stavy zustanou nezměněny.

Přechodovou funkci nejprve upravíme takto:

\begin{align*} \forall q \in Q; l_0,l_1 \in \Sigma; m \in \{ L,R,N \}&: \\
 \delta(q,l_0) &= (q_n, l_1,  m) \implies \\
 \delta^\prime(q,l_0) &= ( (q_n, m), l_1, N) \  \wedge \\
 \delta^\prime((q_n, m), l_1) &= (q_n, l_1, m)    
\end{align*}

Všimněme si, že oba kroky přechodové funkce (oba prvky), zachovávají buď zachovávají písmeno na pásce ($ l_1 \to l_1 $), nebo nehýbou hlavou - pevně nastavené $N$.

Toto není jediný přístup, jako další mě napadá kódovat mezistav ne do stavu, ale na pásku ve formě rozšířené abecedy.


\section{RESTART - Hello, IT, have you tried turning it off and on again?}

Zde budeme muset adresovat dva problémy

\begin{enumerate}
    \item chybí pohyb doleva
    \item chybí pohyb nikam
    \item páska je jednostranná,   
          pohybem doleva lze narazit na kraj
\end{enumerate}

\subsection{Pohyb do leva (L)}

Pohyb do leva nasimulujeme takto:

Abecedu si rozšíříme: $\Sigma^\prime = \Sigma \cup (\Sigma \times \{ \prime, \prime\prime \} )$

Když se chceme pohnout doleva, uděláme toto:

\begin{enumerate}
    \item napíšeme místo znaku znak se zarážkou ($\prime$) (a posuneme se doprava)
    \item provedeme RESTART
    \item pokud jsme na pozici se zarážkou, jsme na začátku pásky (vizte poslední bod řešení)
    \item zkontrolujeme, jestli zarážka není hned za první buňkou, v takovém případě se RESTARTem dostaneme na buňku nalevo. (zarážku odmažeme)
    \item jinak, první pozici označíme druhou, posouvací zarážkou ($\prime\prime$)
    \item opakujeme:
    \begin{enumerate}
        \item zkontrolujeme, jestli první druhá zarážka není dvě buňky před první: 
        \begin{itemize}
            \item Jedeme po pásce v pravo, dokud nenarazíme na posouvací zarážku ($\prime\prime$)
            \item Když na ni narazíme, posuneme se dvakrát v pravo a zkontrolujeme, jestli tam není první zarážka. \\
            Pokud ano, vyskočíme z cyklu, jinak RESTART a pokračujeme na (b)
        \end{itemize}
        \item Posuneme posouvací zarážku doprava
       \begin{itemize}
        \item Jedeme v pravo
        \item až narazíme na posouvací zarážku, smažeme ji, posuneme se doprava a připíšeme ji
        \item RESTART
        \item pokračujeme na (a)
       \end{itemize}
    \end{enumerate}

    \item Nyní jsme ve stavu, kde posouvací zarážka $\prime\prime$ je vlevo od místa, kde chceme mít hlavu, zarážka $\prime$ je o dvě buňky vpravo.
    \item Odmažeme původní zarážku $\prime$
    \item RESTART
    \item Posuneme se na zarážku $\prime\prime$
    \item Odmažeme $\prime\prime$, nastavíme stav na stav před posunem a při tom se posuneme doprava.
\end{enumerate}

Nyní jsme o jednu pozici vlevo, páska je v původním stavu a stav je nastaven jako před posunem.

Všechny tyto operace lze kodifikovat v konečném množství stavů, původní stav si zapamatujeme tak, že stavy obsahují i tento původní stav (opět kartézský součit)

\subsection{Pohyb nikam (N)} 

Pohyb nikam můžeme nasimulovat pohybem doprava a poté doleva dle postupu výše. Zabere to mnoho kroků, ale o to nám zde nejde.

\subsection{Omezení pásky}

Tento problém jsme řešili na cvičení.

V tomto případě ho detekujeme tak, že při pohybu doleva v kroku 3. výše popsaného algoritmu narazíme na hlavní zarážku $\prime$ ihned po restartu.

V tomto případě můžeme celou pásku posunout o jedno do prava, poté provést RESTART.

Nově přidané levé pole bude prázdné, což je správně, nutnost posunu pásky potvrzuje, že pole ještě nebylo navšítveno, tudíž má být prázdné.

Problém, který jsme neadresovali ani na cvičení, může být, že nevíme, jakou část pásky musíme posunout do prava, tedy kdy můžeme přestat s přesunem.

Tento problém se dá vyřešit tak, že si vytvoříme nějakou zarážku napravo od vstupu. Pokaždé, když chceme psát na tuto zarážku, nebo se posunout za ni, posuneme ji doprava a pak se posuneme zpět doleva (simulovaně)

\end{document}
