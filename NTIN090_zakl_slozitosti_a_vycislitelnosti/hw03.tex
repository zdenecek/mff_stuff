\documentclass{article}

\title{%
    Domácí úkol č. 3   \\
    NTIN090
}
\author{Zdeněk Tomis}
\date{19. 11. 2024}

\usepackage[utf8]{inputenc}
\usepackage[czech]{babel}
\setlength{\parindent}{0pt}
\setlength{\parskip}{6pt} 

\usepackage[margin=0.7in]{geometry}
\hyphenpenalty=10000

\usepackage{amssymb}
\usepackage{mathrsfs}
\usepackage{gensymb}
\usepackage{amsmath}
\renewcommand{\familydefault}{\sfdefault}
\usepackage{graphicx}
\usepackage{enumitem}   % For customizing enumerate labels (e.
\usepackage{algorithm}
\usepackage{algpseudocode}

\begin{document}


\maketitle


\section{Variace na existenční kvantifikaci}

\subsection{$\implies$}

\begin{equation}
    \exists B_1 \text { rozhodnutelný} \cap
    A=\{ x \in \Sigma^* \mid(\exists  \in \Sigma^* [\langle x,y\rangle \in B_1]\} \implies A \text{ je rozhodnutelný} 
\end{equation}

Plyne z existenční kvantifikace (5. přednáška). Stačí využít existence jednoho z jazyků $B_1$ a $B_2$.

\subsection{$\impliedby$}

Existuje Turingův stroj $M$ pro jazyk $A$ přijimajici $L(A = L(M))$.

Platí 
\begin{equation}
    L = \{x \mid (\exists n \in N)[M(x)\text{ přijme do n kroků}]\}
\end{equation}

"přijme do n kroků" je rozhodnutelná podmínka, stačí spustit $M$ po $n$ kroků na $x$.

Definujeme 

\begin{equation} 
    B_1 = \{\langle x , \langle n \rangle \rangle \mid M(x)\text{ přijme do n kroků}\}
    B_2 = \{\langle x , \langle n \rangle \rangle \mid M(x)\text{ přijme do n+1 kroků}\}
\end{equation}

Jsou to různé jazyky, oba jsou rozhodnutelné a splňují 

\begin{equation}
    L = \{ x \in \Sigma^* \mid (\exists y \in \Sigma^* )[ \langle x,y \rangle \in B_i ]\}
\end{equation}

kde $y = \langle n \rangle $ pro $B_1$ a $y = \langle n+1 \rangle $ pro $B_2$.

\section{Rozhodnutelnost $PTS$(D)}

$PTS$ není rozhodnutelný. Kdyby byl rozhodnutelný, byl by nutně rozhodnutelný jeho doplněk, který odpovídá problému: Existuje alespoň jeden řetězec $x$, pro který se výpočet $M(x)$ zastaví, ale na pásce zůstane slovo $y$, které není delší než $x$?

Kdyby byl tento jazyk rozhodnutelný, existoval by k němu Turingův stroj $M_{\overline{PTS}}$, který by ho rozhodoval.

Ukažme, že $ \overline{PTS} \le_T L_U $

Kdyby existoval $M_{\overline{PTS}}$, mohli bychom sestrojit $M_{L_U}$, který by ho využíval jako podprogram, a to takto:

Pro daný $M, x$, upravíme $M$ na $M^\prime$ těmito třemi úpravami:

\begin{enumerate}
    \item Uložíme vstup $x^\prime$ bokem a místo něho použijeme $x$ - například stroj simulujeme na části pásky.
    \item Při přijetí před skončením smažeme pásku (nahradíme $\lambda$)
    \item Při zamítnutí před skončením na pásku napíšeme $x^\prime . 0$ a uměle tak necháme na pásce prodloužený vstup.
\end{enumerate}

Takto upravený stroj při přijetí skončí s páskou kratší nebo stejně dlouhou jako vstup (stejně při vstupu nulové délky). Při zamítnutí bude mít pásku delší než vstup.

Spustíme $M_{\overline{PTS}}$ na $M^\prime$. Pokud zamítne, znamená to, že M nepřijme, nebo se nezastaví nad $x$.

Pokud přijme, znamená to, že $M$ přijme a zastaví se nad $x$.

Tímto chytákem odignorování původního vstupu docílíme chování $M_{L_U}$.

To ale není možné, protože $L_U$ není rozhodnutelný, tudíž $\overline{PTS} $ není rozhodnutelný a z uzavřenosti $PTS$ není rozhodnutelný.

\subsection{Částečná rozhodnutelnost}

Ukázali jsme, že  $ \overline{PTS} \le_T L_U $. Ve skutečnosti jsme ale ukázali více, a sice m-převoditelnost, protože konstrukce $M^\prime$ je vyčíslitelná funkce.

Tedy  $ \overline{PTS} \le_m L_U $. 

$L_U$ je částečně rozhodnutelný, tedy i $\overline{PTS}$ je částečně rozhodnutelný.

Zároveň víme, že $PTS$ není ani částečně rozhodnutelný, protože kdyby byl rozhodnutelný $PTS$ i jeho doplněk, byl by $PTS$ rozhodnutelný.

\end{document}
