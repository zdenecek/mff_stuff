\documentclass[10pt,a4paper]{article}
\usepackage[utf8]{inputenc}
\usepackage{amsmath, amsfonts, amssymb, amsthm}
\usepackage{mathtools, array, enumitem, xcolor}
\usepackage[margin=0.7in]{geometry}
\setlength{\parindent}{0em}

\usepackage[czech]{babel}

\theoremstyle{plain}
\newtheorem{veta}{Věta}
\theoremstyle{definition}
\newtheorem{definice}[veta]{Definice}


\title{Řešení domácího úkolu z MA č. 3}
\author{přezdívka: Zdeněk}
\date{}

\begin{document}

\maketitle

\section{Volte n}

\begin{enumerate}[label=(\alph*)]
\item \[ \lim_{n \to \infty} \frac{ \lfloor \sqrt{n} \rfloor }{n} = 0
 \]
 Funkce/posloupnost je kladná, neřešíme dolní omezení.
\[ \varepsilon > \frac{ \lfloor \sqrt{n} \rfloor }{n}   \impliedby 
\varepsilon > \frac{\sqrt{n}}{n} \implies \sqrt{n} > \frac1\varepsilon \]
\[ n > \frac1{\varepsilon^2} \]

\item \[ \lim_{n \to \infty} \sqrt[n]{a} = 1 \]

Funkce/posloupnost je opět kladná, tudíž dolní omezení neřešíme.

\[ 1 + \varepsilon > \sqrt[n]{a}  \]
\[ 1 + \varepsilon > a^{\frac1n} \]
\[ \log_a{(\varepsilon + 1)} > \frac1n \]
\[ n > \frac1{ \log_a{(\varepsilon + 1)}} \]

\end{enumerate}

\section{Spočtěte}

\begin{enumerate}[label=(\alph*)]
\item \begin{equation*}
\lim_{n \to \infty} \sum^n_{i=1} \frac{2^i + 3^i}{6^i} = \lim_{n \to \infty} \sum^n_{i=1} \frac{2^i}{6^i} +  \lim_{n \to \infty}\sum^n_{i=1} \frac{3^i}{6^i} = \lim_{n \to \infty} \sum^n_{i=1} \frac{1}{3^i} +  \lim_{n \to \infty}\sum^n_{i=1} \frac{1}{2^i}
\end{equation*}

Pro součty nekonečných geometrických řad využijeme odvozený vztah:

\[ \sum^n_{i=1} q^i = q \cdot \frac{q^n-1}{q - 1} \]

Pro $0 < q < 1$
\[ \lim_{n \to \infty} q \cdot \frac{q^n-1}{q - 1} = \frac{-q}{q-1}\]

Tedy

\begin{equation*}
\lim_{n \to \infty} \sum^n_{i=1} \frac{1}{3^i} +  \lim_{n \to \infty}\sum^n_{i=1} \frac{1}{2^i} = \frac{-\frac{1}{3}}{\frac{1}{3} - 1} +  \frac{-\frac{1}{2}}{\frac{1}{2} - 1} = \frac{1}{2} + 1 = \frac32
\end{equation*}

\item 
\[ \lim_{n \to \infty} \frac{\frac{\sqrt{n-1} - \sqrt{n}}{n}-1}{n} =
 \lim_{n \to \infty} \frac{\frac{n-1 - n}{n (\sqrt{n-1} + \sqrt{n})}-1}{n} = \lim_{n \to \infty} {\frac{-1}{n^2 (\sqrt{n-1} + \sqrt{n})}} - \lim_{n \to \infty} \frac1n  = \] 
 
 \[
= \lim_{n \to \infty} {\frac{-1}{n^2 (\sqrt{n(1-\frac1n)} + \sqrt{n})}} - \lim_{n \to \infty} \frac1n =\lim_{n \to \infty}  {\frac{-1}{n^{\frac32} (\sqrt{1-\frac1n} + 1)}} - \lim_{n \to \infty} \frac1n = 0 - 0 = 0 \]

\end{enumerate}

\section{Bonus}
V čitateli se jedná o součin dvou polynomů, první z nich má člen nejvyššího řádu $2^{20}n^{20}$ a druhý polynom má člen nějvyššího řádu $3^{30}n^{30}$. Jejich součin je stupně $50$, člen nejvyššího řádu je  $2^{20}3^{30}n^{50}$, ostaní členy jsou nižšího řádu.

\hfill

Ve jmenovateli je polynom, jehož člen nejvyššího řádu je $5^{50}n^{50}$.

\hfill

V čitateli i jmenovateli vytkneme $n^{50}$ a zkrátíme. Všechny členy obou polynomů, jejichž řád je menší než 50, můžeme pominout, protože se limitně rovnají nule. Zůstane tedy

\[\frac{2^{20}3^{30}}{5^{50}} = \frac{215892499727278669824}{88817841970012523233890533447265625} \approx 2.43 \cdot 10^{-15} \]

Výpočet jsem provedl pomocí internetové kalkulačky.

\end{document}