\documentclass[10pt,a4paper]{article}
\usepackage[utf8]{inputenc}
\usepackage{amsmath, amsfonts, amssymb, amsthm}
\usepackage{mathtools, array, enumitem, xcolor}
\usepackage[margin=0.7in]{geometry}
\setlength{\parindent}{0em}

\usepackage[czech]{babel}

\theoremstyle{plain}
\newtheorem{veta}{Věta}
\theoremstyle{definition}
\newtheorem{definice}[veta]{Definice}


\title{Řešení domácího úkolu z MA č. 12}
\author{přezdívka: Zdeněk}
\date{}

\begin{document}

\maketitle

\section{}

\begin{enumerate}[label=(\alph*)]

\item 
\[  \int_0^\infty x\arctan(2x^2+1)\ dx \]

\[ y = 2x^2+1\]
\[ \frac{dy}{dx} = 4x \implies dx = \frac{dy}{4x}   \]

Tedy

\[ =\frac14  \int_1^\infty \arctan(y)\ dy \]

Funkce $\arctan$ je neklesající na daném intervalu a její limita je nenulová, takže tento integrál nekonverguje.

Můžeme to odhadnout například integrálem tvořící obdelník výšky $\frac\pi4$ 

\[ \int_1^\infty \arctan(y)\ dy  \geq \int_1^\infty \frac\pi4  dx = \infty \]


\item

$\sin$ je spojitá na $\pi$, takže platí:

 \[ \int_0^{2\pi} \sin^3(x)\ dx = \int_0^{\pi} \sin^3(x)\ dx + \int_{\pi}^{2\pi} \sin^3(x)\ dx \]
 
 \[ \int_0^{\pi} \sin^3(x)\ dx + \int_{0}^{\pi} \sin^3(x + \pi)\ dx
 = \int_0^{\pi} \sin^3(x)\ dx + \int_{0}^{\pi} (-\sin(x))^3 \ dx \]\[
 = \int_0^{\pi} \sin^3(x)\ dx - \int_{0}^{\pi} \sin^3(x)\ dx = 0 \]
 
\end{enumerate}

\section{}


\[ \sum^\infty_{n=1} \frac{n}{n^2 + 1} - \frac1{n}  =  \sum^\infty_{n=1} \frac{-1}{n(n^2 + 1)} \]


Použijeme srovnávací kriterium:

\[ a =\frac{-1}{n^3 + n}  \]
\[ b =  \frac{-1}{n^3}  \]
\[ \lim_{n \to \infty} \frac{a}{b} = \lim_{n \to \infty} \frac{n^3}{n^3 + n} = 1 \]

Takže suma konverguje právě tehdy, když konverguje suma

\[ \sum^\infty_{n=1} \frac{-1}{n^3}\]

Tato suma konverguje, pokud konverguje opačná suma 

\[ \sum^\infty_{n=1} \frac{1}{n^3}\]

Tato suma konverguje podle kondenzačního kritéria, právě když konverguje

\[ \sum^\infty_{n=1} \frac{2^n}{(2^n)^3} = \sum^\infty_{n=1} \frac{1}{2^{2n}} =  \sum^\infty_{n=1} \left( \frac{1}{4} \right)^n \]

Toto je geometrická řada s kvocientem menším než jedna, takže konverguje.

Tedy původní řada konverguje.

\end{document}