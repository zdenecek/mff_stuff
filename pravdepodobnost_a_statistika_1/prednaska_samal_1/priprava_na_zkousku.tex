\documentclass[10pt,a4paper]{article}
\usepackage[utf8]{inputenc}
\usepackage{amsmath, amsfonts, amssymb, amsthm}
\usepackage{mathtools, array, enumitem, xcolor}
\usepackage[margin=0.7in]{geometry}
\setlength{\parindent}{0em}

\usepackage{tikz}
\usepackage{pgfplots}
\pgfplotsset{compat=1.11}

\usepackage[czech]{babel}

\theoremstyle{plain}
\newtheorem{veta}{Věta}
\theoremstyle{definition}
\newtheorem{definice}[veta]{Definice}


\title{Příprava na zkoušku z předmětu Pravděpodobnost a statistika}
\author{Zdeněk Tomis}
\date{}

\begin{document}

\maketitle

\section{Definice}

\begin{enumerate}
\item \textbf{Pravděpodobnostní prostor} je trojice $(\Omega, \mathcal{F}, P)$, kde

\begin{itemize}
\item $\Omega$ je \textbf{množina elementárních jevů} (sample space), $\Omega \neq \emptyset$
\item $\mathcal{F}$ je \textbf{prostor jevů} (sigma algebra/$\sigma$) $\mathcal{F} 2^\Omega$, takový že
\begin{enumerate}
\item $\emptyset, \Omega \in \mathcal{F} $
\item $A \in \mathcal{F} \implies \Omega \setminus A \in \mathcal{F}$
\item $A_1, A_2, ... \in \mathcal{F} \implies \cup^\infty_{i=1} A_i \in \mathcal{F}$ Pro $A_1, A_2, ...$ po dvou disjunktní
\end{enumerate}

Tedy $\mathcal{F}$ je uzavřená na průniku, sjednodcení (spočetném), doplňku.
\item $P$ je \textbf{pravděpodobnost}  $P: \mathcal{F} \to [0,1]$, taková že
\begin{enumerate}
\item $P(\Omega) = 1$
\item $P(\cup^\infty_{i=1}) = \sum^\infty_{i=1} P(A_i)$
\end{enumerate}
\end{itemize}

\item \textbf{Podmíněná pravděpodobnost} pro $A,B \in \mathcal{F}, P(B) > 0$
\[P(A|B) = \frac{P(A \cap B)}{P(B)}\]

\item \textbf{Nezávislost jevů}. Dva jevy jsou nezávislé, pokud 

\[P(A) \cdot P(B) = P(A\cap B)\]

Pro více jevů:
\[\prod_{j \in I^\prime} P(A_j) = P\left( \bigcap_{j \in I^\prime \subseteq I} A_j\right)\]

\item \textbf{Náhodná diskrétní veličina.} X je náhodná diskrétní veličina, pokud

\begin{itemize}
\item $X: \Omega \to \mathbb{R}$, t. ž. $Im(X)$ je spočetná množina
\item $\forall x \in \mathbb{R}: \{\omega \in Omega: X(\omega) = x \} \in \mathcal{F}$
\end{itemize}

\item \textbf{Pravděpodobnostní funkce} náhodné veličiny $X$ (probability mass function) je \[p_X: \mathbb{R} \to [0,1] \]
taková, že  
\[ p_X(x) = P(X = x) \]

\item \textbf{Indikátorová náhodná veličina} jevu $A$ je 
\[ I_A = \begin{cases} 1 & \omega \in A \\ 0 & \omega \notin A \end{cases}\]

Platí $(I_A) \sim Bern(P(A))$

\item \textbf{Střední hodnota} náhodné veličiny $X$ je

\[ \mathbb{E}(X) = \sum_{x \in Im(X)} x \cdot p_X(x)\]

pokud má součet smysl.

\item \textbf{Podmíněná střední hodnota} náhodné veličiny $X$ je
\[ \mathbb{E}(X|B) = \sum_{x \in Im(X)} x \cdot p(X = x|B)\]

\item \textbf{Rozptyl} náhodné veličiny $X$ je

Střední hodnota čtverců vzdáleností od střední hodnoty.

\[ var(x) = \mathbb{E}\left[(X-\mathbb{E}(X))^2 \right] \]

\item\textbf{Směrodatná odchylka} náhodné veličiny $X$ je

\[ \sigma_X = \sqrt{var(X)} \]

\item \textbf{Sdružená pravděpodobnostní funkce} náhodných veličin $X,Y$

\[ P_{X,Y}: \mathbb{R}^2 \to [0,1]\]
\[ P_{X,Y}(x,y) = P(X = x \  \& \  Y = y)\]

\item \textbf{Nezávislost veličin}. Náhodné veličiny $X,Y$ jsou nezávislé, pokud

\[ \forall x,y: \{ X=x \} \ \text{a} \ \{Y=y\} \text{ jsou nezávislé}\]

\item Multinomiální rozdělení (?) 

\[P_{X_1, ..., X_n}(k_1, ..., k_n) = \binom{n}{k_1, ..., k_n} \  p_1^{k_1} \cdot ... \cdot p_n^{k_n} \]

\item \textbf{Kovariance} náhodných veličin je 
\[ cov(X,Y) = \mathbb{E}(XY) - \mathbb{E}(X) \mathbb{E}(Y) \]

\item \textbf{Korelace} náhodných veličin $X,Y$ je

\[ cor(X,Y) = \frac{cov(X,Y)}{\sigma_X \cdot \sigma_Y} \]

\item \textbf{Obecná náhodná veličina} $X$

\begin{itemize}
\item $X: \Omega \to \mathbb{R}$,  $Im(X)$ nemusí být spočetná
\item $\forall x \in \mathbb{R}: \{\omega \in Omega: X(\omega) = x \} \in \mathcal{F}$
\end{itemize}

\item \textbf{Distribuční funkce} (cummulative distribution function) náhodné veličiny $X$ je $F_X$

\[ F_X(x) = P(X \leq x)\]

\item \textbf{Spojitá náhodná veličina} je taková náhodná veličina, která má hustotu:

\[ \exists f_X: \mathbb{R} \to [0, \infty) \]

taková, že

\[ F_X(x) = \int^x_{-\infty} f_X(t) dt\]

\item \textbf{Střední hodnota spojité veličiny}. $X$ je spojitá náhodná veličina s hustotou $f_X$

\[ \mathbb{E}(X) = \int^{\infty}_{-\infty} x f_X(t)dt\]

\item Kvantilová funkce

Pro $X$ náhodnou veličinu, $Q_X : [0,1] \to \mathbb{R}$

\[ Q_X(p) = min(\{x \in \mathbb{R}: p \leq F_X(x) \})\]

Pokud je $F_X$ spojitá a rostoucí, pak $Q_X = F^{-1}_X$

\item \textbf{Náhodný vektor} je dvojice (n-tice) náhodných veličin

\[\forall x,y \in \mathbb{R}\] je definována \textbf{sdružená distribuční funkce}
\[F_{X,Y}(x,y) = P(X = x\ \&\ Y = y)\]

\item \textbf{Sdružená spojitost}

Pokud $\forall x,y: F_{X,Y}(x,y) = \int^x_{-\infty}\int^y_{-\infty} f_{X,Y}(s,t) dt ds$

$X,Y$ jsou sdružené spojité a $f_{X,Y}$ je jejich sdružená hustota.

\item (Alternativně) \textbf{Nezávislost náhodných veličin}.

\[ \forall x,y: \{ X\leq x \} \ \text{a} \ \{Y \leq y\} \text{ jsou nezávislé}\]

\end{enumerate}

\section{Věty}

\subsection*{Diskrétní}

\begin{veta}[O pravděpodobnostech] \ 


\begin{enumerate}
\item $P(A) + P(A^C) = 1$
\item $A \subseteq B \implies P(A) \leq P(B)$
\item $P(A \cup B) = P(A) + P(B) - P(A \cap B)$
\item $P(A_1 \cup A_2 \cup ...) \leq P(A_1) + P(A_2) + ...$
\end{enumerate}

\begin{proof}
\begin{enumerate}
\item Množiny jsou disjunktní, tedy platí $P(A) + P(A^C) = P(A \cup A^C) = P(\Omega) = 1$
\item $B = A \cup (B \setminus A)$

\[ P(B) = P(A) + P(B \setminus A) \]
\[  P(B \setminus A)  \in [0,1] \]
\[ P(B) \geq P(A) \]

\item \[ P(A \cup B) = P( (A \setminus B) \cup (A \cap B) \cup (B \setminus A))\]
\[ = P(A \setminus B) +  P(A \cap B) + P(B \setminus A)\]
\[ = P(A) -  P(A \cap B)+  P(A \cap B) + P(B) +  P(A \cap B)\]
\[ = P(A) + P(B) - P(A \cap B)  \]

\item Trik zdisjunktnění

$B_1 = A_1$

$B_2 = A_2 - A_1$

$B_i = A_i - \sum_{j=1}^{i-1} A_j$

\[ P(\bigcup A_i) = P(\bigcup B_i) = \sum P(B_i) \leq \sum P(A_i)\]

\end{enumerate}
\end{proof}

\end{veta}

\begin{veta}[O průnicích]
\[ P(A_1 \cap ... \cap A_n) = P(A_1) \cdot P(A_2|A_1) \cdot ... \cdot P(A_n|A_1 \cap ... \cap A_n) \]
\begin{proof}
\[ = P(A_1) \cdot \frac{P(A_1 \cap A_2)}{P(A_1)} \cdot \frac{P(A_1 \cap A_2 \cap A_3)}{P(A_1 \cap A_2)} \cdot ... \cdot 
\frac{P(A_1 \cap ... \cap A_n)}{P(A_1 \cap ... \cap A_{n-1})} = P(A_1 \cap ... \cap A_n) \]
\end{proof}
\end{veta}


\begin{veta}[O úplné pravděpodobnosti]
Pro $A, B_1, ... \in \mathcal{F}$

Kde $B_1, B_2, ...$ je rozklad $\Omega$ platí

\[ P(A) = \sum P(B_i) \cdot P(A|B_i)\] 

Kde pro $P(B_i) = 0$ je sčítanec definován jako nula.

\begin{proof}
\[ P(A) = \sum P(A \cap B_i) = \sum P(A)P(A|B_i) \]
\end{proof}
\end{veta}


\begin{veta}[Bayesova]
\[ P(A|B) = \frac{P(B|A) P(A)}{P(B)}\]
\begin{proof}
\[ P(A|B)  = \frac{P(A \cap B)}{P(B)} = \frac{P(B|A) P(A)}{P(B)} \]
\end{proof}
\end{veta}


\begin{veta}
$g(X)$ je diskrétní náhodná veličina, kdykoliv $X$ je diskrétní náhodná veličina.
\begin{proof}
\begin{enumerate}
\item $Im(g \circ X)$ je nejvýše spočetný
\item $\forall y \in Im(Y): Y^{-1}(y) \in \mathcal{F}$
\end{enumerate}
\end{proof}
\end{veta}


\begin{veta}[Law of the unconscious statistician (LOTUS)]
\[ \mathbb{E}(g(X)) = \sum_{x \in Im(X)} g(x) P(X = x)\]
\begin{proof}

\[ \mathbb{E}(Y) 
= \sum_{y \in Im(Y)} y \  P(Y = y) \]\[ 
\{ Y = y \} = \{ \omega: g(X(\omega)) = y \} = \bigcup_{x \in Im(X) \ \&\ g(x) = y} \{ \omega: X(\omega) = x \} \]

Tedy 

\[ = \sum_{y \in Im(Y)} y \sum_{\begin{matrix}x \in Im(X) \\ g(x) = y \end{matrix}}\ P(X=x) = \sum_{y \in Im(Y)}\ \sum_{\begin{matrix}x \in Im(X) \\ g(x) = y \end{matrix}} g(x)\ P(X=x) =  \]\[
=  \sum_{x \in Im(X)} g(x)\ P(X=x) \]
\end{proof}
\end{veta}


\begin{veta}[Vlastnosti $\mathbb{E}$]
\begin{enumerate} \ 
\item $P(X \geq 0) = 1\ \&\ \mathbb{E}(X) = 0 \implies P(X = 0) = 1$
\item $\mathbb{E}(X) \geq c \implies P(X \geq c) > 0$
\item $\mathbb{E}(aX + b) = a\mathbb{E}(X)+b$
\item $\mathbb{E}(X + Y) = \mathbb{E}(X) + \mathbb{E}(Y)$
\end{enumerate}
\begin{proof}
\begin{enumerate}  
\item[3.] 
$g(x) = ax + b$
\[ \mathbb{E}(aX + b) =\mathbb{E}(g(x)) = \sum_{x \in Im(X)} g(x)\ P(X=x) \]
\[ = a \sum_{x \in Im(X)} x\ P(X=x) + \sum_{x \in Im(X)} b\ P(X=x) = a\mathbb{E}(X)+ b   \]
\item[4.] Pouze pro diskrétní prostor
\[\mathbb{E}(X + Y) = \sum_{\omega in \Omega} (X(\omega) + Y(\omega)) P(\omega) =
\sum_{\omega in \Omega} X(\omega) +  \sum_{\omega in \Omega} Y(\omega) P(\omega) = \mathbb{E}(X) + \mathbb{E}(Y)\]
\item[1.] 
\[ \mathbb{E}(X) = \sum_{x \in Im(X)} x P(X=x) \geq 0\]
Ale zároveň pro $x < 0: P(X = x) = 0$. Máme tedy pouze nezáporné členy
\[ \implies x P(X = x) = 0 \]
\[ \implies \forall x \neq 0: P(X = x) = 0 \]
\item[2.]
\[c \leq \mathbb{E}(X) = \sum x P(X = x)\]

Sporem.

\[P(X \geq c) = 0 \implies \forall x \geq c: P(X = x) = 0\]
\[ \implies \sum x P(X = x) <  \sum c P(X=x) = c \sum P(X = x) = c \]
\[ c < c\]
\end{enumerate}

\end{proof}
\end{veta}


\begin{veta}[O celkové $\mathbb{E}$]

$X$ je diskrétní náhodná veličina. $B_1, B_2, ...$ je rozklad $\Omega$

\[ \mathbb{E}(X) = \sum P(B_i) \ \mathbb{E}(X|B_i)\]
\begin{proof}
\[\mathbb{E}(X) = \sum x P(X=x) = \sum_x x \sum_i P(B_i) P(X=x|B_i)  \]
\[\sum_i \sum_x x\ P(B_i) P(X=x|B_i) = \sum_i \mathbb{E}(X|B_i) \]
\end{proof}
\end{veta}

\begin{veta}[O rozptylu]
\[var(X) = \mathbb{E}(X^2) - \mathbb{E}(X)^2 \]
\begin{proof}
$\nu = \mathbb{E}(X)$

\[ var(X) = \mathbb{E}\left[ (X-\mu)^2 \right] =\mathbb{E}\left[ X^2 - 2X\mu + \mu^2 \right] \]\[= \mathbb{E}(X^2) - 2\mathbb{E}(X)\mathbb{E}(X) + \mathbb{E}(X)^2 = \mathbb{E}(X^2) -\mathbb{E}(X)^2 \]
\end{proof}
\end{veta}

\begin{veta}
\[ var(aX + b) = a^2 var(X) \]
\end{veta}


\begin{veta}
Pro $Im(X) \subseteq \mathbb{N}_0$

\[\mathbb{E}(X)  = \sum^{\infty}_{n=0} P(X > n)\]
\end{veta}

\begin{veta}[Konvoluce]
\[ P(X + Y = z) = \sum_{x \in Im(X)} P(X = x \ \&\ Y=z-x)\]
\[ \sum_{x \in Im(X)} p_X(x) \cdot p_Y(z-x) \]
\end{veta}

\begin{veta}
\[\mathbb{E}(g(X,Y)) = \sum_{\begin{matrix}
x \in Im(X) \\ y \in Im(Y)
\end{matrix}} g(x,y) P_{X,Y}(x,y)\]
\end{veta}

\begin{veta}[Alternativní důkaz pro vlastnosti $\mathbb{E}$]
\[\mathbb{E}(X+Y) = \mathbb{E}(X) + \mathbb{E}(Y)\]
\begin{proof}
\[ = \sum_{\begin{matrix}
x \in Im(X) \\ y \in Im(Y)
\end{matrix}} (x+y)P(X = x \ \&\ Y=y) \]\[
= \sum_{\begin{matrix}
x \in Im(X) \\ y \in Im(Y)
\end{matrix}} xP(X = x \ \&\ Y=y) \quad + \sum_{\begin{matrix}
x \in Im(X) \\ y \in Im(Y)
\end{matrix}} yP(X = x \ \&\ Y=y)
\]\[
= \sum_x x \sum_y P(...) +  \sum_y y \sum_x P(...) \]\[
=  \mathbb{E}(X) + \mathbb{E}(Y)
\]
\end{proof}
\end{veta}

\begin{veta}
Pro nezávislé náhodné veličiny $X,Y$
\[ \mathbb{E}(XY) = \mathbb{E}(X)\mathbb{E}(Y)\]
\begin{proof}
\[\mathbb{E}(XY) = \sum_{x,y} (xy) P(x = X\ \&\ Y=y) = \sum_{x,y} (x,y)P(Y=y)P(X=x)  \]\[ 
= \left( \sum_x p_X(x)x) \right) \cdot \left( \sum_y p_Y(y)y) \right) 
= \mathbb{E}(X)\mathbb{E}(Y)\]
\end{proof}
\end{veta}


\begin{veta}
\[var(X+Y) = var(X) + var(Y) + 2cov(X,Y) \]
\begin{proof}
\[ a = \mathbb{E}\left[ (X+Y)^2 \right] = \mathbb{E}(X^2)  + \mathbb{E}(Y^2) + 2\mathbb{E}(XY)  \]
\[ b  = \left[ \mathbb{E}(X+Y) \right]^2  =  \mathbb{E}(X^2)  + \mathbb{E}(Y^2) + 2 \mathbb{E}(X) \mathbb{E}(Y)\]
\[ var(X+Y) = a - b = var(X) + var(Y) + 2(\mathbb{E}(XY) -\mathbb{E}(X) \mathbb{E}(Y)) \]
\end{proof}
\end{veta}


\begin{veta}
\[ cov(X,Y) = \mathbb{E}\left[ (X-\mathbb{E}(X))(Y-\mathbb{E}(Y)) \right]\]
\begin{proof}
Roznásobit a sečíst.
\end{proof}
\end{veta}


\begin{veta}[Cauchyho nerovnost]
\[cov(X,Y) \leq \sqrt{var(X)\cdot var(Y)}\]

neboli

\[-\sigma_X\cdot \sigma_Y \leq cov(X,Y) \leq \sigma_X\cdot \sigma_Y\]
\end{veta}


\begin{veta}[Rozptyl součtu]
\[ var(X_1 + ... + X_n) = \]
\begin{enumerate}
\item \[ = \sum_i \sum_j cov(X_i, X_j)\]
\item \[ = \sum_i  var(X_i) \sum_{j \neq j} cov(X_i, X_j)\]
\item Pro nezávislé veličiny
 \[ = \sum_i  var(X_i) \]
\end{enumerate}
\begin{proof}
Kombinatoricky
\[ \mathbb{E}\left[\left(\sum X_i\right)^2\right] = \sum_i \sum_j \mathbb{E}( X_i X_j) \] 

\[ \left[ \mathbb{E} \left(\sum X_i\right)^2\right]^2 = \sum_i \sum_j \mathbb{E}( X_i) \mathbb{E}(X_j) \]

\begin{enumerate}
\item \[\implies  var\left(\sum X_i \right) = \sum_i \sum_j \mathbb{E}( X_i X_j) - \mathbb{E}( X_i) \mathbb{E}(X_j)  =   \sum_i \sum_j = cov(X_i, X_j) \] 
\item $cov(X,X) = var(X)$
\item Pro nezávislé veličiny je kovariance nulová.
\end{enumerate}
\end{proof}
\end{veta}

\subsection*{Spojité náhodné veličiny}

\begin{veta}[O]
\[\]
\begin{proof}
\[\]
\end{proof}
\end{veta}


\begin{veta}[O]
\[\]
\begin{proof}
\[\]
\end{proof}
\end{veta}


\begin{veta}[O]
\[\]
\begin{proof}
\[\]
\end{proof}
\end{veta}


\begin{veta}[O]
\[\]
\begin{proof}
\[\]
\end{proof}
\end{veta}


\subsection*{Statistika}

\section{Veličiny}

\subsection{Diskrétní veličiny}

\renewcommand{\arraystretch}{3}

\begin{tabular}{| l | c | c | c | c | }
\hline
  & Značení & pravděpodobnostní funkce & $\mathbb{E}$ & rozptyl \\ 
   \hline
Bernouliho  & $Bern(p)$ & $p_X(1) = p$, $p_X(0) = 1 - p$  & $p$ & $p(1-p)$ \\
 \hline
Geometrické  &  $Geom(p)$ & $p_X(x) = (1-p)^{x-1}p$ & $\frac1p$ & $\frac{1-p}{p^2}$\\
 \hline
Binomiální  & $Bin(n,p)$ & $p_X(k) \binom{n}{k}p^k(1-p)^{n-k}$ pro správná k, jinak $0$ & $np$ & $np(1-p)$ \\
 \hline
Hypergeometrické  & & $p_X(k) = \frac{\binom{K}{k}\binom{N-K}{n-k}}{\binom{N}{n}}$ & & $n\frac{K}{N}$ \\
 \hline
Poissonovo  & $Poi(\lambda)$ & $p_X(k) = e^{-\lambda} \frac{\lambda^k}{k!}$ & & $\lambda$\\
 \hline
  
\end{tabular}

\subsection{Spojité veličiny}

\begin{tabular}{| l | c | c | c | c | c | }
\hline
  & Značení & hustota & $F$ & $\mathbb{E}$ & rozptyl \\ 
 \hline
Uniformní & $U(a,b)$ & $\begin{cases} \frac1{b-a} & x \in [a,b] \\ 0 & x \notin [a,b] \end{cases}$ &  $\frac{x-a}{b-a}$ & $\frac{a+b}2$ & $\frac{(b-a)^2}{12}$ \\
 \hline
Exponenciální & $Exp(\lambda)$ & $\lambda e^{-\lambda x}$ &  $1 - e^{-\lambda x}$ & $\frac{1}{\lambda}$&  $\frac{1}{\lambda^2}$  \\
 \hline
Normální & $N(\mu, \sigma^2)$& $\frac{1}{\sigma}\varphi(\frac{x-\mu}{\sigma})$ & $\Phi(\frac{x-\mu}{\sigma})$  & $\mu$ & $\sigma^2$ \\
 \hline
 
 Cauchyho & & $\frac1{\pi(x+1)^2}$ & $\frac1\pi \arctan x - \frac12 $ & nemá &  nemá \\
 \hline
  
\end{tabular}

\paragraph{Náhodné vektory}

Hustota a distribuční funkce
\[ F_{X,Y}(x,y) = \int_{-\infty}^x  \int_{-\infty}^y  f_{X,Y} (s,t) dt ds\]
\[  f_{X,Y}(x,y)= \int_{-\infty}^x = \frac{\partial^2 F_{X,Y}(x,y)}{\partial x \partial y} \]
Marginální hustota
\[ f_X(x) = \int^\infty_{-\infty} f_{X,Y}(x,y)dy\]
Množiny
\[ P((X,Y) \in A) = \int_A f_{X,Y}(x,y) dx dy \]
Nezávislost X,Y
\[  \iff f_{X,Y}(x,y) = f_{X}(x) f_{Y}(y) \iff F_{X,Y}(x,y) = F_{X}(x) F_{Y}(y) \]

\renewcommand{\arraystretch}{0.8}


\begin{tabular}{| c | c | c | c | c | c |}
\hline
x & 0 & 1 & 2 & 3 & 4 \\
\hline
$\Phi(x)$ & 0.5 & 0.84135 & 0.97725 & 0.99865 & 0.99997 \\
\hline
\end{tabular}

\end{document}
