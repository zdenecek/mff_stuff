\documentclass[10pt,a4paper]{article}
\usepackage[utf8]{inputenc}
\usepackage{amsmath, amsfonts, amssymb, amsthm}
\usepackage{mathtools, array, enumitem, xcolor}
\usepackage{stackengine}
\usepackage[margin=0.7in]{geometry}
\setlength{\parindent}{0em}
\setlength{\parskip}{0.6em}

\usepackage{graphicx}
\newcommand\sbullet[1][.5]{\mathbin{\vcenter{\hbox{\scalebox{#1}{$\bullet$}}}}}


\usepackage{tikz}
\usepackage{pgfplots}
\pgfplotsset{compat=1.11}

\usepackage[czech]{babel}

\newtheorem*{lemma}{Lemma}
\theoremstyle{plain}
\newtheorem*{veta}{Věta}
\newtheorem*{pozorovani}{Pozorování}
\newtheorem*{tvrzeni}{Tvrzeni}
\newtheorem*{domnenka}{Domněnka}


\renewcommand{\arraystretch}{1.5}


\title{Příprava na zkoušku z předmětu Kombinatorika a Grafy 1}
\date{}

\begin{document}

\maketitle

\subsection{Odhady}

\begin{veta}
	\[ \forall n \in \mathbb{N}: n^{\frac{n}2} \leq n! \leq n^n\]
	\begin{proof}
		\begin{itemize}
			\item \[ (n!)^2 = (1\cdot n) \cdot ... \cdot (n\cdot 1) \geq n^n \]
			Rozepsat.
			\item triv.
		\end{itemize}
	\end{proof}
\end{veta}

\begin{lemma}[1.3]
	\[\forall r \in \mathbb{R}: 1+x \leq e^x \]
	\begin{proof}
		Analyzujeme funkci $f(x) = e^x - x - 1$.
	\end{proof}
\end{lemma}

\begin{veta}[1.2]
	\[ e\left(\frac{n}{e}\right)^n \leq n! \leq e\cdot n \left(\frac{n}{e}\right)^n \]
	\begin{proof}
		Obě části indukcí. Vyjdeme z úpravy $n \left( \frac{n-1}{e} \right)^{n-1} = e \left( \frac{n}{e} \right)^n \cdot e \left( \frac{n-1}{n} \right)^{n-1}$
	\end{proof}
\end{veta}

\begin{veta}[Stirlingova formule]
	\[ n! \sim \left( \frac{n}e \right)^n \cdot \sqrt{2\pi e} \]
\end{veta}

\begin{veta}[1.5] \begin{enumerate}[]
		\item \[ \left) \frac{n}{K} \right)^n \leq \binom{n}{K} \leq n^K \]
		\item Největší binomický koeficient je pro $\lceil  \frac{K}2 \rceil$ a $\lfloor  \frac{K}2 \rfloor$
\end{enumerate}	
\begin{proof}
	\begin{enumerate}
		\item \[ \binom{n}{K} = \prod_{i=0}^{K-1} \frac{n-i}{K-i} \begin{cases}
		\leq n \\
		\geq \frac{n}{K}		
		\end{cases} \]

	\item  \[ \binom{n}{K} = \binom{n}{K-1} \frac{n+1-K}{K} \begin{cases}
		< 1 & \text{pro } k > \frac{n+1}2\\
		> 1  & \text{pro } k < \frac{n+1}2
		\end{cases} \]
\end{enumerate}
\end{proof}
\end{veta}

\begin{veta}[1.6]
	\[\forall m \in \mathbb{N}: \frac{2^{2m}}{2\sqrt{m}} \leq \binom{2m}{m} \leq  \frac{2^{2m}}{\sqrt{2m}}\]
	\begin{proof}
		\begin{itemize}
			\item Definujeme $P = \frac{1 \cdot 3 \cdot 5 \cdot \cdots \cdot (2m-1)}{2 \cdot 4 \cdot 6 \cdot \cdots \cdot (2m)} $
			\[ P = P \cdot \frac{2 \cdot 4 \cdot 6 \cdot \cdots \cdot 2m}{2 \cdot 4 \cdot 6 \cdot \cdots \cdot 2m} = \frac{(2m)!}{2^{2m}}(m!)^2 = \frac{1}{2^{2m}} \binom{2m}{m} \]

			Stačí ukázat $P < \frac{1}{\sqrt{2m}}$
			\[ 1 > \left( 1- \frac{1}{2^2} \right)\left( 1- \frac{1}{4^2} \right)\cdots \left( 1- \frac{1}{(2m)^2} \right)
			= \left( \frac{1 \cdot 3}{2^2} \right)\left( \frac{3 \cdot 5}{4^2} \right)\cdots \left( \frac{(2m-1) \cdot (2m+1)}{(2m))^2} \right)
			= P^2 (2m+1) \] \[
				\implies P < \frac{1}{\sqrt{2m+1}} < \frac{1}{\sqrt{2m}}
				\]
		\end{itemize}
	\end{proof}
\end{veta}

\subsection{Vytvořující funkce}

\paragraph{Vytvořující funkce posloupnosti} $(a_i)^\infty_{i=0}$ je \[ \sum_{i=0}^\infty a_i x^i \]
\paragraph{Základní vytvořující funkce}
\begin{enumerate}
\item \[(\underbrace{1,1,...,1}_{(n+1)-\text{krát}},0,0,...) \to 1 + x + x^2 + ... + x^n = \frac{1-x^{n+1}}{1-x} \]
\item \[(1,1,1,1,...) \to 1 + x + x^2 + ...  = \frac{1}{1-x} \]
\item \[\left(\binom{n}{0},\binom{n}{1},\binom{n}{2},... \right) \to \binom{n}{0} + \binom{n}{1}x + \binom{n}{2}x^2 + ... + \binom{n}{n}x^n  = (1+x)^n \]
\end{enumerate}


\paragraph{Operace s vytvořujícími funkcemi}

\subsection{Binomická věta}

\paragraph{Zobecněný binomický koeficient} pro $r \in \mathbb{R}$ a $k \in \mathbb{N}_0$ definujeme jako $\binom{r}{k} = \frac{r(r-1)...(r-k+1)}{k!}$

\begin{veta}[Zobecněná binomická věta]
	\[ \forall r \in \mathbb{R} \text{ je } (x+1)^r \text{ vytvořující funkcí } \left( \binom{r}0, \binom{r}1, ... \right)\]
	Navíc pro $x \in (-1;4)$ řada $\sum_{i=0}^\infty \binom{r}i x^i$ konverguje.
	\begin{proof}
		Přes taylorův rozvoj.
	\end{proof}
\end{veta}

\begin{veta}
	\[  \]
\end{veta}

\begin{veta}[Počet zakořeněných stromů]
	\[b_n = \frac{1}{n+1} \binom{2n}{n} \]
	\begin{proof}
	
	\end{proof}
\end{veta}



\subsection{Konečné projektivní roviny}

\paragraph{Hypergraf} je trojice $(V,E, k), k \in k \geq 2$, $E \subseteq \binom{V}{k}$

\paragraph{Množinový systém} je dvojice $(X, \mathcal{P})$, kde \begin{itemize}
\item $X$ je nosná množina
\item $\mathcal{P} \subset 2^X$
\end{itemize}
$X$ jsou body a $\mathcal{P}$ jsou přímky.

\paragraph{Konečná projektivní rovina} je množinový systém $(X, \mathcal{P})$, ve kterém platí tyto tři podmínky: 
\begin{enumerate}[label=(A\arabic*)]
\item Každé dva body tvoří právě jednu přímku
\[\forall x,y \in X, x \neq y: \exists! Q \in \mathcal{P}: x,y \in P \]
\item Každé dvě přímky se protínají v právě jednom bodě
\[\forall P,Q \in \mathcal{P}, P \neq Q: \exists! x \in X: x \in P \wedge x \in Q \]
\item Exitují čtyři body v obecné poloze, kde žádné tři nejsou na přímce:
\[ \exists C \subseteq X, |C| = 4: \forall P \in \mathcal{P}: |C \cap P| \leq 2 \]
\end{enumerate}

\subparagraph{Přiklad: Fanova rovina}

\paragraph{Řád KPR} je $\# \text{bodů na přímce roviny} - 1$
\paragraph{Duální množinový systém} k $(X, \mathcal{P})$ je množinový systém 
\[ (\mathcal{P}, \{ \{ P \in \mathcal{P} | x \in P  \} | x \in X\})\]
\paragraph{Incidenční graf} pro množinový systém $(X, \mathcal{P})$ je bipartitní graf 
$(X \cup \mathcal{P}, E)$, kde $E = \{ \{ x, P \} | x \in P \}$

\subsection{Latinské čtverce}

\paragraph{Latinský čtverec řádu $n$} je tabulka $n \times n$,  s prvky $\{1,...,n\}$, kde v každém řádku i sloupci je každý prvek právě jednou.

\paragraph{Ortogonalita} Latinské čtverce $L, L^\prime$ jsou ortogonální, pokud \[ \forall l, l^\prime \in \{1,...,n\}: \exists i,j \in \{1,...,n\}: L_{i,j}^\prime = l \& L_{i,j} = l^\prime \]

\subsection{Grafy}

\paragraph{Vrcholové pokrytí} grafu $G = (V,E)$ je množina $C \subseteq V$, kde platí
$\forall e \in E: C \cap e \neq \emptyset$

\paragraph{Párování v grafu} G je podgraf tvořený disjunktními hranami.

\paragraph{Množinový systém} Nechť $X$ a $I$ jsou množiny: Množinový systém na $X$ je $|I|$-tice $\mathcal{M}=(M_i | i \in I)$, kde $M_i \subseteq X$.

\paragraph{Systém různých reprezentantů} (SRR) pro $\mathcal{M}$ množ. systém je prostá $f: I \to X$, t.ž. $\forall i \in I: f(i) \in M_i$.

\subsection{Souvislost grafů}
\paragraph{Hranový řez} v $G=(V,E)$ je $F \subseteq E$, t.ž. $G \setminus F = (V; E \setminus F)$ je nesouvislý.

\paragraph{Vrcholový řez} v   $G=(V,E)$ je $A \subseteq V$, t.ž. $G \setminus A = (V \setminus A, E \cap \binom{V \setminus A}{2})$ je nesouvislý.

\paragraph{Hranová souvislost} $k_e(G) = \begin{cases} 
\text{minimální velikost hranového řezu} \\
 1 \text{ pro } G \equiv K_1 \end{cases}$
 
\paragraph{Vrcholová souvislost} $k_v(G) = \begin{cases} 
\text{minimální velikost vrcholového řezu} \\
 n-1 \text{ pro } G \equiv K_n \\
 1 \text{ pro } G \equiv K_1 \end{cases}$
 
\paragraph{k-souvislost} G je $\circ\  k$-souvislý, pokud $k_\circ(G) \geq k$ 
 
\paragraph{Most} je hranový řez velikosti 1
\paragraph{Artikulace} je vrcholový řez velikosti 1
\paragraph{Podrozdělení hrany} Pro $G=(V,E), e \in E$ je podrozdělení hrany $e$ nahrazení hrany cestou.

\paragraph{Laplacián} grafu $G$ je $L(G) \in \mathbb{R}^{n \times n}$, kde 
\[ L(G)_{i,j} = \begin{cases}
deg(i) & i=j \\
-1		& i\neq j, \{i,j\} \in E \\
1		& i\neq j, \{i,j\} \notin E
\end{cases}\]

\subsection{Ramseyova teorie}

\paragraph{Nezávislý systém} Systém $M \subseteq 2^\{1,...,n\}$ je nezávislý, pokud
\[\forall A,B \in M, A \neq B: A \nsubseteq B\ \&\ B \nsubseteq A\]

\paragraph{Ramseyovo číslo} $R(k,l)$ je $\min n \in \mathbb{N}$, t.ž. každé červeno-modré obarvení $E(K_n)$ obsahuje červené $K_k$, nebo modré $K_l$

\paragraph{Obecné Ramseyovo číslo} $R_p(n_1, ..., n_r)$ je $\min n \in \mathbb{N}$, t.ž. 

\[\forall X, |X| \geq n\] 
\[\forall r\text{-obarvení } \chi \text{ množiny } \binom{X}{p}\]
\[\exists \text{barva } i \in \{1,...,r\}\]
\[\exists Y \subseteq X:\]
\[|Y| = n_i \text{ a všechny p-tice z } \binom{Y}{p} \text{ mají barvu } i\]

\subsection{Samoopravné kódy}

\paragraph{Abeceda} $\Sigma$ je konečná množina symbolů.
\paragraph{Slovo délky n} je uspořádáná n-tice symbolů z abecedy
\paragraph{ $\Sigma^n$ } je množina všech slov délky $n$
\paragraph{Hammingova vz8dlenost slov} $x=(x_1,...,x_n)$ a $y=(y_1,...,y_n)$ je $d(x,y)$ definována jako počet indexů, kde se x a y liší. 

$d$ je metrika a  $(\Sigma^n, d)$ je metrický prostor

\paragraph{(Blokový) Kód} C je $C \subseteq \Sigma^n$ množina kódových slov.

\paragraph{Parametry kód} jsou \begin{itemize}
\item Délka slov n
\item Velikost abecedy q
\item Dimenze $k = log_q|C|$
\item Minimální vzdálenost d dvou různých kódových slov
\end{itemize}  

Kód s parametry $n,q,k,d$ značíme $(n,k,d)_q$.

\paragraph{Hadamardova matice} je $H \in \{-1,1\}^{n\times n}$, kde $HH^T = n \cdot I_n$

\paragraph{Příklady kódů} \begin{itemize}
\item \textbf{Opakovací kód} $(n,1,n)_q$, vznikne opakováním symbolů. Malá dimenze, ale velká vzdálenost.
\item \textbf{Charakteristické vektory přímek KPR} $(X, \mathcal(P))$ řádu n. $(n^2+n+1=|X|,log_2 n^2+n+1 = |\mathcal{P}|, 2n)_2$. $\Sigma=\{0,1\}, C=\{ v \in \{0,1\}^{|X|} \text{ je charakteristický vektor nějaké přímky } p \in \mathcal{P} \}$

Minimální vzdálenost: $n+1$, $n+1$ dvě přímky, průnik v jednom bodě.

Lze přidat $\{1\}^{|X|}$ a doplňky slov.
\item \textbf{Hadamardovy kódy} Jsou tvořeny řádky Hadamardovy matice.  $(n,log_2 2n, \frac{n}2)_2$

\end{itemize}

\paragraph{Sylvesterova konstrukce Hadamardových matic} \[H_1 = (1)\]
\[ H_{2n} = \begin{pmatrix}
H_n & H_n \\
H_n & -H_n
\end{pmatrix} \]

\paragraph{Ekvivalence kódů} Kódy $c$ a $c^\prime$ jsou si ekvivalentní, pokud $\exists \pi$ permutace na $\{1,...,n\}$, kde 
\[ \forall x \in \Sigma^n: x \in C \iff (x_{\pi_1}, x_{\pi_2}, ..., x_{\pi_n}) \in C^\prime \]

\paragraph{Kombinatorická koule} $B(x,t)$ se středem x a poloměrem t je $\{y \in \Sigma^n: d(x,y) \leq t\}$

\paragraph{Objem koule $B(x,t)$} je \[ V(t) = |B(x,t)| = \sum^t_{i=0} \binom{n}i (q-1)^i\]

\paragraph{Perfektní kód} je kód $(n,k,2t+1)_q$ s $\frac{q^n}{V(t)}$ slovy

\paragraph{Lineární kód} je vektorový podprostor prostoru $\mathbb{K}^n$ kde $\mathbb{K}$ je konečné těleso. 
Značíme $[n,k,d]_q$. $|C| = |\mathbb{K}^{\dim(C)}| = |\mathbb{K}^{k}|$

\begin{itemize}
\item Opakovací kódy jsou lineární, $\Sigma = \mathbb{Z}_q$
\item Ch. vektory KPR \textbf{nejsou} lineárními kódy
\item Hadamardovy kódy obecně nejsou, ty z Sylvesterovy konstruce jsou.
\end{itemize}

\paragraph{Generující matice}  lineárního kódu C je $M in \mathbb{F}_q^{K\times n}$, kde řádky jsou vektory nějaké báze C.

Na $\mathbb{F}_q^{K\times n}$ definujeme "skalární součin" pro $x=(x_1,...,x_n), y=(y_1, ..., y_n), x,y \in\mathbb{F}_q^{n}$ jako $\langle x,y \rangle = \sum^n_{i=1} x_i y_i$

\paragraph{Duální kód $C^\perp$} ke kódu C je ortogonální doplněk C.
\[ C^\perp = \{ x \in \mathbb{F}_q^{n} : \langle x,y \rangle = 0 \text{ pro } y \in C \} \]

\paragraph{Kontrolní matice $M^\perp$} je generující matice $C^\perp$

\paragraph{Standardní tvar} BÚNO je Generující matice ve Standardním tvaru, tedy prvních n sloupců tvoří $I_k$:

\[
\begin{array}{|c|c|}
\hline
I_k & B \\
\hline
\end{array}
\]

Pak je kontrolní matice ve tvaru:

\[ M^\perp =
\begin{array}{|c|}
\hline
I_k \\
\hline
-B^T \\
\hline
\end{array}
\]

\paragraph{Syndrom slova} y je $S(y)$, kde $S: \mathbb{F}_q^{n} \to \mathbb{F}_q^{n-k}$ je zobrazení \[ S(y) = M^\perp g\]

\paragraph{Hammingovy kódy} jsou perfektní kódy s parametry $q=2, r \geq 2$ dané generujícimi maticemi
\[ M = \begin{array}{|c|c|}
\hline
I_{2^r-r-1} & \text{řádky}\\
\hline
\end{array}\]

\[ M^\perp = \begin{array}{|c|c|}
\hline
-B^T & I_r \\
\hline
\end{array}\]


\section{Věty}



\subsection{KPR Konečné projektivní roviny}

\begin{tvrzeni}[4.1  O velikosti přímek]
V každé KPR mají všechny přímky stejnou velikost.
\begin{proof}
Mějme $P,Q \in \mathcal{P}, P \neq Q$

Existuje $x \in X: x \notin P, Q$ (Rozepsat důkaz z pomocí A3).

Platí $|P| \leq |Q|$:

Definujeme $\varphi: P \to Q$, kde pro $y \in P$ je $\varphi(y) =$ bod průniku $\overline{xy}$ s Q.

Z A2 plyne $\varphi$ je zobrazení.

Ukažme, že je prosté.

$P = \{y_1,...,y_k\}, z_i = \varphi(y_i)$

Sporem $\exists z_i = z_j$ pro $i \neq j$

potom $\overline{xy_i}, \overline{xy_j}$ obsahují x.

Musí platit $z_i \neq z_j$, jinak by se přímky $\overline{xy_i}, \overline{xy_j}$ protly ve dvou bodech.

Spor s A2 ?

Tedy $|P| \leq |Q|$.

Stejně $|Q| \leq |P|$.

$|P| = |Q|$
\end{proof}
\end{tvrzeni}

\begin{tvrzeni}
Pro KPR $(X, \mathcal{P})$ platí řádu $n$:
\begin{enumerate}
\item $\forall x \in X$: bodem x prochází $n+1$ přímek
\item $|X| = n^2 + n+ 1$
\item $|\mathcal{P}| = n^2 + n+ 1$
\end{enumerate}
\begin{proof}
\begin{enumerate}
\item Mějme $x \in X$, ukážeme, že existuje přímka bez x. 

(A3) $\implies \exists a,b,c,d$ v obecné poloze. BÚNO $x \notin \{a,b,c\}$. Pak $\overline{ab}$ nebo $\overline{ac}$ neobsahují x. Pro každý bod na přímce existuje další přímka, která prochází x.

$\implies$ alespoň $n+1$

(A2) $\implies$ každá přímka s x protíná P v jednom bodě.

$\implies$ nejvýše $n+1$
\item Uvažme $a \in X$ a $P \in \mathcal{P}: a \notin P$

\[ |X| \geq (n+1)(n+1-1) + 1\]

Dále $a \neq x \in X$ x leží na přímce, ta protíná P. Podle (A1) $x \in \overline{ax_i}$ pro nějaké i.

\[ |X| \leq (n+1)(n+1-1) + 1\]

Tedy 

\[ |X| = n^2 + n+1\]

\item Plyne z (2) a z duality
\end{enumerate}
\end{proof}
\end{tvrzeni}

\begin{tvrzeni}[4.3 O duálu KPR]
Duálem KPR řádu n je KRP řádu N
\begin{proof}
Zachování řádu plyne z $|X| = |\mathcal{P}| = n^2+n+1$.

Ověříme axiomy \begin{enumerate}[label=(A\arabic*)*]
\item \[\forall P \neq Q \in \mathcal{P} \exists! L_x = \{ S \in \mathcal{P} | x \in S\}: P,X \in L_X \]
\[\iff  \forall P \neq Q \in \mathcal{P} \exists! x \in X: x \in Q \cap P \iff (A2) \]

\item \[\forall L_x = \{ S \in \mathcal{P} | x \in S\}, L_y = \{ S \in \mathcal{P} | y \in S\} \exists! P \in mathcal{P}: P \in L_x \cap L_y \]
\[\iff \forall x\neq y in X: \exists! P \in \mathcal{P}: \{x,y\} \subseteq Q \iff (A1)\]
\item $(A3) \implies \exists a,b,c,d \in X$ v obecné poloze.
$C^\prime = \{\overline{ab}, \overline{ad}, \overline{bc}, \overline{cd}\}$ je v obecné poloze duálu, každá trojice má dvě přímky, které sdílí bod, který není ve třetí.
\end{enumerate}
\end{proof}
\end{tvrzeni}

\begin{domnenka}
Existuje KPR řádu n, pokud n je mocnina prvočísla.
\end{domnenka}

\begin{veta}[5.1 KPR a tělesa] 
Existuje algebraické těleso s n prvky $implies$ existuje KPR řádu n.
\begin{proof}
$\mathbb{K} = $ těleso s n prvky.

Zavedeme na $\mathbb{K}^3$ ekvivalenci $\sim$, kde $(x,y,z) \sim (\alpha x, \alpha y, \alpha z)$ pro $\alpha \in \mathbb{K}, \alpha \neq 0$.

Body jsou třídy ekvivalence na $\mathbb{K}^3 \setminus \{(0,0,0)\}$

\[|X| = \frac{n^3-1}{n-1}\]

Přímky jsou $\forall (a,b,c) \in mathbb{K}^3 \setminus \{0\}:$
\[ P_{a,b,c} = \{ [(x,y,z)]_\sim | ax+by+cz = 0 \}\]

Pro  $(a,b,c) \sim (a^\prime,b^\prime,c^\prime): P_{a,b,c} = P_{a^\prime,b^\prime,c^\prime}$

$(X, \mathcal{P})$ je KPR:

\begin{enumerate}[label=A\arabic*]
\item mějme $(x,y,z)\nsim (x^\prime,y^\prime,z^\prime) \in X$. Položme 
\[\begin{pmatrix}
x&y&z \\ x^\prime&y^\prime&z^\prime
\end{pmatrix} = M \]
\[ Rank(M) = 2 \implies Rank(Ker(M)) = 1 \]
\item Analogicky přes $\begin{pmatrix}
a&b&c \\ a^\prime&b^\prime&c^\prime
\end{pmatrix}$
\item Stačí zvolit \[ C = \{ [(0,0,1)]_\sim, [(0,1,0)]_\sim, [(0,0,1)]_\sim, [(1,1,1)]_\sim,\}\]
\end{enumerate}
\end{proof}
\end{veta}

\subsection{Latinské čtverce}

\begin{pozorovani}
Pro $L$ a $L^\prime:$
\[ \forall l,l^\prime \in [n]: \exists! i,j: L_{i,j} = l \wedge L^\prime = l^\prime \]
\end{pozorovani}

\begin{pozorovani}
Permutace latinského čtverce je latinský čtverec. Stejné permutace dvou ortogonálních čtverců jsou ortogonální. Speciálně, odteď mají všechny latinské čtverce první řádek 1,2,3,...
\end{pozorovani}

\begin{pozorovani}
Navzájem ortogonálních latinských čtverců řádu $n$ je nejvýše $n-1$.
\begin{proof}
Sporem triv.
\end{proof}
\end{pozorovani}

\begin{veta}
\[\forall n \geq 2: \exists KPR \text{ řádu } n \iff \exists n - 1 \text{ NOLČ řádu }n\]
\begin{proof}
\begin{itemize} $\implies$

Uvažme přímku $P = \{ y_0, ..., y_n \}$, $\forall y_i$ prochází bodem přímky $P_{i_i}, P_{i_n}$

Buď $\{X_{i,j} = P_{0_i} \cap P_{n_j}\}$.

Pro $k=1,...,n-1$ definujeme
\[ {L_k}_{i,j} = l \iff X_{i,j} \in P_{k,l}\]

Tyto čtverce jsou NOLČ. \begin{itemize}
\item Každé pole má přiřazené číslo: Podle (A1) $y_k, x_{i,j}$ musí procházet nějaká přímka.
\item V řádcích nejsou opakování. Sporem \[\exists i,j \neq j^\prime,l (L_k)_{i,j}= (L_k)_{i,j^\prime} = l\]
\[ \implies x_{i,j}= x_{i,j^\prime} \in P_{k,l} \implies |P_{k,l} \cap P_{0,i} | > 1\]
Spor s A2
\item Analogicky pro sloupce. 
\item Ortogonalita: Pro $k\neq k^\prime, l \neq l^\prime$ chceme $i,j$...

Přímky $P_{k,l}, P_{k^\prime,l^\prime}$. Tyto přímky mají nutně průnik, $exists x_{i,j}$.
\end{itemize}
\item $\impliedby$
TODO
\end{itemize}
\end{proof}
\end{veta}

\subsection{Grafy}

\begin{veta}[König-Egerváry]
V bipartitním grafu je velikost maximálního párování rovna velikosti minimálního vrcholového pokrytí.
\begin{proof}
\end{proof}
\end{veta}

\begin{veta}[7.2 Hallova]
Množinový systém $\mathcal{M}$ pro I,X má systém různých reprezentantů $\iff \forall J \subseteq I: |J| \leq \left|\bigcup_{j \in J} M_j \right|$
\begin{proof}
\begin{itemize}
\item $\implies$ mějme f SRR a zvolme $J \subseteq I$
\[ \left|\bigcup_{j \in J} M_j \right| \overset{\supseteq}{\geq}  |\{ f(j) | j \in J\}| \overset{\text{f je prosté}}{=} |J| \]

\item $\impliedby$
\end{itemize}
\end{proof}
\end{veta}

\begin{veta}[7.3]
V každém bipartitiním grafu $(A \cup B, E)$ kde $\deg(a) \geq deg(b)$ a $E \neq \emptyset$
exxistuje párování velikosti $|A|$
\begin{proof}

\end{proof}
\end{veta}

\begin{veta} [7.4]
Každý latinský obdelník typu $k \times n$ lze doplnit na čtverec.
\begin{proof}
Mějme obdelník. S sloupce, H hodnoty. Definujeme $G = (S \cup H, E)$, kde $\{ S_{i,j} \} \in E \iff j$ není ve sloupci $i$.

Každý sloupec má n-k hran, $\deg(S_i) = k-n$ Každá hodnota má $deg(n) = n-k$.

Podle věty o párování v bipartitním grafu existuje párovnání velikosti n. Indukcí dotvoříme čtverec.
\end{proof}
\end{veta}

\subsection{Kódy}

\newcommand*{\maxerrors}{\lfloor \frac{d-1}2 \rfloor}

\begin{veta}[12.0.1 Charakteristika kódu]
Za pomoci kódu c jsme schopni opravit $\leq t$ chyb, pokud $\forall y \in \Sigma^n$ existuje nanejvýš jedno $x \in C$, t.ž. $d(x,y) \leq t$
\end{veta}

\begin{veta}[12.0.2 Charakteristika kódu]
S kódem $(n,K,d)_q$ jsme schopni opravit nanejvýš $\maxerrors$ chyb.
\begin{proof}
Množiny slov ve vzdálenosti $\leq \maxerrors$ od kódových slov jsou disjunktní.
\end{proof}
\end{veta}

\begin{veta}[Hadamardova domněnka]
\[\forall n, 4|n: \exists H_n \]
\end{veta}

\begin{pozorovani}[12.1 (Koule kódových slov o malém poloměru jsou disjunktní)] 
$\forall$ kód se vzdáleností $2t+1, \forall x,y \in C, x \neq y$
\[ B(x,t) \cap B(y,t) = \emptyset 
\]
\begin{proof}
Sporem: $ \exists z \in  B(x,t) \cap B(y,t) $
\[ d(x,y) \leq d(x,z)+d(z,y) \leq t + t = 2t < 2t+1 \]
Spor s minimální vzdáleností.
\end{proof}
\end{pozorovani}

\begin{veta}[12.2 Hammingův odhad]
$\forall$ kód C s parametry $(n,k,2t+1)_q$ platí \[ |C| \leq \frac{q^n}{V(t)} \]
\begin{proof}
Máme $q^n$ slov. Podle pozorování platí dvě koule $B(x,t)$ jsou disjunktní pro různá $x \in C$. 
\[ \bigcup_{x \in C} B(x,t) \subseteq \Sigma^n \implies |C|V(t) \leq |\Sigma^n| = q^n \]
Úpravou dostaneme požadovaný odhad.
\end{proof}
\end{veta}

\begin{veta}[12.3 Gilbert;v-Varschamův odhad]
$\forall n,d,q \in \mathbb{N}: \exists$ kód C s parametry $|C| \geq \frac{q^n}{V(d-1)}$
\begin{proof}
C vyrobíme hladově z $\Sigma^n$ odebíráním kódových slov s příslušnými koulemi. Toto lze iterovat $ \frac{q^n}{V(d-1)}$ krát.
C je kód.
\end{proof}
\end{veta}

\begin{pozorovani}[12.4 minimální vzdálenost lineárních kódů] 
\[d = \min_{x \in C, x \neq 0} d(0,x) \]
\end{pozorovani}




\end{document}