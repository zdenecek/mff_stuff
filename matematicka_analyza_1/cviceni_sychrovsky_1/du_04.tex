\documentclass[10pt,a4paper]{article}
\usepackage[utf8]{inputenc}
\usepackage{amsmath, amsfonts, amssymb, amsthm}
\usepackage{mathtools, array, enumitem, xcolor}
\usepackage[margin=0.7in]{geometry}
\setlength{\parindent}{0em}

\usepackage[czech]{babel}

\theoremstyle{plain}
\newtheorem{veta}{Věta}
\theoremstyle{definition}
\newtheorem{definice}[veta]{Definice}


\title{Řešení domácího úkolu z MA č. 4}
\author{přezdívka: Zdeněk}
\date{}

\begin{document}

\maketitle

\section{}

Odstraníme odmocniny.

\[ \lim_{n \to \infty} n(\sqrt{n^2 + 2n} - 2\sqrt{n^2 + n} + n) =
 \lim_{n \to \infty} \frac{n(\sqrt{n^2 + 2n} + n - 2\sqrt{n^2 + n} )(\sqrt{n^2 + 2n}+ n + 2\sqrt{n^2 + n} )}{\sqrt{n^2 + 2n}+ n + 2\sqrt{n^2 + n}}
\] \[
 = \lim_{n \to \infty} \frac{n(n^2 + 2n + 2n\sqrt{n^2 + 2n} + n^2 - 4n^2 - 4n )}{\sqrt{n^2 + 2n}+ n + 2\sqrt{n^2 + n}}
 =  \lim_{n \to \infty} \frac{-2n(n + 1 - \sqrt{n^2 + 2n})}{\sqrt{1 + \frac2n}+ 1 + 2\sqrt{1 + \frac1n}}
 \] \[
 =  \lim_{n \to \infty} \frac{-2n(n + 1 - \sqrt{n^2 + 2n})}{\sqrt{1 + \frac2n}+ 1 + 2\sqrt{1 + \frac1n}} \cdot \frac{n + 1 + \sqrt{n^2 + 2n}}{n + 1 + \sqrt{n^2 + 2n}}
 = \lim_{n \to \infty} \left( \frac{-2n(n^2 + 2n + 1 - n^2 - 2n)}{\sqrt{1 + \frac2n}+ 1 + 2\sqrt{1 + \frac1n}} \cdot \frac{1}{n\left(1 + \frac1n + \sqrt{1 + \frac2n}\right)}
 \right) \]
\[ = \frac{-2}{\left(\sqrt{1} + 1 + 2\sqrt{1} \right)\left( 1 + 0 + \sqrt1 \right)} = -\frac14\]

\section{}

\[\lim_{n \to \infty} \frac{9n + \lfloor \sqrt[3]{n} \rfloor^3}{2n - \lfloor \sqrt{5n+5} \rfloor}\]

Použijeme větu o dvou policajtech.

Zmenšením jmenovatele a zvětšením čitatele zlomek vždy zvětším.


\[ \lfloor \sqrt[3]{n} \rfloor^3 \leq  n  \wedge  - \lfloor \sqrt{5n+5} \rfloor \geq - \sqrt{5n+5} \implies \frac{9n + \lfloor \sqrt[3]{n} \rfloor^3}{2n - \lfloor \sqrt{5n+5} \rfloor} \leq \frac{10n}{2n -  \sqrt{5n+5}}  \]
\[ \lim_{n \to \infty} \frac{10n}{2n -  \sqrt{5n+5}} = \lim_{n \to \infty} 
\frac{10}{2 - \sqrt{\frac5n+ \frac5{n^2}}} = \frac{10}2 = 5\]

A druhý, smutnější policista:

\[
\lfloor \sqrt[3]{n} \rfloor^3 \geq  (\sqrt[3]{n}- 5 )^3  \wedge  - \lfloor \sqrt{5n+5} \rfloor \leq - \sqrt{5n+5} + 5 \implies \frac{9n + \lfloor \sqrt[3]{n} \rfloor^3}{2n - \lfloor \sqrt{5n+5} \rfloor} \geq \frac{9n + (\sqrt[3]{n}-5)^3}{2n -  \sqrt{5n+5} + 5}
\]
\[ \lim_{n \to \infty} \frac{9n + (\sqrt[3]{n}-5)^3}{2n -  \sqrt{5n+5} +5} =
\lim_{n \to \infty} \frac{9n + n -15n^{\frac23} + 75n^{\frac13} - 125}{2n + 5 - \sqrt{5n+5} } = \frac{10 -15n^{-\frac13} + 75n^{-\frac23} - \frac{125}n}{2 + \frac5n - \sqrt{\frac5n+\frac5{n^2} }}  = \frac{10}2 = 5 \]

A tedy

\[ 
\lim_{n \to \infty} \frac{9n + (\sqrt[3]{n}-5)^3}{2n -  \sqrt{5n+5} + 5} 
\leq 
\lim_{n \to \infty} \frac{9n + \lfloor \sqrt[3]{n} \rfloor^3}{2n - \lfloor \sqrt{5n+5} \rfloor} 
\leq 
\lim_{n \to \infty} \frac{10n}{2n -  \sqrt{5n+5}}\]
\[
5
\leq 
\lim_{n \to \infty} \frac{9n + \lfloor \sqrt[3]{n} \rfloor^3}{2n - \lfloor \sqrt{5n+5} \rfloor} 
\leq
5
\implies \lim_{n \to \infty} \frac{9n + \lfloor \sqrt[3]{n} \rfloor^3}{2n - \lfloor \sqrt{5n+5} \rfloor}  = 5 \]


\end{document}