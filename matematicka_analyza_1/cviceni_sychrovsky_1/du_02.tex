\documentclass[10pt,a4paper]{article}
\usepackage[utf8]{inputenc}
\usepackage{amsmath, amsfonts, amssymb, amsthm}
\usepackage{mathtools, array, enumitem, xcolor}
\usepackage[margin=0.7in]{geometry}
\setlength{\parindent}{0em}

\usepackage[czech]{babel}

\theoremstyle{plain}
\newtheorem{veta}{Věta}
\theoremstyle{definition}
\newtheorem{definice}[veta]{Definice}


\title{Řešení domácího úkolu z MA č. 2}
\author{přezdívka: Zdeněk}
\date{}

\begin{document}

\maketitle

\section{Najděte supremum $\{ 0.5, 0.55, ... \}$ v $\mathbb{Q}$
}

Uvažme číslo $a = 0.\overline{5} = 0.555555...$

\begin{enumerate}
\item
Je $a$ horní závorou?

Ano, uvažme libovolný $n$-tý prvek posloupnosti. Ten bude mít na $n-1$-ní pozici nulu, zatímco v $a$ tam bude pětka, tudíž $\forall i: a > m_i$

\item 
Je $a$ nejmenší horní závorou?

Ano, uvažme sporem nějakou menší závoru $b$, ta se musí lišit alespoň v jedné cifře, a aby byla menší, musí být první rozdílná cifra menší než 5 (nebo $b$ může být záporné, ale to vyloučíme). Řekněme třeba, že první menší cifra bude $n$-tá cifra. Potom ale $m_n$ bude větší a $b$ tedy není horní závora.

\item
Je $a$ racionální?

Vyjáření $a$ můžeme nalézt klasickým trikem, který využívá axiomů tělesa (ekvivalentní úpravy):
\begin{align*}
10a &= 5.\overline{5} \\
10a - a &= 5.\overline{5} - 0.\overline{5} \\
9a &= 5 \\
a &= \frac{5}9 \\
\end{align*}

Ano, tedy $a$ je hledaným supremem.
\end{enumerate}

\section{$\phi, \varphi, \psi$}

Inverzní funkce $\psi^{-1}$ bude existovat právě tehdy, pokud je $\psi$ prostá 

Uvažme tedy $x,y, x \neq y$. 

Protože je $\varphi$ bijekce, je taky prostá, takže $x \neq y \implies \varphi(x) \neq \varphi(y)$

Podívejme se, jestli pro různé $\varphi$ může výraz nabývat stejných hodnot

\[ \sqrt{\varphi^2(x) - 1} = \sqrt{\varphi^2(y) - 1} \]

Obě strany můžeme umocnit (vyjdeme z axiomů - vynásobíme jedním z výrazů a na jedné straně pak výraz zaměníme za druhý podle původní rovnice)

\[{\varphi^2(x) - 1} = {\varphi^2(y) - 1} \]
\[\varphi^2(x) = \varphi^2(y) \]

Abychom se vyhli přímému odmocnění, výraz přeházíme:

\[ \frac{\varphi^2(x)}{\varphi^2(y)} = 1 \]
\[ \left(\frac{\varphi(x)}{\varphi(y)}\right)^2 = 1 \]

Zde vidíme, že zlomek musí nabývat hodnot $\pm 1$. Aby byl ale záporný, musí být alespoň jeden z dvojice čitatel, jmenovatel záporný, což nemůže nastat, protože $\varphi$ nabývá hodnot $[1, \infty)$.

\[\frac{\varphi(x)}{\varphi(y)} = 1 \implies \varphi(y) = \varphi(x)\]

Máme tedy \[\sqrt{\varphi^2(x) - 1} = \sqrt{\varphi^2(y) - 1} \implies \varphi(y) = \varphi(x) \implies x = y \]

a vidíme, že funkce je prostá.

Vyjádřeme její inverzi:

\begin{align*}
\psi(x) &= \sqrt{\varphi^2(x) - 1} \\
\psi^2(x) &= \varphi^2(x) - 1  \\
\psi^2(x) + 1 &= \varphi^2(x)   \\
\sqrt{\psi^2(x) + 1} &= |\varphi(x)|   \\
\sqrt{\psi^2(x) + 1} &= \varphi(x)  \\
\varphi^{-1}\left(\sqrt{\psi^{2}(x) + 1}\right) &= x   \\
\varphi^{-1}\left(\sqrt{x^2 + 1}\right) &= \psi^{-1}(x)   \\
\end{align*}

Absolutní hodnotu lze odstranit, protože $\varphi$ nabývá pouze nezáporných hodnot. Inverzi $\varphi$ můžeme použít, protože se jedná o prostou funkci.


Její definiční obor bude celé těleso, jelikož hodnota pod odmocninou bude $\geq 1$, tedy i celá odmocnina bude $\geq 1$ a definičním obor $\varphi^{-1}$ jsou všechna čísla $x \geq 1$, což odpovídá.

\section{Nový příklad}

Upravíme výraz:

\[ \frac{p}{p+q} = \frac{p+q -q}{p+q}= 1 - \frac{q}{p+q} \]

Pro každé q se jedná o rostoucí funkci, tudíž bude mít minimum, a to v bodě $p = 1$, stačí prověřit pro které $q$, pro obecné $q$ si to můžeme opět upravit:

\[1-\frac{q}{1+q}=1-\frac{q+1-1}{1+q}=\frac{1}{1+q}\]

To je klesající funkce, takže minimum bude v bodě $q = 5$

\[ \min N = \sup N = \frac16 \]

Funkce je rostoucí, tudíž maximum neexistuje: Ukažme sporem:

Nechť $\max N = m = 1 - \frac{q}{p+q}$

Zafixujeme $q$ a podívejme se na člen, kde $p^\prime = p+1$

Aby $m$ bylo maximum, musí platit:

\[ 1 - \frac{q}{p+1+q} \leq 1 - \frac{q}{p+q} \]
\[ \frac{1}{p+1+q} \geq  \frac{1}{p+q} \]
\[ {p+1+q} \leq  {p+q} \]
\[ 1 \leq  0 \]

To je spor.

Maximum tedy neexistuje. 

Horní mez je určite pro každé $q$ jednička, protože odečítaný člen je vždy kladný.

Je to i supremum?

Uvažme, že existuje menší supremum:

\[\sup N = 1 - \epsilon\]

Pro zafixované $q = 1$ nalezněme větší prvek množiny pro spor:

\[ 1 - \frac{1}{p+1} > 1 - \epsilon\]
\[  \frac{1}{p+1} <  \epsilon\]
\[  1 <  \epsilon p+ \epsilon \]
\[  \frac{1 - \epsilon}{\epsilon} < p \]

No ale pro libovolné číslo jsme vždy schopni nalézt větší přirozené čislo, a to třeba tak, že číslo zaokrouhlíme nahoru a přičteme jedničku, tudíž menší supremum opravdu neexistuje.

\[ \sup N = 1 \] 

\[ \max N \text{ neexistuje} \]

\[\min N = \frac16 \]  

\[\inf N = \min N = \frac16 \]






\end{document}