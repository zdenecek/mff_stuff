\documentclass[10pt,a4paper]{article}
\usepackage[utf8]{inputenc}
\usepackage{amsmath, amsfonts, amssymb, amsthm}
\usepackage{mathtools, array, enumitem, xcolor}
\usepackage[margin=0.7in]{geometry}
\setlength{\parindent}{0em}

\usepackage[czech]{babel}

\theoremstyle{plain}
\newtheorem{veta}{Věta}
\theoremstyle{definition}
\newtheorem{definice}[veta]{Definice}


\title{Řešení domácího úkolu z MA č. 5}
\author{přezdívka: Zdeněk}
\date{}

\begin{document}

\maketitle

\section{}

\begin{enumerate}[label=(\alph*)]
\item $\sum_{n=1}^\infty (-1)^n \frac{x^{2n+1}}{
2n + 1}$

\begin{enumerate}[label=(\roman*)]
\item Pro $|x| > 1$ Řada nesplňuje konvergenční kritérium, ani pro absolutní konvergenci, tudíž diverguje.
\item Pro $|x| = 1$. BÚNO $x=1$ (jinak se pouze změní znaménko, které vytkneme)


\[\sum_{n=1}^\infty  \frac{(-1)^n}{2n + 1} \] 

Použijeme \textbf{Dirichletovo kritérium}: 
 
$\sum^n (-1)^n$ má omezené částečné součty a $\lim_{n \to \infty} \frac1{2n + 1} = 0$, tedy řada konverguje.

\hfill

Absolutně řada konvergentní není:

\[\sum_{n=1}^\infty  \frac{1}{2n + 1} > \sum_{n=1}^\infty  \frac{1}{3n} = \frac13 \sum_{n=1}^\infty  \frac{1}{n} \] 

Jedná se o harmonickou řadu s exponentem 1, která diverguje.

\item Pro $|x| \in (0;1)$

\[\sum_{n=1}^\infty  (-1)^n \frac{x^{2n+1}}{2n + 1}\]

Použijeme opět \textbf{Dirichletovo kritérium}:

$\sum^n (-1)^n$ má omezené částečné součty a
\[ \lim_{n \to \infty} \frac{x^{2n+1}}{2n + 1} = \frac0\infty = 0 \] 
 tedy řada konverguje.
 
\hfill
 
 Pro absolutní konvergenci použijeme \textbf{d'Alambertovo podílové kritérium}:
 
 \[\sum_{n=1}^\infty  \frac{x^{2n+1}}{2n + 1}\]
 
 \[\limsup_{n \to \infty} \frac{\frac{x^{2(n+1)+1}}{2(n+1) + 1}}{\frac{x^{2n+1}}{2n + 1}}
 = \limsup_{n \to \infty} \frac{x^{2n+3}}{x^{2n+1}} \frac{2n+1}{2n + 3} = x^2 \lim_{n \to \infty} (1 - \frac2{2n+3}) 
 = x^2 < 1\]
  
Řada konverguje i absolutně.

\item Pro $x = 0$

Řada triviálně konverguje v nule, i absolutně.

\end{enumerate}

\item \[\sum_{n=1}^\infty  \left(\frac{2+\cos(n)}{3 + \cos(n)}\right)^{2n - \ln(n)} \]

Předpokládejme, že řada konverguje

Zlomek je vždy kladný a menší než jedna, nejvíce ho zvětšíme, když nahradíme cosinus maximem. Navíc umocněním na vyšší exponent sumu zmenšíme, zmenšením exponentu ji tedy zvětšíme. Odhadněme tedy shora $ \ln(n) <n$
\[\sum_{n=1}^\infty  \left(\frac{2+\cos(n)}{3 + \cos(n)}\right)^{2n - \ln(n)} \leq 
\sum_{n=1}^\infty  \left(\frac{2+1}{3+1}\right)^{2n - ln(n)} \leq \sum_{n=1}^\infty  \left(\frac34\right)^{2n - n} 
=\sum_{n=1}^\infty  \left(\frac34\right)^{n}  \]

Dle \textbf{Cauchyho odmocninového kritéria} tato řada konverguje, takže podle srovnávacího kritéria konverguje i řada původní.

\end{enumerate}

\section{}

\[ \sum_{n=1}^\infty \frac{cos(nx) - cos((n + 1)x)}{n}\]

Čitatel může být i záporný a chová se nepředvídatelně pro netriviální $x$.

Kdybychom dokázali, že posloupnost tvořená čitatelem má omezené součty, mohli bychom použít \textbf{Dirichletovo kritérium}.

Ve skutečnosti má taková řada opravdu omezené součty (minimálně jsem to teda ověřil ve Wolframu), ale nenapadá mě, jak to ukázat.

Můžeme to odhadnout a říci, že řada konverguje.

\section{Bonus}

\begin{enumerate}
\item \[\lim_{h \to 0} \frac{a^{x+h} - a^x}h = a^x \]
\[\lim_{h \to 0} \frac{a^{h} - 1}h = 1 \]
\item \[\lim_{h \to 0} \frac{(\sqrt[h]{h+1+o(h)})^{h}  - 1 }h  =\lim_{h \to 0} \frac{o(h)}h + 1 \]

To se má rovnat jedné, tedy $\lim_{h \to 0} \frac{o(h)}h$ se má rovnat nule.

\item \[ e= \lim_{h \to 0} a(h) = \lim_{n \to \infty} a(\frac1{n}) = \lim_{n \to \infty} \sqrt[\frac1n]{\frac1n+1+o\left(\frac1n\right)} 
= \lim_{n \to \infty} \left(\frac1n+1+o\left(\frac1n\right)\right)^n \]

\item \[  \lim_{n \to \infty} \left(\frac1n+1\right)^n 
= \lim_{n \to \infty} \binom{n}{0} + \binom{n}{1}\frac1n 
+   \binom{n}{2}\frac1{n^2}  +  \binom{n}{3}\frac1{n^3}   + ... 
\]\[
= \lim_{n \to \infty} \frac{n!}{n!} + \frac{n!}{(n-1)!n} 
+   \frac{n!}{(n-2)!n^2 2!} +  \frac{n!}{(n-3)!n^3 3!}  + ...\]

Všimněme si, že v každém zlomku má polynom v čitateli stejný stupeň jako ve jmenovateli a kromě faktoru $k!$ mají členy nejvyššího stupně v obou polynomech koeficient 1, tudíž je můžeme vytknout člen nejvyššího stupně, zbytek pominout jako nulu a zkrátit na 1.

Neboli: \[
\lim_{n \to \infty} \frac{n!}{(n-k)!n^k k!} 
= \lim_{n \to \infty} \frac{n(n-1) ... (n-k+1)}{n^k k!}
= \lim_{n \to \infty} \frac{n(n-1) ... (n-k+1)}{n^k} \cdot   \lim_{n \to \infty} \frac1{k!} \]

A zde můžeme provést ono zmíněné zkrácení a první limitu spočítat na 1, protože jediný člen s exponentem k v čitateli bude mít koeficient 1.

Tedy

\[ e = \lim_{n \to \infty} \sum^n_{k=0} \frac1{k!}\]

\item

Najděme, kolik členů budeme muset sečíst:

\[ \sum^\infty_{k=x} \frac1{k!} \leq \frac14\]

Použijeme odhad:

\[ k! \geq \left(\frac{k}2\right)^\frac{k}{2} \implies \sum^\infty_{k=x} \frac1{\left(\frac{k}2\right)^\frac{k}{2}} \geq \sum^\infty_{k=x} \frac1{k!} \]

\[ \sum^\infty_{j=\frac{x}2} \frac1{j^j} \leq 
\sum^\infty_{j=\frac{x}2} \frac1{2^j}= 2^{-\frac{x}2}\frac{1}{1-\frac12} = 2^{1-\frac{x}2} \]

\[ 2^{1-\frac{x}2} < \frac14 \implies 2^{3-\frac{x}2} < 1 \implies 3 < \frac{x}2 \implies 6 < x  \]

Sečteme tedy prvních šest členů, chybu zanedbáním zbylých členů jsme odhadli na nižší než požadovanou.

Výsledek jsem spočítal: 

\[ 1 + 1 + \frac12 + \frac16 + \frac1{24} + \frac1{120} = \frac{163}{60} = 2 + \frac23 + \frac1{20} = 2.\overline{6} + 0.05 =   2.71\overline{6} \]


\end{enumerate}

\end{document}