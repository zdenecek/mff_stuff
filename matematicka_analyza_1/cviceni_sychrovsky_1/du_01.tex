\documentclass[10pt,a4paper]{article}
\usepackage[utf8]{inputenc}
\usepackage{amsmath, amsfonts, amssymb, amsthm}
\usepackage{mathtools, array, enumitem, xcolor}
\usepackage[margin=0.7in]{geometry}
\setlength{\parindent}{0em}

\usepackage[czech]{babel}

\theoremstyle{plain}
\newtheorem{veta}{Věta}
\theoremstyle{definition}
\newtheorem{definice}[veta]{Definice}


\title{Řešení domácího úkolu z MA č. 1}
\author{přezdívka: Zdeněk}
\date{}

\begin{document}

\maketitle

\section{}

$0 = a + b$

$1 = 2a + b$

Odečtením rovnic

$a = 1$

Dosazením

$b = -1$


\section{}

Tvrzení \[ \forall n \in \mathbb{N}_0 \exists k \in \mathbb{N}_0 :  n^3 - n = 6k \]

\begin{proof} Indukcí.

\begin{itemize}
\item Základní případy $n \in \{0, 1\}$
Za $k$ můžeme dosadit v obou případech nulu.
\item Indukční krok: Předpokládejme, že tvrzení platí pro $n$, a rozeberme platnost pro $n+2$:

Rozepišme rovnici:


\[ (n+2)^3 - (n + 2) = 6k \]

\[ n^3 + 6n^2 + 12n + 8 - n - 2 = 6k \]

\[ (n^3 - n) + 6(n^2 + 2n + 1) = 6k \]

Využijeme indukční předpoklad:

\[ 6k^\prime + 6(n^2 + 2n + 1) = 6k \]
\[ k = k^\prime + n^2 + 2n + 1  \]


\end{itemize}


Pro každé $n$ se můžeme indukčními kroky dostat na základní případ, tedy tvrzení platí. 

\end{proof}


\end{document}