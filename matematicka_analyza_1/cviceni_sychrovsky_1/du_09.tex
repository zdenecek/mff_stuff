\documentclass[10pt,a4paper]{article}
\usepackage[utf8]{inputenc}
\usepackage{amsmath, amsfonts, amssymb, amsthm}
\usepackage{mathtools, array, enumitem, xcolor}
\usepackage[margin=0.7in]{geometry}
\setlength{\parindent}{0em}

\usepackage[czech]{babel}

\theoremstyle{plain}
\newtheorem{veta}{Věta}
\theoremstyle{definition}
\newtheorem{definice}[veta]{Definice}


\title{Řešení domácího úkolu z MA č. 9}
\author{přezdívka: Zdeněk}
\date{}

\begin{document}

\maketitle

\section{}

\[ \sqrt[3]{26} = \sqrt[3]{27-1} = 3\sqrt[3]{1 - \frac1{27}}\]

Použijeme Taylorův rozvoj v bodě $0$ pro funkci $\sqrt[3]{1-x}$:

\[ \sqrt[3]{1-x} = 1 - \frac{x}3 - \frac{x^2}{9} + o(x^2) \]

Chyba pro dané $x$: 

\[ R_2 = \frac{10}{27 \cdot 3!}\frac{1}{27^3} = \frac{5}{3^13} < \frac{1}{3^{11}} < \frac{1}{3 \cdot 1000}  \]

Chyba je dostatečeně malá, rozvoj třetího stupně stačí.

\[  \sqrt[3]{1-\frac{1}{27}} \approx 1 - \frac1{3^4} - \frac{1}{3^8} = 
 \frac{3^8 - 3^4 - 1}{3^8} = \frac{6479}{6561} \]
 
 \[ \sqrt[3]{26} \approx 2.962  \]

\section{}

\begin{enumerate}[label=|\alph*|]
\item Mohli bychom použít Taylorův rozvoj, ale pro jednoduchost použijme l'Hospitalovo pravidlo


\[ \lim_{x \to 0} \frac{e^{3x} -\sin x - \cos x + ln(1-2x)}{\cos(5x)-1} \overset{l'H}{=}
 \lim_{x \to 0} \frac{3e^{3x} - \cos x + \sin x - \frac{2}{1-2x}}{-5\sin(5x)}\]\[
 \overset{l'H}{=}
 \lim_{x \to 0} \frac{9e^{3x} + \sin x + \cos x - \frac{4}{(1-2x)^2}}{-25 \cos(5x)}  = - \frac{9+1-4}{25} = -\frac{6}{25} \]
 
 \item 
 
 \[ \lim_{x \to 0} \left( \frac{1}{x^2} - \frac{1}{\sin^2 x} \right) 
  = \lim_{x \to 0} \frac{\sin^2x - x^2}{x^2\sin^2x} 
\]

Zde použijeme Taylorův rozvoj

\[ \sin x = x - \frac{x^3}{3!} + \frac{x^5}{5!} + ... \]
\[ \sin^2 x = x^2 - 2\frac{x^4}{3!} + \left( \frac{2}{5!} + \frac{1}{6!} \right)x ^6 + o(x^7) \]

Tedy

\[  \lim_{x \to 0} \frac{- 2\frac{x^4}{3!} + \left( \frac{2}{5!} + \frac{1}{6!} \right)x ^6 + o(x^7)}{ x^4 - 2\frac{x^6}{3!} + o(x^7)  } = - \frac13 \]
\end{enumerate}

\section{}

\[ p(0) = \frac{5\sqrt{3}}{2} \approx 4,330 \]

\[ p^{(1)}(t) =  5 \cos\left(\frac{\pi}{3}\sqrt{t+1}\right)\frac{\pi}{6}\frac1{\sqrt{t+1}} - \frac1{t^2 + 1} \]

\[ p(1) = p(0) +  \frac{5\pi\sqrt{2}}{12} \cos\left(\frac{\sqrt{2}\pi}{3}\right) - \frac12 =  \frac{30\sqrt{3} + 5\pi\sqrt{2} \cos\left(\frac{\sqrt{2}\pi}{3}\right) - 6 }{12} \]\[
\approx 4
\]

Papír se nevyplatí koupit.

\end{document}