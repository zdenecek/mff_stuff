\documentclass[10pt,a4paper]{article}
\usepackage[utf8]{inputenc}
\usepackage{amsmath, amsfonts, amssymb, amsthm}
\usepackage{mathtools, array, enumitem, xcolor}
\usepackage[margin=0.7in]{geometry}
\setlength{\parindent}{0em}

\usepackage{tikz}
\usepackage{pgfplots}
\pgfplotsset{compat=1.11}

\usepackage[czech]{babel}

\theoremstyle{plain}
\newtheorem{veta}{Věta}
\theoremstyle{definition}
\newtheorem{definice}[veta]{Definice}


\title{Řešení domácího úkolu z MA č. 8}
\author{přezdívka: Zdeněk}
\date{}

\begin{document}

\maketitle

\section{}

Poznámka: Obrázky jsou kresleny pomocí latexu, nejsou to ale grafy daných funkcí, pouze odhady pomocí bodů, které jsem tam ručně zadal, zbytek je automaticky dokreslená vyhlazená křivka.

\begin{enumerate}[label=(\alph*)	]
\item $\frac1{1+e^{-x}}$

Funkce je definována na celém intervalu. 

\[ \frac{d}{dx} \frac1{1+e^{-x}} = \frac{e^{-x}}{(1+e^{-x})^2}  \]
\[ \frac{e^{-x}}{(1+e^{-x})^2} = 0 \]
\[ e^{-x} = (1+e^{-x})^2 =  1 + 2e^{-x} + e^{-2x} \]
\[ 0 =  1 + e^{-x} + (e^{-x})^2 \]

Extrémy neexistují.

Jedná se o rostoucí funkci funkci (klesající funkce v čitateli), která je kladná.

\begin{figure}[h!]
  \begin{center}
    \begin{tikzpicture}
    
	\begin{axis}	[
	grid=both,
	ymin=0,
	ymax=2,
	xmax=2,
	xmin=-2,
	minor tick num=1,
	axis lines = middle,
	xlabel=$x$,
	ylabel=$y$,
	label style = {at={(ticklabel cs:1.1)}}]
	\addplot[blue, smooth] plot coordinates {
		(-2,0.2)
		(0,0.5)
		(2,1.5)	
	};
    \end{axis}         
    \end{tikzpicture}
    \caption{Funkce 1}
  \end{center}
\end{figure}

\item $g(x) = \frac{x^2e^{-x}}{1+e^{-x}}$

\[ \frac{dg}{dx} 
= \frac{(x^2e^{-x})^\prime(1+e^{-x}) - (x^2e^{-x})(1+e^{-x})^\prime }{(1+e^{-x})^2} 
=  \frac{(2xe^{-x} - x^2e^{-x})(1+e^{-x}) - (x^2e^{-x})(-e^{-x}) }{(1+e^{-x})^2}  \]\[
= \frac{2xe^{-x} - x^2e^{-x} + 2x(e^{-x})^2 - x^2(e^{-x})^2 + x^2(e^{-x})^2 }{(1+e^{-x})^2} 
\]\[
= \frac{2xe^{-x} - x^2e^{-x} + 2x(e^{-x})^2}{(1+e^{-x})^2} 
= \frac{xe^{-x}(2 - x + 2e^{-x})}{(1+e^{-x})^2} 
\]

Využijeme poznatku ze zadání: funkce nabývá nuly v $\sim 2.22$ a $0$

\[  g(0) = 0 \]
\[  g(2.2) \approx g(2) \approx \frac4{1 + e^2} \approx \frac12 \]

\[ \lim_{x \to \infty } g(x) = \lim_{x \to \infty } \frac1{1+e^{-x}} \cdot \lim_{x \to \infty } x^2 e^{-x} = \lim_{x \to \infty } \frac{x^2}{e^x} = 0  \]

\[ \lim_{x \to -\infty } g(x) = \lim_{x \to -\infty } \frac{ x^2 e^{-x}}{1+e^{-x}} \overset{L'H}{=} \lim_{x \to -\infty } \frac{ 2x e^{-x} -x^2 e^{-x} }{-e^{-x}} =  \lim_{x \to -\infty } x^2 - 2x = \infty  \]


Graf:


\begin{figure}[h!]
  \begin{center}
    \begin{tikzpicture}
    
	\begin{axis}	[
	grid=both,
	ymin=0,
	ymax=5,
	xmax=10,
	xmin=-10,
	minor tick num=1,
	axis lines = middle,
	xlabel=$x$,
	ylabel=$y$,
	label style = {at={(ticklabel cs:1.1)}}]
	\addplot[red, smooth] plot coordinates {
		
		(-10,5)
		(-4,2)
		(-0.3,0.05)
		(2.2,0.5)
		(10, 0.3)	
	};
    \end{axis}         
    \end{tikzpicture}
    \caption{Funkce 2}
  \end{center}
\end{figure}

\item $h(x) = x \ln(x)$

Definiční obor je omezený logaritmem: $(0, \infty)$

\[\lim_{x \to 0} h(x) = \lim_{x \to 0} x \cdot o(x) = 0 \]

Derivace:

\[\frac{dh}{dx} = ln(x) + 1 \] 

Nabývá nuly v bodě $x = e^{-1} \approx \frac12$, s hodnotou $-e^{-1} \approx -\frac12$

Záporná do $e^{-1}$, pak kladná, tudíž nalezený bod je minimum.
	
Graf: 

\begin{figure}
  \begin{center}
    \begin{tikzpicture}
    
	\begin{axis}	[
	grid=both,
	ymin=-1,
	ymax=5,
	xmax=5,
	xmin=0,
	minor tick num=1,
	axis lines = middle,
	xlabel=$x$,
	ylabel=$y$,
	label style = {at={(ticklabel cs:1.1)}}]
	\addplot[green, smooth] plot coordinates {
		(0,0)
		(1,0)
		(6,10)	
	};
    \end{axis}          
    \end{tikzpicture}
    \caption{Funkce 3}
  \end{center}
\end{figure}

\end{enumerate}

\newpage

\section{}

Omezme se na výpočet, kde x,y jsou kladné, a nalezněme maximum obsahu obdelníku pod bodem $(x,y)$. To bude i maximum obsahu obdelníku čtyřnásobného.

\[ S = xy \]
\[ \frac{x^2}{4} + y^2 = 1 \implies y = \sqrt{1-\frac{x^2}{4}} \]

Tedy 

\[ S : x \to x \sqrt{1-\frac{x^2}{4}}  \]

nalezněme maximum $S(x)$ na intervalu $x \in [0,2]$:

Nalezněme extrémy:

\[ \frac{dS}{dx} = \sqrt{1-\frac{x^2}{4}} + x \frac{d}{dx} \left( \sqrt{1-\frac{x^2}{4}}\right) = \sqrt{1-\frac{x^2}{4}} + x \frac{1}{2\sqrt{1-\frac{x^2}{4}}} \left( - \frac{2x}{4} \right) 
= \sqrt{1-\frac{x^2}{4}} -  \frac{x^2}{4\sqrt{1-\frac{x^2}{4}}}  \]
\[ \sqrt{1-\frac{x^2}{4}} -  \frac{x^2}{4\sqrt{1-\frac{x^2}{4}}}   = 0 \]
\[ 1-\frac{x^2}{4} -  \frac{x^2}{4}   = 0 \]
\[ 2   = x^2 \]
\[ x = \sqrt{2} \]

Nyní stačí odargumentovat, že se jedná o maximum:

V bodech $x = 0$ a $x = 2$ funkce určitě nabývá hodnot menších - obsah je nulový, tudíž se nejedná o minimum a je to tedy maximum.

\[y= \frac{\sqrt{2}}{2}\]




\end{document}