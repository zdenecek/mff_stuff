\documentclass[10pt,a4paper]{article}
\usepackage[utf8]{inputenc}
\usepackage{amsmath, amsfonts, amssymb}
\usepackage{mathtools, array, enumitem, xcolor}
\usepackage[margin=0.7in]{geometry}
\setlength{\parindent}{0em}


\title{Řešení domácího úkolu č. 12}
\date{}

\begin{document}

\maketitle

\section{} 
\begin{equation*}
{}_{B^\prime}[id]_B = \begin{pmatrix}
1 & 2 \\ 3 & 4
\end{pmatrix},
B = \{u,v\}, B^\prime = \{x,y\}
\end{equation*}

\begin{enumerate}[label=\roman*]
\item od $\{v, u\}$ k $\{y, x\}$
Vezměme matici přechodu jakožto zobrazení $f$. Víme, že matice zobrazení nám udává obrazy bázických vektorů, takže víme, že:
	
\begin{equation*}
f([u]_B) = f((1, 0)^T) =[(1, 3)^T]_{B^\prime}
\end{equation*}	
\begin{equation*}
f([v]_B) = [(2, 4)^T]_{B^\prime}
\end{equation*}

Jestliže se nám prohodily báze v $B$, potom musíme prohodit sloupce matice přechodu. Prohodily se nám také vektory báze $B^\prime$, takže musíme prohodit i řádky.

\begin{equation*}
{}_{\{y, x\}}[id]_{\{v, u\}} = \begin{pmatrix}
 4 & 3 \\2 & 1 
\end{pmatrix}
\end{equation*}

\item od $\{5u, 2v\}$ k $\{x, y\}$
První bázický vektor je pětkrát větší, tudíž výsledný obraz musí být pětkrát větší. Kontrolou třeba když aplikuji na vektor $[\frac15, 0]^T_{\{5u, 2v\}} $, musí mi vyjít $(1,3)^T$ protože $[\frac15, 0]^T_{\{5u, 2v\}} = u$. Obdobně vynásobím druhý sloupec dvojkou.

\begin{equation*}
{}_{B^\prime}[id]_{\{5u, 2v\}} = \begin{pmatrix}
 5 & 4 \\ 15 & 8 
\end{pmatrix}
\end{equation*}

\item $\{u + v, v\}$ k $\{x, y\}$

Druhý sloupec matice zůstává stejný jako v zadání. K prvnímu sloupci musím přičíst kam se promítne bázický vektor $v$, takže druhý sloupec:

\begin{equation*}
{}_{B^\prime}[id]_{\{u + v, v\}} = \begin{pmatrix}
 3 & 2 \\ 7 & 4 
\end{pmatrix}
\end{equation*}

\end{enumerate}
\newpage
\section{}

\begin{equation*}
A = \begin{pmatrix}
1 & 0 & -1 \\ 0 & 1 & 0 \\ 1 & 1 & 0
\end{pmatrix},
B^\prime = \{ (1,1,0)^T, (1,0,0)^T, (0,1,-1)^T\}
\end{equation*}

\begin{enumerate}
\item $A = {}_{B^\prime}[id]_B$

Matice nám udává obrazy bázických vektorů, ale v bázi $B^\prime$, převeďme tyto vektory do kanonické báze, matici přechodu vytvoříme poskládáním bázických vektorů do sloupců:

\begin{equation*}
\begin{pmatrix}
1 & 1 & 0\\ 
1 & 0 & 1\\
0 & 0 & -1
\end{pmatrix}\begin{pmatrix}
1 & 0 &-1 \\ 
0 & 1 & 0 \\
1 & 1 & 0
\end{pmatrix} = 
\begin{pmatrix}
1 & 1 &-1 \\ 
2 & 1 &-1 \\
-1&-1 & 0
\end{pmatrix}
\end{equation*}


\begin{equation*}
B = \{(1, 2, -1)^T, (1,1,-1)^T, (-1,-1,0)^T \}
\end{equation*}

\item naopak

Matice udává nám známe vektory v neznáme bázi.

Dva postupy:

Zjistíme matici opačného přechodu jako inverzi a postup z prvního bodu.

NEBO řešíme jako soustavu, jedna z rovnic je například:

\begin{equation*}
b_1 + b_3 =(1,1,0)^T
\end{equation*} 

Nechci zadávat inverzi do wolframu, zvolím tedy druhý způsob.

Sestavíme matici (levá část je transpozice, pravá známé vektory):

\begin{equation*}
\left( \begin{array}{ccc|c}
1 & 0 & 1 & (1,1,0)^T \\ 
0 & 1 & 1 & (1,0,0)^T \\
-1& 0 & 0 & (0, 1,-1)^T
\end{array} \right) \sim 
\left( \begin{array}{ccc|c}
1 & 0 & 1 & (1,1,0)^T \\ 
0 & 1 & 1 & (1,0,0)^T \\
0 & 0 & 1 & (1, 2,-1)^T
\end{array} \right) \sim 
\left( \begin{array}{ccc|c}
1 & 0 & 0 & (0,-1,1^T) \\ 
0 & 1 & 0 & (0,-2,1)^T \\
0 & 0 & 1 & (1, 2,-1)^T
\end{array} \right)
\end{equation*}

\begin{equation*}
B = \{(0, -1, 1)^T, (0,-2,1)^T, (1,2,-1)^T \}
\end{equation*}

\end{enumerate}
\newpage
\section{}

Nejrychlejší bude asi si to přepsat ručně:

\begin{itemize}
\item $f(1) = 1 + x^2$
\item $f(1 + x) = 2 + 1 + x = 3 +x$
\item $f(x^2) = 3 + 2 + 2x = 5 + 2x$
\end{itemize}


Z toho můžeme zpočítat, kam by se zobrazí nové bázické vektory.

\begin{enumerate}
\item 1 ... víme
\item $f(x) = f(1 + x) - f(1) = 3 + x  - (1 + x^2) = -x^2 + x + 2 $
\item $f(1 + x^2) = f(1) + f(x^2)= x^2 + 1 + 5 + 2x = x^2 + 2x + 6$
\end{enumerate}

Teď už jen vyjádřit pomocí nové báze:

\begin{enumerate}
\item $1 + x^2 = [0,0,1]$
\item $-x^2 + x + 2 = 3\cdot 1 + 1\cdot x -1(x^2+1) = [3,1,-1]$
\item $x^2 + 2x + 6 =5\cdot 1 + 2\cdot x + 1(x^2+1) = [5,2,1] $
\end{enumerate}

Teď obrazy zapíšeme za sebe:

\begin{equation*}
{}_{B^\prime}[f]_{B^\prime} = \begin{pmatrix}
0 & 3 & 5\\
0 & 1 & 2\\
1 &-1 & 1
\end{pmatrix}
\end{equation*}

\end{document}