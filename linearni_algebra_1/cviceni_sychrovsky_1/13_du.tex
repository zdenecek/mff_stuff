\documentclass[10pt,a4paper]{article}
\usepackage[utf8]{inputenc}
\usepackage{amsmath, amsfonts, amssymb}
\usepackage{mathtools, array, enumitem, xcolor}
\usepackage[margin=0.7in]{geometry}
\setlength{\parindent}{0em}


\title{Řešení domácího úkolu č. 13}
\date{}

\begin{document}

\maketitle

\section{}

Průchod bodem $[0,0]$ nasvědčuje nulovému konstantnímu členu.
Koeficient $a$ určíme dosazením.

$a = \frac12$

$b = 0$

\section{}

Tvrzení \[ \forall n \in \mathbb{N}_0 \exists k \in \mathbb{N}_0 :  n^3 - n = 6k \]

\begin{proof} indukcí:

\begin{itemize}
\item Základní případy $n \in \{0, 1\}$
Za $k$ můžeme dosadit v obou případech nulu.
\item Indukční krok: Předpokládejme, že tvrzení platí pro $n$, a rozeberme platnost pro $n+2$:

Rozepišme rovnici:


\[ (n+2)^3 - (n + 2) = 6k \]

\[ n^3 + 6n^2 + 12n + 8 - n - 2 = 6k \]

\[ (n^3 - n) + 6(n^2 + 2n + 1) = 6k \]

Využijeme indukční předpoklad:

\[ 6k^\prime + 6(n^2 + 2n + 1) = 6k \]
\[ k = k^\prime + n^2 + 2n + 1  \]

Pro každé $n$ se můžeme indukčními kroky dostat na základní případ, tedy tvrzení platí. 


\end{itemize}
\end{proof}


\end{document}