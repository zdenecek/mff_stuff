\documentclass[10pt,a4paper]{article}
\usepackage[utf8]{inputenc}
\usepackage{amsmath}
\usepackage{amsfonts}
\usepackage{amssymb}
\usepackage{array}
\setlength{\parindent}{0em}
\begin{document}
1. 

Neutrální prvek existuje:

\begin{equation*}
e = 0 =
\begin{pmatrix}
cos(0) & -sin(0) \\
sin(0) & cos(0)
\end{pmatrix} 
=
\begin{pmatrix}
1 & 0 \\
0 & 1
\end{pmatrix} 
\end{equation*}

Najděme inverzní prvek:

\begin{align*}
\alpha^{-1} &= \beta 
\end{align*}
\begin{align*}
0 &=
\begin{pmatrix}
cos(\alpha) & -sin(\alpha) \\
sin(\alpha) & cos(\alpha)
\end{pmatrix} 
\cdot
\begin{pmatrix}
cos(\beta) & -sin(\beta) \\
sin(\beta) & cos(\beta)
\end{pmatrix} \\
0 &=
\begin{pmatrix}
cos(\alpha + \beta) 	& -sin(\alpha + \beta) \\
sin(\alpha + \beta)	&  cos(\alpha + \beta)
\end{pmatrix} \\
0 &= \alpha + \beta\\
\alpha^{-1} &= -\alpha
\end{align*}

Asociativita vyplývá z předchozího vzorce, pro reálné úhly platí:

\begin{equation*}
(\alpha + \beta) + \gamma = \alpha + (\beta + \gamma)
\end{equation*}

Aplikujme matici na vektor, mělo by se jednat o lineární transformaci, protože prvky matice jsou reálná čísla:

\begin{align*}
\begin{pmatrix}
cos(\alpha) & -sin(\alpha) \\
sin(\alpha) & cos(\alpha)
\end{pmatrix} 
\cdot
\begin{pmatrix}
x  \\
y
\end{pmatrix} 
=
\begin{pmatrix}
x \cdot cos(\alpha) - y \cdot sin(\alpha) \\
x \cdot sin(\alpha) - y \cdot cos(\alpha)
\end{pmatrix} 
\end{align*}

Zakresleme si transformaci na jednotkové kružnici:

Dejme, že transformovaný vektor $\vec{v}$ leží na ose x a promítne se do jiného vektoru na jednotkové kružnici:

\begin{align*}
\begin{pmatrix}
x \cdot cos(\alpha) - y \cdot sin(\alpha)  \\
x \cdot sin(\alpha) - y \cdot cos(\alpha)
\end{pmatrix} = 
\begin{pmatrix}
1 \cdot \frac{x^{\prime}}{|\vec{v}^{\prime}|} - 0 \\
1 \cdot \frac{y^{\prime}}{|\vec{v}^{\prime}|} - 0
\end{pmatrix} = 
\begin{pmatrix}
 \frac{x^{\prime}}{1}\\
 \frac{y^{\prime}}{1}
\end{pmatrix} = 
\begin{pmatrix}
x^{\prime}\\
y^{\prime}
\end{pmatrix} 
\end{align*}

Jedná se tedy o rotaci o úhel.


\hfill

2.

Neutrální prvek

\begin{align*}
(a \cdot e) \mod 6 = a \\
e = 1
\end{align*}
 
Inverzní prvek:

\begin{align*}
(a \cdot a^{-1}) \mod 6 = 1  \\
a \cdot a^{-1} = k \cdot 6 + 1
\end{align*}

Například pro dvojku takový prvek nenajdeme: dvojka je sudé číslo, součinem nevznikne liché číslo.

O grupu se nejedná.

\hfill

3.

Úvahou nemá řešení. Výpočtem:

\begin{equation*}
\left(
\begin{array}{ccc|c}
0 & 1 & 1 & 1 \\
1 & 0 & 1 & 1 \\
1 & 1 & 0 & 1 \\
\end{array}
\right)
\sim
\left(
\begin{array}{ccc|c}
1 & 1 & 0 & 0 \\
1 & 0 & 1 & 1 \\
1 & 1 & 0 & 1 \\
\end{array}
\right)
\sim
\left(
\begin{array}{ccc|c}
1 & 1 & 0 & 0 \\
0 & 1 & 1 & 1 \\
0 & 0 & 0 & 1 \\
\end{array}
\right) 
\end{equation*}
\begin{equation*}
(0 + 0 + 0) \mod 2 \neq 1
\end{equation*}

Nemá řešení

\hfill

4.

Spočtěme exponent v $\mathbb{Z}_7$:

\begin{equation*}
5! = 5 \cdot 4 \cdot 3 \cdot 2 = 6 \cdot 3 \cdot 2 = 4 \cdot 2 = 1
\end{equation*}

Nyní je to jednoduché, umocněním na prvou získáme opět $A$.




\end{document}