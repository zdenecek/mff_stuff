\documentclass[10pt,a4paper]{article}
\usepackage[utf8]{inputenc}
\usepackage{amsmath, amsfonts, amssymb, amsthm}
\usepackage{mathtools, array, enumitem, xcolor}
\usepackage[margin=0.7in]{geometry}
\setlength{\parindent}{0em}

\usepackage{tikz}
\usepackage{pgfplots}
\pgfplotsset{compat=1.11}

\usepackage[czech]{babel}

\theoremstyle{plain}
\newtheorem{veta}{Věta}
\theoremstyle{definition}
\newtheorem{definice}[veta]{Definice}

\title{Vypracování domácí úlohy č. 9 z předmětu Lineární algebra 2}

\counterwithin*{equation}{section}

\begin{document}

\begin{Large}
Vypracování domácí úlohy č. 9 z předmětu Lineární algebra 2
\end{Large}

\begin{large}
Jméno: Zdeněk Tomis
\end{large}

Potřebný čas: $\sim 1$  hod

\section{}

Skalární součin musí být roven nule, tedy se podíváme na soustavu tří rovnic a nalezneme jejich řešení.

\[\begin{pmatrix}
2 & 1 & 0 \\
1 & 2 &-1 \\
0 &-1 & 1
\end{pmatrix} = A\]

\begin{equation} (1,2,3)A = (4,2,1)  \end{equation}
\begin{equation} (4,5,6)A = (13,8,1)  \end{equation}
\begin{equation} (7,8,9)A = (22,12,1)  \end{equation}

\[ \begin{pmatrix}
4 & 2 & 1  \\
13 & 8 & 1 \\
22 & 14 & 1
\end{pmatrix}x = 0  \]

\[ \begin{pmatrix}
4 & 2 & 1  \\
13 & 8 & 1 \\
22 & 14 & 1
\end{pmatrix} \sim \begin{pmatrix}
4 & 2 & 1  \\
9 & 6 & 0 \\
18 & 12 & 0
\end{pmatrix} \sim \begin{pmatrix}
1 & 0 & 1  \\
3 & 2 & 0 \\
0 & 0 & 0
\end{pmatrix} \]

$x_2 = -\frac32x_1$

$x_3 = -x_1$

\[ x = p \cdot (1,-\frac32,-1)^T = t \cdot (-2,3,2)^T \]

Pro parametry $p,t$

Ortogonální doplněk množiny \[(\text{Zadaná množina})^\perp = span((-2,3,2)^T )\]

\section{}

Máme čtyři skalární součiny, se kterými můžeme pracovat. Protože se pohybujeme v reálném skalárním součinu, dva z nich jsou shodné, máme tedy tři rovnice.

  
  \begin{equation}
\left\langle (4,1,3)^T|(4,1,3)^T\right\rangle = \left\langle f\left((4,1,3)^T\right)|f\left((4,1,3)^T\right)\right\rangle
\end{equation}
\[ \implies 26 = a^2 + 1 \overset{a \geq 0}{\implies} a = 5\]

\begin{equation}
\left\langle (4,1,3)^T|(1,2,4)^T \right\rangle = \left\langle f\left((4,1,3)^T\right)|f\left((1,2,4)^T\right)\right\rangle
\end{equation}
\[ \implies 18 = \left\langle (1,0,a)^T|(3,b,c)^T \right\rangle
= 3 + ac
 \]\[ 
  15 = ac\]
  
  Můžeme dosadit
  
  \[ c = 3 \]


  \begin{equation}
\left\langle (1,2,4)^T|(1,2,4)^T\right\rangle = \left\langle f\left((1,2,4)^T\right)|f\left((1,2,4))^T\right)\right\rangle
\end{equation}
\[ \implies 21 = 9 + b^2 + c^2 \]

Dosadíme

\[  b = \sqrt{3} \]

\[ (a,b,c) = (5, \sqrt{3}, 3)\]
\end{document}