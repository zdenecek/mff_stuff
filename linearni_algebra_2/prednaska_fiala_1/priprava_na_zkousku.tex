\documentclass[10pt,a4paper]{article}
\usepackage[utf8]{inputenc}
\usepackage{amsmath, amsfonts, amssymb, amsthm}
\usepackage{mathtools, array, enumitem, xcolor}
\usepackage{stackengine}
\usepackage[margin=0.7in]{geometry}
\setlength{\parindent}{0em}

\usepackage{graphicx}
\newcommand\sbullet[1][.5]{\mathbin{\vcenter{\hbox{\scalebox{#1}{$\bullet$}}}}}


\usepackage{tikz}
\usepackage{pgfplots}
\pgfplotsset{compat=1.11}

\usepackage[czech]{babel}

\newtheorem{lemma}{Lemma}
\theoremstyle{plain}
\newtheorem{veta}{Věta}

\renewcommand{\arraystretch}{1.5}


\title{Příprava na zkoušku z předmětu Lineární algebra 2}
\author{Zdeněk Tomis}
\date{}


\begin{document}


\maketitle

\part{Definice}

\begin{enumerate}

\item Definujte permutaci.

\paragraph{Permutace} je bijekce na $[n]$.

\item Definujte znaménko permutace.

\paragraph{Znaménko permutace} je $(-1)^{\# \text{ inverzí v permutaci}}$.

\item Definujte determinant.

\paragraph{Determinant čtvercové matice} $A \in \mathbb{K}^{n \times n}$ je definován jako.

\[ det(A) = \sum_{p \in S_n} sgn(p) \prod^n_{i = 1} a_{i, p(i)} \]

\item Definujte adjungovanou matici.

\paragraph{Adjugovaná matice} $adj(A)$ čtvercové matice  $A \in \mathbb{K}^{n \times n}$ je definována jako

\[ (adj(A))_{i,j} = (-1)^{i+j} det(A^{j,i})\]
\item Definujte Laplaceovu matici.

\paragraph{Laplaceova matice} $L_G$ grafu $G$ s $n$ vrcholy je čtvercová reálná matice řádu $n$ definována jako

\[ (L_G)_{i,j} = \begin{cases}  deg(v_i) & i = j  \\ 
								-1 & i \neq j \ \&\  (v_i, v_j) \in E(G) \\ 
								0  & i \neq j \ \&\  (v_i, v_j) \notin E(G) \\ \end{cases} \]

\item Definujte polynom nad tělesem.

\paragraph{Polynom stupně $n$ proměnné $x$ nad $\mathbb{K}$} je výraz
\[ p(x) = a_nx^n + a_{n-1}x^{n-1} + ... + a_1x + a_0\]

kde $a_n \neq 0$ a $\forall i: a_i \in \mathbb{K}$. Píšeme $p \in \mathbb{K}(x)$.

\item Definujte kořen polynomu a jeho násobnost.

\paragraph{Kořen polynomu $p$} je taková hodnota $r \in \mathbb{K}$, pro kterou platí že $p(r) = 0$.

\paragraph{Násobnost kořene $r$ polynomu $p$} je největší kladné číslo $k$, t.ž. $(x-r)^k$ dělí $p$.

\item Definujte algebraicky uzavřené těleso.

\paragraph{Algebraicky uzavřené těleso} je takové těleso $\mathbb{K}$, pro které platí, že každý polynom $p \in \mathbb{K}(x)$ má kořen.

\item Definujte Vandermondovu matici.


Pro $x_0, ..., x_n$ je $\begin{pmatrix}
1 & x_0 & x_0^2 & ... & x_0^{n} \\
1 & x_1 & x_1^2 & ... & x_1^{n} \\
\vdots &&& \\
1 & x_n & x_n^2 & ... & x_n^{n	} 
\end{pmatrix} = V_{n+1}(x_0, ..., x_n)$ tzv. \textbf{Vandermondova matice}.

\item Definujte vlastní číslo a vlastní vektor lineárního zobrazení.

\paragraph{Vlastní číslo} lineárního zobrazení $f: V \to V$ je jakékoliv $\lambda \in \mathbb{K}$, pro které $\exists u \in V \setminus 0$ takový, že $f(u) = \lambda u$

\paragraph{Vlastní vektor} lineárního zobrazení $f$ odpovídající vlastnímu číslu $\lambda$ je libovolný $u \in V$, t.ž. $f(u) = \lambda u$

\item Definujte vlastní číslo a vlastní vektor matice.

\paragraph{Vlastní číslo} matice $A \in \mathbb{K}^{n \times n}$ je jakékoliv $\lambda \in \mathbb{K}$, pro které $\exists u \in \mathbb{K}^n \setminus 0$ takový, že $Au = \lambda u$

\paragraph{Vlastní vektor} matice $A$ odpovídající vlastnímu číslu $\lambda$ je libovolný $u \in \mathbb{K}^n$, t.ž. $Au = \lambda u$

\item Definujte charakteristický polynom.
 
\paragraph{Charakteristický polynom} matice $A \in \mathbb{K}^{n \times n}$ je

\[ p_A(t) = \det(A - tI_n)\]


\item Definujte algebraickou násobnost vlastního čísla.

\paragraph{Algebraická násobnost vlastního čísla} je jeho násobnost jako kořene charakteristického polynomu.

\item Definujte geometrickou násobnost vlastního čísla.

\paragraph{Geometrická násobnost vlastního čísla} je dimenze prostoru vlastních vektorů číslu odpovídajících.

\item Definujte podobné matice.

\paragraph{Podobnost} Matice $A,B \in \mathbb{K}^{n \times n}$ jsou si \textbf{podobné}, pokud

\[ \exists R \in \mathbb{K}^{n \times n} \text{ regulární}: A = R^{-1}BR\]

\item Definujte diagonalizovatelnou matici.

\paragraph{Diagonalizovatelnost} Matice $A \in \mathbb{K}^{n \times n}$ je \textbf{diagonalizovatelná}, pokud je podobná diagonální matici.

\item Definujte Jordanův blok.

\paragraph{Jordanův blok} je čtvercová matice ve tvaru

\[ J_\lambda = \begin{pmatrix}
\lambda & 1		  &  	   & 		 &   \\
 		& \lambda & 1  	   & \mbox{\huge{0}}		 &   \\
		& 		  & \ddots & \ddots  &   \\
		& \mbox{\huge{0}}& 		   & \lambda & 1 \\
		& 		  & 		   & 		 & \lambda 
\end{pmatrix} \]

\newpage

\item Definujte Jordanův normální tvar matice.

\paragraph{Jordanova normální forma} je bloková matice ve tvaru

\[ J = \begin{pmatrix}
J_{\lambda_1} && 	&&\\
& J_{\lambda_2} &&\mbox{\huge{0}} &\\
&& J_{\lambda_3}&&&\\
&\mbox{\huge{0}}&& \ddots & \\
& &	&& J_{\lambda_k}
\end{pmatrix}\]

Kde $J_{\lambda_i}$ jsou Jordanovy bloky.

\item Definujte zobecněný vlastní vektor

\paragraph{Zobecnený vlastní vektor} matice $A$ k vlastnímu číslu $\lambda$ je libovolný vektor $x$ splňující $(A-\lambda I)^kx = 0$ pro nějaké $k \in \mathbb{N}$ 

\item Definujte hermitovskou matici.

Matice $A$ je \textbf{hermitovská}, pokud $A^H = A$.

\item Definujte unitární matici.


Matice $A$ je \textbf{unitární}, pokud $A^{-1} = A^H$.


\item Definujte skalární součin pro vektorové prostory nad komlexními čísly.

\paragraph{Skalární součin} na vektorovém prostoru V nad $\mathbb{C}$ je zobrazení $\langle \sbullet[.75] | \sbullet[.75] \rangle : V \times V \to \mathbb{C}$, pro které platí:

\begin{enumerate}[label=(\roman*)]
\item $\forall u \in V:\langle u | u \rangle \in \mathbb{R}^+_0$
\item $\forall u \in V:\langle u | u \rangle = 0 \iff u = 0$
\item $\forall u,v \in V:\langle u | v \rangle = \overline{\langle v | u \rangle}$
\item $\forall u,v,w \in V:\langle u + v | w \rangle = \langle u | w \rangle  + \langle  v | w \rangle $
\item $\forall u,v \in V, a \in \mathbb{C}: \langle au | v \rangle = a \langle u | v \rangle $
\end{enumerate}


\item Definujte normu spojenou se skalárním součinem.

\paragraph{Norma} odvozená ze skalárního součinu pro prostor $V$ nad $\mathbb{C}$, nebo $\mathbb{R}$ se skalárním součinem je zobrazení $|| \sbullet[.75]|| : V \to \mathbb{R}$ dané předpisem $||u|| = \sqrt{\langle u | u \rangle}$

\item Definujte kolmé vektory.

Vektory $u,v$ z prostoru se skalárním součinem jsou \textbf{kolmé}, pokud $\langle u | v \rangle = 0$. 

\item Definujte ortonormální bázi.

Báze $Z = \left\lbrace v_1, v_2, ..., v_n \right\rbrace$ je \textbf{ortonormální}, pokud platí $ \forall i: ||v_i|| = 1 $ a $ \forall i,j: v_i \perp v_j $

\item Definujte Fourierovy koeficienty.

\paragraph{Fourierovy koeficienty} jsou prvky vektoru souřadic vyjádřeného vůči ortonornální bázi.

\item Definujte izometrii.

\paragraph{Izometrie} je takové zobrazení $f: V \to W$ z vektorového prostoru se skalárním součinem do vektorového prostoru se skalárním součinem, které zachovává skalární součin. Neboli \[\forall u,v \in V: \langle u | v \rangle = \langle f(u) | f(v) \rangle\]

\item Definujte kolmou projekci.

Prostor $W$ se skalárním součinem a jeho podprostor V s ortonormální bází  $Z = ( v_1, v_2, ..., v_n ) $ je zobrazení 
\[ p_Z: W \to V,\quad p_Z(u) = \sum^n_{i=1} \langle u|v_i \rangle v_i \]

\textbf{kolmou projekcí $W$ na $V$}.

\item Definujte ortogonální doplněk.

Ortogonální doplňek v prostoru se skalárním součinem $W$ podmnožiny $V \subseteq W$ je množina všech vektorů, které jsou kolmé na všechny vektory $V$.

\[ V^\perp = \lbrace u \in W: \forall v \in V: u \perp v \rbrace \]

\item Definujte Gramovu matici.

Pro $V$ vektorový prostor se skalárním součinem a bází $X = (v_1, ..., v_n)$ je \textbf{Gramova matice} definována jako $A \in \mathbb{C}^{n \times n}$ pro kterou platí $a_{i,j} = \langle v_i | v_j \rangle$


\item Definujte pozitivně definitní matici.

Matice $A$ řádu $n$ je \textbf{pozitivně definitní}, pokud
\begin{itemize}
\item je hermitovská a
\item $\forall x \in \mathbb{C}^n \setminus 0: x^HAx > 0$
\end{itemize}

\item Definujte Choleského rozklad.

\paragraph{Choleského rozklad} (pozitivně definitní) matice $A$ je (unikátní) horní trojúhelníková matice $U$ s kladnou diagonálou, t.ž. $A = U^HU$

\item Definujte bilineární formu.

Pro prostor $V$ nad $\mathbb{K}$ se zobrazení $f: V \times V \to \mathbb{K}$ nazývá \textbf{bilineární formou}, pokud splňuje následující podmínky 
\begin{enumerate}[label=(\roman*)]
\item[(i, ii)] $\forall u,v \in V, \forall a \in \mathbb{K}: f(au,v) = f(u,av) = af(u,v)$
\item[(iii)] $\forall u,v,w \in V : f(u+v,w) = f(u,w) + f(v,w)$
\item[(iv)] $\forall u,v,w \in V : f(u,v+w) = f(u,v) + f(u,w)$
\end{enumerate}

\item Definujte kvadratickou formu.

$g: V \to \mathbb{K}$ se nazývá \textbf{kvadratickou formou}, pokud existuje bilineární forma $f$, t.ž. $\forall u \in V: g(u) = f(u,u)$.

\item Definujte matici bilineární formy vzhledem k bázi

Pro vektorový prostor $V$ nad $\mathbb{K}$ s bází $X = (v_1, ..., v_n)$ je \textbf{matice bilineární formy} $f$ vzhledem k bázi $X$ matice $B: b_{i,j} = f(v_i, v_j)$

\item Definujte matici kvadratické formy vzhledem k bázi*

Pro vektorový prostor $V$ nad $\mathbb{K}$ s bází $X = (v_1, ..., v_n)$ je \textbf{matice kvadratické formy} $g$ matice odpovídající symetrické bilineární formy $f$, pokud taková symetrická $f$ existuje

\item Definujte analytický výraz formy

\paragraph{Analytické vyjádření} formy $f$ nad $\mathbb{K}^n$ s maticí formy $B$ je homogenní polynom
\[ f((x_1,...,x_n)^T,(y_1,...y_n)^T)= \sum^n_{i=1}\sum^n_{j=1}b_{i,j}x_iy_j\]

\item Definujte signaturu formy.

\paragraph{Signatura formy g} je trojice $(\#1, \#-1, \#0)$, počítáno na
diagonále matice $B$ vůči vhodné bázi takové, že $B$ je diagonální s $0,1,-1$ na diagonále.


\end{enumerate}

\newpage

\part{Věty}
\begin{enumerate}
\item Vyslovte a dokažte větu o linearitě determinantu.

\begin{veta}

Determinant matice je lineárně závislý na každém jejím řádku a sloupci, to je vzhledem ke skalárnímů násobku a vzhledem ke sčítání.

\begin{proof}
\begin{itemize}
\item Pro součin: $i$-tý řádek se v součinu vyskytuje vždy právě jednou, můžeme ho vytknout před součin a poté i před sumu.
\item Pro součet:
\begin{align*}
det(a) &= \sum_{p \in S_n} sgn(p) \prod_{k=1}^n a_{k, p(k)} 
		= \sum_{p \in S_n} sgn(p)\ a_{i,p(i)} \prod_{k\ in [n] \setminus \{i\}} a_{k, p(k)} \\
	   &=  \sum_{p \in S_n} sgn(p)\ (b_{i,p(i)} + c_{i,p(i)}) \prod_{k \in [n] \setminus \{i\}} a_{k, p(k)} \\
	   &=  \sum_{p \in S_n} sgn(p)\ b_{i,p(i)} \prod_{k \in [n] \setminus \{i\}} b_{k, p(k)} + \sum_{p \in S_n} sgn(p)\  c_{i,p(i)} \prod_{k \in [n] \setminus \{i\}} c_{k, p(k)} \\
	   &= det(b) + det(c)
\end{align*}
\end{itemize}
\end{proof}
\end{veta}



\item Vyslovte a dokažte větu o determinantu součinu dvou matic.

\begin{veta}
\[ \forall A,B \in \mathbb{K}^{n \times n}: det(AB) = det(A)det(B) \]
\begin{proof}
\begin{enumerate}[label=(\alph*)]
\item Alespoň jedna matice je singulární: $0 = 0$
\item Obě matice jsou regulární: 

Součiny s elementárními maticemi zachovají determinant $det(EB)=det(E)det(B)$, protože \begin{itemize}
\item Přičtením $i$-tého řádku k j-tému: $det(E) = 1$
\item Vynásobením $i$-tého řádku koeficientem $t$ : $det(E) = t$
\end{itemize}

Z těchto dvou lze odvodit zbylé dvě operace.

\hfill

Rozložíme regulární A na elementární matice $A = E_1E_2 ... E_k$:

\begin{align*}
det(AB) &= det(E_1E_2 ... E_kB) = det(E_1)det(E_2 ... E_kB) = ... = \\
&=  det(E_1)det(E_2) ... det(E_k)det(B) = det(E_1)det(E_2) ... det(E_{k-1}E_k)det(B) = ... = \\
&= det(E_1E_2 ... E_k)det(B) \\
&= det(A)det(B)
\end{align*}

\end{enumerate}
\end{proof}
\end{veta}

\item Vyslovte a dokažte větu o Laplaceově rozvoji determinantu.

\begin{veta}
\[ \forall A \in \mathbb{K}^{n \times n}: \forall i \in [n]: det(A) = \sum_{i=1}^n a_{i,j}(-1)^{i+j}det(A^{i,j})\]
\begin{proof}
$i$-tý řádek vyjádříme jako lin. kombinaci vektorů kanonické báze.

\[ \begin{vmatrix}
\rule[1ex]{3.5em}{0.4pt}\\
a_{i,1}\  ...\  a_{i,n} \\
\rule[.5ex]{3.5em}{0.4pt}
\end{vmatrix} =
a_{i,1}\begin{vmatrix}
\rule[1ex]{3em}{0.4pt}\\
1\ 0\ ...\ 0 \\
\rule[.5ex]{3em}{0.4pt}
\end{vmatrix} + ... + 
a_{i,n}\begin{vmatrix}
\rule[1ex]{3em}{0.4pt}\\
0\ ...\ 0\ 1 \\
\rule[.5ex]{3em}{0.4pt}
\end{vmatrix} \]

Pro $j$-tý člen:

\[ \begin{vmatrix}
\rule[1ex]{5em}{0.4pt}\\
0\ ...\ 0\ 1\ 0\ ...\ 0 \\
\rule[.5ex]{5em}{0.4pt}
\end{vmatrix} = \begin{vmatrix}
\rule[1ex]{5em}{0.4pt}\\
\rule[1ex]{1.5em}{0.4pt}\ \ e_j^T\ \rule[1ex]{1.5em}{0.4pt}\\
\rule[.5ex]{5em}{0.4pt}
\end{vmatrix} = (-1)^{i+1} \begin{vmatrix}
\rule[1ex]{1.5em}{0.4pt}\ \ e_j^T\ \rule[1ex]{1.5em}{0.4pt}\\
\rule[1ex]{5em}{0.4pt}\\
\rule[.5ex]{5em}{0.4pt}
\end{vmatrix}
 = (-1)^{i+1+j+1} \begin{vmatrix}
e_1^T \rule[1ex]{3.2em}{0.4pt}\\
\rule[1ex]{5em}{0.4pt}\\
\rule[.5ex]{5em}{0.4pt}
\end{vmatrix}  \]


\[
 = (-1)^{i+j} \det \left( \ 
\begin{array}{|c|c|}
\hline
1 & 0^T \\
\hline
0 & A^{i,j} \\
\hline
\end{array} \  \right)
\]

Nenulové budou permutace s pevným bodem $p(1)=1$, ostatní lze pominout, to odpovídá permutacím $S_{n-1}$.

\[ = (-1)^{i+j} det(A^{i,j})\]

Tedy pro první rovnici:

\[ = \sum^n_{j=1} a_{i,j}(-1)^{i+j}det(A^{i,j})\]



\end{proof}
\end{veta}

\item Uved'te a dokažte Cramerovo pravidlo (řešení systémů s determinanty).

\begin{veta}
Pro regulární $A \in \mathbb{K}^{n \times n}$: $\forall b \in \mathbb{K}^n$ pravé strany řešení $x$ soustavy $Ax = b$ splňuje
  \[x_i = \frac1{\det(A)}det(A_{i \to b})\]

Kde $A_{i \to b}$ získáme nahrazením $i$-tého sloupce vektorem $b$.

\begin{proof}

Uvažme matici $I_{i \to x}$

\[A \cdot I_{i \to x} = A_{i \to b} \]
\[\implies \det(A) \det(I_{i \to x}) = \det(A_{i \to b}) \]
\[x_i = det(I_{i \to x}) \]
\[\implies x_i = \frac1{\det(A)}\det(A_{i \to b}) \]

\end{proof}
\end{veta}

\item Vyslovte a dokažte větu o adjungované matici.

\begin{veta}
Pro libovolnou regulární matici $A \in \mathbb{K}^{n \times n}$

\[ A^{-1} = \frac1{\det(A)}\cdot adj(A)\]

\begin{proof}
%@TODO%

Přes Laplaceův rozvoj $\det(A)$

\[ det(A) = \sum^n_{j=1} a_{i,j} (-1)^{i+j} \det(A^{i,j})\]

\hfill

\[ (i\text{-tý řádek z }A)\cdot(i\text{-tý sloupec z }adj(A)) = \det(A) \]
\[ (j\text{-tý řádek z }A)\cdot(i\text{-tý sloupec z }adj(A)) = \det(A^\prime) = 0 \]

$A^\prime$ získáme nahrazením $i$-tého řádku z $j$-tý, tedy máme dva stejné řádky, matice je singulární a determinant nulový.

Tedy
\[ A \cdot adj(A) = \det(A) \cdot I_n \implies A^{-1} = \frac1{\det(A)} adj(A)\]


\end{proof}
\end{veta}

\item Vyslovte a dokažte větu o počtu koster grafu.


\begin{veta}
Každý graf $G$ na alespoň dvou vrcholech má právě $\det(L_G^{1,1})$ koster.
\begin{proof}

\begin{enumerate}[label=(\alph*)]
\item $G$ je nesouvislý, dle pozorování je $L_G^{1,1}$ singulární, determinant je nulový.
\item $G$ je souvislý: Indukcí podle $m = |E_G|$

\begin{itemize}
\item Pro $m = 1$: $\kappa(G) = \det(\deg(v_1)) = det(1) = 1$
\item Indukční krok:



Zvolme libovolnou hranu $e$, BÚNO $e = (v_1, v_2)$.
Nechť $A = L_{G}^{1,1}$, $B = L_{G \setminus e}^{1,1}$, $C = L_{G \circ e}^{1,1}$



$A$ a $B$ jsou shodné až na $a_{1,1} - 1 = b_{1,1}$. První sloupec $A$ vyjádříme jako součet prvního sloupce $B$ a elementárního vektoru $e^1$

\[\det(A) = \det(B) + \det\left(\ \begin{array}{|c|c|}
\hline 1 & * \\  \hline  0 & C \\ \hline
\end{array} \ \right) = \det(B) + \det(C)\]

Počet koster můžeme vyjádřit rekurentním vztahem
\[ \kappa(G) = \kappa(G \setminus e) + \kappa(G \circ e) \overset{IP}{=} \det(B) + \det(C) = det(A) \]

\end{itemize}
\end{enumerate}
\end{proof}
\end{veta}

\item Vyslovte a dokažte malou Fermatovu větu.


\begin{veta}

Nechť p je prvočíslo a $0 \neq a \in \mathbb{Z}_p$. Pak $a^{p-1} = 1$ v tělese $\mathbb{Z}_p$.

\begin{proof}
Využijeme zobrazení z předchozího důkazu: 
Definujeme pro každé $a$ zobrazení $f_a: \{1, ..., p-1 \} \to \{1, ..., p-1 \}$ předpisem $f_a(x) = a \cdot x \mod p$

Ukažme, že $f_a$ je prosté: Sporem. Kdyby nebylo, $\exists b,c, b \neq c: f_a(b) = f_a(c) \implies 0 \equiv ab-ac \implies a (b-c) \equiv 0$. Ale $a \neq 0$ a $b-c \neq 0$. Spor.

$f_a$ je prosté $\implies$ je na $\implies$ je bijekcí.

Potom 

\begin{equation*}
\prod_{i=1}^{p-1} i = \prod_{i=1}^{p-1} f_a(i) = \prod_{i=1}^{p-1} a\cdot i = a^{p-1} \prod_{i=1}^{p-1} i  \implies a^{p-1} = 1
\end{equation*}
\end{proof}
\end{veta}

\item Vyslovte a dokažte větu o Vandermondově matici.
\begin{veta}
Vandermondova matice je regulární $\iff x_0, ..., x_n$ jsou po dvou různé. 
\begin{proof}
Odečteme první řádek od každého:
\[ \begin{pmatrix}
1 & x_0 & x_0^2 & ... & x_0^n \\
0 & x_1 - x_0 & x_1^2 - x_0^2 & ... & x_1^n - x_0^n \\ 
0 & x_2 - x_0 & x_2^2 - x_0^2 & ... & x_2^n - x_0^n \\ 
\vdots & \vdots & \vdots & & \vdots \\
0 & x_n - x_0 & x_n^2 - x_n^2 & ... & x_n^n - x_0^n 
\end{pmatrix}\]

Rozvoj podle prvního řádku:

\[ \det(V_{n+1}) = \begin{vmatrix}
 x_1 - x_0 & x_1^2 - x_0^2 & ... & x_1^n - x_0^n \\ 
 x_2 - x_0 & x_2^2 - x_0^2 & ... & x_2^n - x_0^n \\ 
\vdots & \vdots &  & \vdots \\
 x_n - x_0 & x_n^2 - x_n^2 & ... & x_n^n - x_0^n 
\end{vmatrix}\]

Postupně vytkneme z $i$-tého řádku $x_i - x_0$

\[ \det(V_{n+1}) = \prod^n_{i=1} (x_i - x_j) \begin{vmatrix}
 1 & x_1 +  x_0  & ... &  x_1^{n-1} + x_1^{n-2}x_0 + ... + x_0^{n-1}  \\ 
 1 & x_2 + x_0 & ... &  x_2^{n-1} + x_2^{n-2}x_0 + ... + x_0^{n-1}  \\ 
\vdots & \vdots &  &\vdots \\
 1 & x_n + x_0 & ... & x_n^{n-1} + x_n^{n-2}x_0 + ... + x_0^{n-1} 
\end{vmatrix}\]

Kde upravíme \[  \begin{vmatrix}
 1 & x_1 +  x_0  & ... &  x_1^{n-1} + x_1^{n-2}x_0 + ... + x_0^{n-1}  \\ 
 1 & x_2 + x_0 & ... &  x_2^{n-1} + x_2^{n-2}x_0 + ... + x_0^{n-1}  \\ 
\vdots & \vdots &  &\vdots \\
 1 & x_n + x_0 & ... & x_n^{n-1} + x_n^{n-2}x_0 + ... + x_0^{n-1} 
\end{vmatrix} \]

odečtením $x_0$ násobku každého sloupce od následujícího sloupce, postupně od konce, čímž získáme Vandermondovu matici nižšího řádu.

\[ \det(V_{n+1}) = \prod^n_{i=1} (x_i - x_0) \det(V_n) \]

Tento rekurentní vztah dává

\[ \det(V_{n+1}) = \prod^n_{i < j} (x_j - x_i) \]

Tento výraz je nenulový, pokud žádný součinitel není nulový.

\end{proof}
\end{veta}

\item Uved'te a dokažte správnost Lagrangeovy interpolace.
\begin{veta}

Pro $n+1$ dvojic $(x_i, y_i=p(x_i))$ najdeme vyjádření polynomu takto:

\begin{enumerate}
\item Určíme $n+1$ polynomů stupně $n$ jako

\[ p_i(x) = \frac{\prod_{i \neq j} x - x_j}{\prod_{i \neq j} x_i - x_j}\]
Kde $p_i(x_i) = 1$ a $p_i(x_j) = 0$  pro $i \neq j$
\item sestavíme $p$ jako lineární kombinaci
\[ p(x) = \sum^{n+1}_{i=1} y_ip_i(x) \]
\end{enumerate}

\begin{proof}
Správnost vyplývá s postupu sestavení. Platí $p(x_i) = y_i$, jednoduše ověříme rozepsáním lineární kombinace.
\end{proof}
\end{veta}

\item Vyslovte a dokažte větu o podprostoru vlastních vektorů.
\begin{veta}
Vlastní vektory odpovídající jednomu vlastnímu číslu $\lambda$ tvoří podprostor.
\begin{proof} Uvažme $\lambda, f: V \to V, U = \{u \in V: f(u) = \lambda u\}$

Platí \[ \forall u \in U, \forall a \in \mathbb{K}: f(au) = af(u) = a \lambda u = \lambda(au)  \]
\[ \forall u,v \in U: f(u+v) = f(u) + f(v) = \lambda (u + v) \]
\end{proof}
\end{veta}

\item Vyslovte a dokažte větu o lineární nezávislosti vlastních vektorů.
\begin{veta}
Pro $f: V \to V$ a $\lambda_1, ..., \lambda_n$ různá vlastní čísla, jsou  odpovídající vlastní vektory $u_1, ..., u_n$ nezávislé.
\begin{proof} Sporem.
Předpokládejme nejnižší $k$, t.ž. $\lambda_1, ..., \lambda_k$ a $u_1, ..., u_k$ oodporují větě, tedy
\[ \exists a_1, ..., a_k \in \mathbb{K} \setminus 0: \sum^k_{i=1} a_i u_i = 0 \]

Nyní vyjádřeme 0 dvěma způsoby:

\[0 = \lambda_k 0 = \lambda_k \sum^k_{i=1} a_i u_i = \sum^k_{i=1} \lambda_k a_i u_i \]
\[ 0 = f(0) = f\left(\sum^k_{i=1} a_i u_i \right) = \sum^k_{i=1} a_i f(u_i) = \sum^k_{i=1} \lambda_i a_i u_i\]

\hfill

\[ 0 = 0 - 0 = \sum_{i = 1}^{k-1} (\lambda_i - \lambda_k) a_i u_i \]

$\lambda_i \neq \lambda_k \implies (\lambda_i - \lambda_k)a_i \neq 0$

$u_1, ..., u_{k-1}$ jsou lineárně závislé $\to$ spor s volbou $k$.

\end{proof}
\end{veta}

\item Vyslovte a dokažte větu o kořenech charakteristického polynomu.
\begin{veta}
Číslo $\lambda \in \mathbb{K}$ je vlastní $A$, pokud $p_A(\lambda) = 0$
\begin{proof}
$\lambda $ je vlastní číslo $\iff \exists x \in \mathbb{K}^n \setminus 0: Ax = \lambda x \iff $

$\iff Ax - \lambda x = 0 \iff (A - \lambda I_n)x = 0  \iff$

$\iff A - \lambda I_n$ je singulární $\iff det(A - \lambda I_n) = p_A(\lambda) = 0$
\end{proof}
\end{veta}

\item Uveďte a dokažte Cayley-Hamiltonovu větu.


\begin{veta}
Pro každou $A \in \mathbb{K}^{n \times n}: p_A(A) = 0_n$, kde $p_A(A)$ je matice $A$ umocněná a přenásobená koeficienty stejně jako v charakteristickém polynomu.
\begin{proof}
Použijeme větu $M \cdot adj(M) = \det(M) \cdot I_n$ pro $M = A - tI_n$

Složky $adj(A - tI_n)$ jsou determinanty podmatic, tedy polynomy stupně nejvýše $n-1$.

To můžeme rozepsat jako

$t^{n-1}B_{n-1} + ... + tB_1 + B_0$ pro $B_i \in \mathbb{K}^{n \times n}$

Dosazení do věty:

\begin{align*}
(A - tI_n)adj(A - tI_n) &= \det(A - tI_n) \cdot I_n  \\
(A - tI_n)(t^{n-1}B_{n-1} + ... + tB_1 + B_0) &= \det(A - tI_n) \cdot I_n \\
-tI_nB_{n-1} + At^{n-1}B_{n-1} - t^{n-1}B_{n-2} + ... + AtB_1 - tB_0 + AB_0) &= (-1)^n t^n I_n + a_{n-1} t^{n-1}I_n + ... + a_1tI_n + a_0I_n
\end{align*} 

Rozdělíme na tři rovnice podle exponentu $t$:

\begin{itemize}[label=pro]
\item $t^n: -B_{n-1} = (-1)^n I_n $ přenásobíme zleva $A^n$
\item $t^i: AB_i -B_{i-1} = a_i I_n $ přenásobíme zleva $A^i$
\item $t^0: AB_0 = a_0 I_n $
\end{itemize}

Rovnosti opět sečteme, dostaneme na levé straně: 
\[ -A^nB_{n-1} + A^{n-1}(AB_{n-1} -B_{n-2}) + A^{n-2}(AB_{n-2} -B_{n-3}) + ... + A(AB_1 - B_0) + AB_0 = 0_n \]

A na pravé straně:

\[  (-1)^nA^n  + a_{n-1} A^{n-1} + a_{n-2} A^{n-2} + ... + a_1A + a_0 I_n = p_A(A)\]

Tedy

\[p_A(A) = 0_n\]
\end{proof}
\end{veta}

\item Uved'te a dokažte nezbytnou a postačující podmínku, kdy je matice diagonalizovatelná.


\begin{veta}
$A \in \mathbb{K}^{n \times n}$ je podobná diagonální matici $\iff K^{n}$ má bázi sestávající z vlastních vektorů $A$
\begin{proof}
$AR = RD$

$\forall i: Ax_i = x_i d_{i,i} = d_{i,i}x_i = \lambda_i x_i \implies \forall i: \exists x_i $ vlastní vektor odpovídající $\lambda_i$. Matice musí být regulární, takže sloupce nezávislé, tedy tvoří bázi.

$R = (x_1\ x_2\ ...)$
\end{proof}
\end{veta}

\includegraphics[scale=0.5]{build/diagonalizace.PNG} 

\item Vyslovte a dokažte větu o diagonalizaci speciálních komplexních matic.


\begin{veta}
Každá hermitovská matice má všechna vlastní čísla reálná. Navíc $\exists R$ unitární, t.ž. $R^{-1}AR$ je diagonální
\begin{proof} Indukcí podle n.

Pro $n = 1$ věta platí.

\hfill

Označme $A_n = A$

V tělese $\mathbb{C}$ má matice $A_n$ vlastní číslo $\lambda$ s vlastním vektorem $x$.

Zvětšíme $x$ faktorem $\frac{1}{\sqrt{x^Hx}}$, abychom dostali $x^Hx=1$.

Doplníme $x$ na unitární matici $P_n$. (To je možné dle "faktu" z přednášky).

$P^H_nA_nP_n$ je hermitovská: $(P^H_nA_nP_n)^H = P^H_nA_n^H(P_n^H)^H = P^H_nA_nP_n$.

Protože $A_nx = \lambda x$, matice $A_nP_n$ má $\lambda x$ jako první sloupec.

Protože $P_n$ je unitární, první sloupec $P^H_nA_nP_n$ je $\lambda P^Hx = \lambda(1, 0, ..., 0)^T = (\lambda, 0, ... , 0)^T$. (součin unitární matice a hermitovské transpozice je jednotková matice)

Protože je daná matice hermitovská a prvek $\lambda$ leží na diagonále, $\lambda \in \mathbb{R}$, zbytek prvního řádku je taktéž nulový.
 
 Tedy $P^H_nA_nP_n =  \begin{array}{|c|c|} \hline  \lambda & 0^T  \\ \hline  0 & A_{n-1} \\ \hline \end{array}$, kde $A_{n-1}$ je hermitovská.
 
Podle indukčního předpokladu existuje $R_{n-1}$, t.ž. $R^{-1}_{n-1}A_{n-1}R_{n-1} = D_{n-1}$.

Položme $R_n = P_n \cdot\begin{array}{|c|c|} \hline  1 & 0^T  \\ \hline  0 & R_{n-1} \\ \hline \end{array}$.

Pak $R^{-1}_nA_nR_n = D_n$
 
 
\end{proof}
\end{veta}

\item Uved'te a dokažte Cauchy-Schwarzovu nerovnost.
\begin{veta}
Pro libovolné $u, v \in V$ nad $\mathbb{C}$:
\[ |\langle u | v \rangle | \leq ||u|| \cdot ||v|| \]
\begin{proof} Pro $u = 0$ nebo $v = 0$ platí. Jinak:

Pro $a \in \mathbb{C}$ platí $||u + av||^2 \geq 0$.

\[||u + av||^2 = \sqrt{\langle u+av|u+av \rangle}^2 
= \langle u|u \rangle  + a\langle v|u \rangle + \overline{a}\langle u|v \rangle  + a \overline{a}\langle v|v \rangle \]

Pro odečtení posledních dvou členů zvolme $a = - \frac{\langle u|v \rangle}{\langle v|v \rangle}$

Dostaneme 
\begin{align*}
0 &\leq \langle u|u \rangle  - \frac{\langle u|v \rangle}{\langle v|v \rangle} \langle v|u \rangle \\
 \frac{\langle u|v \rangle}{\langle v|v \rangle} \langle v|u \rangle  &\leq \langle u|u \rangle \\
 \langle u|v \rangle \langle v|u \rangle  &\leq \langle u|u \rangle  \langle v|v \rangle \\
 \langle u|v \rangle^2   &\leq ||u||^2 \cdot ||v||^2 \\
  \langle u|v \rangle   &\leq ||u|| \cdot ||v|| 
\end{align*}

\end{proof}
\end{veta}

\item Uved'te a dokažte trojúhelníkovou nerovnost.
\begin{veta}
Pro každý skalární součin:
\[ ||u+v|| \leq ||u|| + ||v|| \]
\begin{proof}
\[  ||u+v||  = \sqrt{\langle u +v | u + v \rangle} =
 \sqrt{\langle v | v \rangle + \langle u | v \rangle + \langle v | u \rangle + \langle u | u \rangle}\]
 
Součet dvou komplexně sdružených čísel je nejvýše dvojnásobek absolutní hodnoty.
 
 \[ \leq  \sqrt{||u||^2 + 2 |\langle u | v \rangle | +  ||v||^2}   \]
 
 Dle Cauchy-Schwarzovy nerovnosti
 
  \[ \leq  \sqrt{||u||^2 + 2 ||u|| \cdot ||v|| +  ||v||^2} = ||u|| + ||v||  \]
  
  Tedy 
  
  \[ ||u+v|| \leq ||u|| + ||v||  \]
\end{proof}
\end{veta}

\item Vyslovte a dokažte větu o Fourierových koeficientech.
\begin{veta}
Nechť $Z = \{v_1, ..., v_n\}$ je ortonormální báze $V$, pak
\[\forall u \in V: u = \langle u | v_1 \rangle v_1 + ... +  \langle u | v_n \rangle v_n \]
\begin{proof}
\[ u = \sum^n_{i=1} a_i v_i \implies\]
\[\implies  \langle u | v_j \rangle =  \left\langle \sum^n_{i=1} a_i v_i  | v_j \right\rangle = \sum^n_{i=1} a_i \langle  v_i  | v_j \rangle = a_j \]
\end{proof}
\end{veta}


\item Uved'te a dokažte správnost Gram-Schmidtovy ortonormalizace.

\begin{lemma}[o ortogonální projekci] Pro prostor $V$ s bází $Z$ $ \forall u \in V: \forall v_i \in Z: u - p_Z(u) \perp v_i $
\begin{proof}
\[   \langle u - p_Z(u) | v_i \rangle =  \left\langle u - \sum_{j=j}^n  \langle u | v_j \rangle v_j | v_i \right\rangle = \langle u | v_i \rangle - \sum_{j=1}^n  \langle u | v_j \rangle \langle v_j | v_i \rangle = 0  \]
\end{proof}
\end{lemma}

\begin{lemma}[o výměně] $V$ je vektorový prostor nad $\mathbb{K}$ a $X = \{v_1, ..., v_n\}$ je systém generátorů $V$.

Potom pro libovolný vektor $u \in V$ platí, že když lze $u$ vyjádřit jako $u = \displaystyle \sum^n_{i=1} a_1v_1$, tak pro každé $i$, kde $a_i \neq 0$ platí, že $X^\prime = (X \setminus v_i) \cup u$ také generuje $V$

\begin{proof}
\begin{equation*}
u = a_1v_1 + ... + a_nv_n \implies v_i = \frac1a_i (u - (a_1v_1 + ... + a_{i-1}v_{i-1}+  a_{i+1}v_{i+1} + ...  a_{n}v_{n}))
\end{equation*}
\begin{equation*}
\forall w \in V: w = \sum_{j=1}^n b_j v_j
\end{equation*}
V tomto součtu můžeme za $v_i$ nahradit vyjádření vyše, což bude lineární kombinace $X^\prime$
\end{proof}
\end{lemma}

\begin{veta} Gram-Schmidtova ortonormalizace převede libovolnou bázi prostoru $V$ se skalárním součinem na ortonormální bázi. 

In: báze $(u_1, ..., u_n)$

Out: ortonormální báze $(v_1, ..., v_n)$

for $i = 1, ..., n$

$w_i = u_i - \sum^{i-1}_{j = 1} \langle u_i | v_j \rangle v_j$

$v_i = \frac1{||w_i||} w_i$

end
\begin{proof}
\begin{itemize}
\item Dle lemmatu jsou všechny vektory po dvou kolmé
\item Díky druhému kroku jsou všechny vektory normální
\item Díky lemmatu o výměně: $\mathcal{L}(v_1, ..., v_{i-1}, v_i) = \mathcal{L}(v_1, ..., v_{i-1}, w_i) =  \mathcal{L}(v_1, ..., v_{i-1}, u_i) $ Tedy se jedná o bázi.
\end{itemize}
\end{proof}
\end{veta}


\item Vyslovte a dokažte větu o izometrii a normě.

\begin{veta}
Lineární zobrazení $f$ je izometrie $\iff ||u|| = ||f(u)||$ 
\begin{proof}\begin{itemize}
\item $\implies$ 

Vyplývá z definice normy
\item $\impliedby$

\[ ||u + aw||^2 = ||u||^2 + a \langle w|u \rangle + \overline{a}\langle u | w \rangle + a\overline{a} ||w||^2 \]

\[ ||f(u + aw)||^2 = ||f(u)||^2 + a \langle f(w)|f(u) \rangle + \overline{a}\langle f(u)| f(w) \rangle + a\overline{a} ||f(w)||^2 \]

Ale 

\[ ||u + aw|| = ||f(u + aw)|| \]
\[ ||u||^2 + a \langle w|u \rangle + \overline{a}\langle u | w \rangle + a\overline{a} ||w||^2 = ||f(u)||^2 + a \langle f(w)|f(u) \rangle + \overline{a}\langle f(u)| f(w) \rangle + a\overline{a} ||f(w)||^2  \]

Normy se rovnají

\[  a \langle w|u \rangle + \overline{a}\langle u | w \rangle +  =  a \langle f(w)|f(u) \rangle + \overline{a}\langle f(u)| f(w) \rangle \]

Pro $a = 1$
\[  \langle w|u \rangle + \langle u | w \rangle   =   \langle f(w)|f(u) \rangle + \langle f(u)| f(w) \rangle \]
Pro $a = i$
\[  \langle w|u \rangle -\langle u | w \rangle  =   \langle f(w)|f(u) \rangle -\langle f(u)| f(w) \rangle \]

Sečtením:

\[  \langle w|u \rangle   =   \langle f(w)|f(u) \rangle  \]
\end{itemize}\end{proof}
\end{veta}

\item Vyslovte a dokažte větu o izometrii a vlastnostech její matice.
\begin{veta}
Pro $V,W$ prostory konečné dimenze se skalárním součinem a $X,Y$ jejich ortonormální báze:

\[ f: V \to W \text{ je bijektivní izometrie } \iff [f]_{XY} \text{ je unitární}\]
\begin{proof}
Lineární bijekce implikuje stejnou dimenzi. Protože X,Y jsou ortonormální, platí
\[ \langle u | w \rangle = [w]_X^H[u]_X\]
\[ \langle f(u) | f(w) \rangle = [f(w)]_Y^H[f(u)]_Y = [w]^H_X[f]^H_{XY}[f]_{XY}[u]_X\]

Tedy \[ [w]_X^H[u]_X = [w]^H_X[f]^H_{XY}[f]_{XY}[u]_X\]
\[[f]^H_{XY} =[f]^{-1}_{XY}\]

$[f]_{XY}$ je unitární.

\end{proof}
\end{veta}

\item Vyslovte a dokažte větu o ortogonálním doplňku.

%@TODO overit, kterou myslel%
\begin{veta}
Pro prostor $W$ se skalárním součinem a konečnou dimenzí a podprostor $V$ platí \begin{enumerate}[label=(\alph*)]
\item $V = (V^\perp)^\perp$
\item $\dim(W) = \dim(V) + \dim(V^\perp)$
\end{enumerate}
\begin{proof}
Zvolíme ortonormální bázi $X$ prostoru $V$ a doplníme ji na ortonormální bázi $Z$ prostoru $W$

$Y = Z \setminus X \quad X = (x_1, ..., x_k) \quad Y = (y_1, ..., y_l)$

Každý $u \in \mathcal{L}(X) = V \perp$ ke každému $v \in \mathcal{L}(Y)$
\[ \langle u|v \rangle = \left\langle \sum^n_{i=1} a_ix_i | \sum^n_{j=1} b_jy_j \right\rangle =  \sum^n_{i=1}  \sum^n_{j=1} a_i \overline{b_j} \langle x_i | y_j \rangle = 0 \]

Tedy $ \mathcal{L}(Y) \subseteq V^\perp$

Abychom dokázali $ \mathcal{L}(Y) \supseteq V^\perp$, vezměme $w \in V^\perp$ a uvažme $[w]_Z$

$Z$ je ortonormální, takže koeficienty $[w]_Z$ jsou Fourierovy koeficienty dané skalárním součinem $w$ s prvky báze $Z$.
Protože $w \in V^\perp$, máme $\langle w|x_i \rangle i = 0$ pro každé $x_i \in X$. Všechny x-ové souřadnice jsou nulové, tedy \[w \in \mathcal{L}(Y) \implies V^\perp \subseteq \mathcal{L}(Y) \implies V^\perp = \mathcal{L}(Y)\]

\hfill

Nyní \[ \dim(V) + \dim(V^\perp) = |X| + |Y| = |Z| = \dim(W) \]
a \[(V^\perp)^\perp =  \mathcal{L}(Z \setminus Y) =  \mathcal{L}(X) = V  \]
\end{proof}
\end{veta}

\item Vyslovte a dokažte větu o skalárním součinu dvou vektorů a Gramově matici.


\begin{veta}
V je prostor se skalárním součinem a bází $X = (v_1, ..., v_n)$

Potom gramova matice $A,\ a_{ij} = \langle v_i | v_j \rangle$ splňuje

\[ \forall u,w \in V: \langle u | w \rangle = [w]_X^H A^T [u]_X \]
\begin{proof}
Označme $[u]_X = (\alpha_1, \alpha_2, ..., \alpha_n)^T$ a

$[w]_X = (\beta_1, \beta_2, ..., \beta_n)^T$

Pak
\[  \langle u | w \rangle = \left\langle \sum^n_{i=1} \alpha_i v_i | \sum^n_{j=1} \beta_j v_j \right\rangle = \sum^n_{i=1}\sum^n_{j=1} \alpha_i \beta_j  \langle v_i |  v_j \rangle = [w]_X^H A^T [u]_X \]

\end{proof}
\end{veta}


\item Vyslovte a dokažte větu o třech ekvivalentních podmínkách pro pozitivně definitní matice.

\begin{veta}
Pro hermitovskou matici $A$ jsou následující podmínky ekvivalentní:
\begin{enumerate}
\item $A$ je pozitivně definitní
\item $A$ má všechna vlastní čísla kladná
\item Existuje regulární $U$, t.ž. $A = U^HU$
\end{enumerate}
\begin{proof}$\ $
\begin{itemize}
\item[$1 \implies 2$] 
$A$ je hermitovská $\implies$ má vlastní čísla reálná.

Nechť $x$ je netriviální vlastní vektor, který odpovídá vlastnímu číslu $\lambda$.

Pak $0 < x^HAx =  \lambda x^Hx = \lambda \langle x | x \rangle$

$\langle x | x \rangle > 0 \implies \lambda > 0$

\item[$2 \implies 3$]
$A$ je hermitovská $\implies$ existuje unitární $R$ a diagonální $D$, t.ž. $A = R^HDR$.

Vezměme diagonální $D^\prime: d^\prime_{i,i} = \sqrt{d_{i,i}}$ a $U = D^\prime R$.

Nyní \[ U^HU =  R^H D^{\prime H} D^\prime R = R^HDR = A\]

$U$ je regulární, protože diagonální i unitární matice jsou regulární.
\item[$3 \implies 1$]
Pokud $x \in \mathbb{C}^n \setminus 0$, pak $Ux \neq 0$, protože $U$ je regulární.

$x^HAx = x^HU^H Ux = (Ux)^H Ux = \langle Ux | Ux \rangle > 0 $

\end{itemize}\end{proof}
\end{veta}

\item Vyslovte a dokažte větu o rekurentní podmínce pro pozitivně definitní matice.

\begin{veta}
Bloková matice $A = \begin{array}{|c|c|} \hline  \alpha & a^H  \\ \hline  a & \tilde{A} \\ \hline \end{array}$ je pozitivně definitní právě tehdy, když $\alpha > 0$ a $\tilde{A} - \frac1\alpha a a^H$ je pozitivně definitní
\begin{proof} $\ $
\begin{itemize}
\item[$\impliedby$]
Pro $x \in \mathbb{C}^n \setminus 0 \quad x^T = \begin{array}{|c|c|} \hline x_1 & \tilde{x}^T \\ \hline \end{array}$


\[ x^HAx = \begin{array}{|c|c|} \hline \overline{x_1} & \tilde{x}^H \\ \hline \end{array} \cdot \begin{array}{|c|c|} \hline  \alpha & a^H  \\ \hline  a & \tilde{A} \\ \hline \end{array} \cdot \begin{array}{|c|} \hline x_1  \\ \hline \tilde{x}^T \\ \hline \end{array} \]

\[ = \alpha x_1 \overline{x_1} + x_1 \tilde{x}^H a + \overline{x_1}a^H \tilde{x} + \tilde{x}^H \tilde{A} \tilde{x} \]
\[ = \alpha x_1 \overline{x_1} + x_1 \tilde{x}^H a + \overline{x_1}a^H \tilde{x} +{ \color{blue} \tilde{x}^H \tilde{A} \tilde{x} } +  \frac1\alpha\tilde{x}^Haa^H\tilde{x} { \color{blue} - \frac1\alpha\tilde{x}^Haa^H\tilde{x} } \]\[
{ \color{blue} \tilde{x}^H (\tilde{A} - \frac{1}{\alpha} aa^H)\tilde{x} } + (\sqrt{\alpha}\overline{x_1} + \frac1{\sqrt{\alpha}}\tilde{x}^H a)(\sqrt{\alpha}x_1 + \frac1{\sqrt{\alpha}}a^H\tilde{x} ) \]

Oba sčítance jsou kladné: První je z indukčního předpokladu pozitivně definitní matice, druhý protože se jedná o komplexně sdružená čísla.

Navíc alespoň jeden ze sčítanců musí být vždý ryze kladný: \begin{enumerate}
\item $\tilde{x} = 0 \implies x_1 \neq 0$, druhý sčítanec je nenulový
\item $\tilde{x} \neq 0$, první sčítanec je nenulový.

Tedy matice je pozitivně definitní.
\end{enumerate}
\item[$\implies$] $\tilde{x} \in \mathbb{C}^{n-1} \setminus 0$, vezmeme $x_1 = - \frac1\alpha a^H \tilde{x}$.

$x^T = \begin{array}{|c|c|}\hline x_1 & \tilde{x}^T \\ \hline \end{array}$
\hfill


Dle našeho výběru: \[ \sqrt{\alpha}x_1 + \frac{1}{\sqrt{\alpha}} a^H \tilde{x} = 0\]

Nyní \[ 0 < x^HAx = \tilde{x}^H(\tilde{A} - \frac{1}{\alpha} a a^H) \tilde{x} + 0 \cdot 0\]

A taky \[ {e^1}^HAe^1 = \alpha > 0\]

Takže $\tilde{A} - \frac1\alpha a a^H$ je pozitivně definitní a $\alpha > 0$

\end{itemize}

\end{proof}
\end{veta}

\item Vyslovte a dokažte větu o pozitivně definitních maticích a determinantech.


\begin{veta} [Sylvesterova podmínka]
Matice $A$ řádu $n$ je pozitivně definitní právě tehdy, když matice $A_1, A_2, ..., A_n$ mají kladný determinant.

Kde $A_i$ je hlavní podmatice $A$, neboli prvních $i$ řádků a sloupců $A$.
\begin{proof}
Použijeme Gaussovu eliminaci $A \sim ... \sim A^\prime$ pro test, zda je $A$ pozitivně definitní. Nechť $\alpha_1, ..., \alpha_n$ jsou prvky na diagonále $A^\prime$. Protože jsme eliminovali řádky shora dolů, máme
\[ \det(A_i) = \det(A^\prime_i) = \prod_{j \leq i} \alpha_j = \det(A_{i-1})\alpha_i \]

\[A \text{ je pozitivně definitní } \iff \alpha_1, ..., \alpha_n > 0 \iff \det(A_1), ..., \det(A_n) > 0 \]
\end{proof}
\end{veta}

\item Uved'te a dokažte správnost algoritmu pro výpočet Choleského rozkladu.

 
\begin{veta}
Pro každou pozitivně definitní matici $A$ exituje unikátní horní trojúhelníková matice $U$ s kladnou diagonálou, t.ž. $A = U^HU$. $U$ nazýváme Choleského rozklad.


\includegraphics[scale=0.5]{build/choleskeho_rozklad.png}
\begin{proof}$\ $
 \includegraphics[scale=0.5]{build/spravnost_choleskeho.png} 
\end{proof}
\end{veta}

\item Vyslovte a dokažte větu o diagonalizovatelnosti matic forem.

\begin{veta}
Pokud je $g$ kvadratická forma na $V$ dimenze $n$ nad $\mathbb{K}$, $char(\mathbb{K} \neq 2$, pak má forma diagonální matici vzhledem k vhodné bázi $X$.

Neboli: Pro symetrickou $A \in \mathbb{K}^{n \times n}$, $char(\mathbb{K}) \neq 2$ existuje regulární $R$, t.ž. $R^TAR$ je diagonální.
\begin{proof} 
Indukcí podle n.

\[ A = A_n = \begin{array}{|c|c|} \hline  \alpha & a^T  \\ \hline  a & \tilde{A} \\ \hline \end{array} \] \begin{itemize}

\item Když $\alpha \neq 0$

Volíme $P_n = \begin{array}{|c|c|} \hline  1 & 0^T  \\ \hline  -\frac{1}{\alpha}a & I_{n-1} \\ \hline \end{array} $

Pak 

\[P_nA_nP_n^T =  \begin{array}{|c|c|} \hline  \alpha & 0^T  \\ \hline  0 & A_{n-1} \\ \hline \end{array} \]

Kde $A_{n-1} = \tilde{A} - \frac{1}{\alpha} aa^T$ je symetrická.

Dle indukčního předpokladu $\exists R_{n-1}$ pro $A_{n-1}$

Zvolíme $R_n = \begin{array}{|c|c|} \hline  1 & 0^T  \\ \hline 0 & R_{n-1} \\ \hline \end{array} \cdot P_n$

Pak $ R_nA_nR_n^T = \begin{array}{|c|c|} \hline  \alpha & 0^T  \\ \hline 0 & R_{n-1}A_{n-1}R_{n-1}^T \\ \hline \end{array}  $ je diagonální.

\item $\alpha = 0$ a $a \neq 0$, pak $a_{i,1} \neq 0$ pro nějaké $i$.

Použijeme elementární matici $E$ pro přičtení $i$-tého řádku k prvnímu, vezmeme $A^\prime = EAE^T$ namísto $A$ a můžeme postupovat jako v prvním případě.

\item $\alpha = 0$ a $a = 0$, vezmeme $A_{n-1} = \tilde{A}$ a $R_n = \begin{array}{|c|c|} \hline  1 & 0^T  \\ \hline 0 & R_{n-1} \\ \hline \end{array} $

\end{itemize}
\end{proof}
\end{veta}

\item Uved'te a dokažte Sylvesterův zákon setrvačnosti o diagonalizaci kvadratických forem.
\begin{veta}
Každá kvaddratická forma na konečně generovaném reálném prostoru má vzhledem k vhodné bázi diagonální matici pouze s $-1$, $1$ a $0$.
\begin{proof}
\begin{enumerate}
\item \textbf{Existence}

$B$ je matice formy vzhledem k nějaké bázi $X$.

Reálné symetrické matice lze diagonalizovat.

$B = R^TDR$ pro $R$ regulární.

Nyní $D = S^TD^\prime S$ kde $d_{i,i} \begin{cases} 
= 0 & d^\prime_{i,i} = 0 \quad s_{i,i} = 1 \\
> 0 & d^\prime_{i,i} = 1 \quad s_{i,i} = \sqrt{d_{i,i}} \\
< 0 & d^\prime_{i,i} = -1 \quad s_{i,i} = \sqrt{-d_{i,i}} \end{cases}$

$SR$ je regulární.

\[ B = (SR)^TD^\prime(SR)\]

Zvolíme bázi $Y$: souřadnice vektorů $Y$ vzhledem k $X$ jsou sloupce $SR$.

\[ [id]^T_{XY}B[id]_{XY} = ((SR)^{-1})(SR)^TD^\prime(SR)(SR)^{-1}  = D^\prime\]

\item \textbf{Jednoznačnost počtu $\pm 1$}

Nechť $X,Y$ jsou dvě báze: $X = (u_1, ..., u_n )$, $Y = (v_1, ..., v_n)$, t. ž. odpovídající matice $B$ a $B^\prime$ formy $g$ jsou diagonální s $1, -1, 0$ uspořádanými tak, že nejdříve jsou $1$, pak $-1$ a pak $0$.

Protože součiny s regulární maticí $[id]_{XY}$ nemění hodnost, platí, že počty nul v obou maticích jsou shodné, a sice: $\# 0 = n - rank(B)$

\hfill

Nechť $r = \#0$ v $B$ a $s = \#0$ v $B^\prime$

Pokud $r > s$, uvažme podprostory $\mathcal{L}(v_1, ..., v_r)$ a  $\mathcal{L}(v_{s+1}, ..., v_n)$

Součet jejich dimenzí přesahuje $n$, mají tedy netriviální průnik. 

Zvolme $w \in \mathcal{L}(v_1, ..., v_r) \cap \mathcal{L}(v_{s+1}, ..., v_n) \setminus 0$.

\[[w]_X = (x_1, ..., x_r, 0, ..., 0)^T\]

\[[w]_Y = ( 0, ..., 0, y_{s+1}, ..., y_n)^T\]

\hfill

Nyní

\[ g(w) = [w]_X^TB[w]_X = x_1^2 + \cdot x_r^2 \geq 0 \]
\[ g(w) = [w]_X^TB[w]_X = -y_{s+1}^2 - \cdot -y_n^2 \leq 0 \]

\hfill

\[ 0 \leq g(w) \geq 0 \]

Spor, tedy $r \ngtr s$. Symetricky $r \nless s$, tedy $r = s$. Počet $-1$ je doplněk do řádu matice, takže taky shodný pro obě matice. 
 
\end{enumerate}
\end{proof}
\end{veta}

\item Vyslovte a dokažte větu o počtu přímek pod stejným úhlem.

\begin{veta}
V $\mathbb{R}^d$ může nejvíce $\binom{d+1}{2}$ přímek svírat stejný úhel 
\begin{proof}
Předpokládejme, že existuje $n$ takových přímek a svírají úhel $\phi$. Zvolíme jednotkové vektory $v_1, ..., v_n$, z každé přímky jeden.

Dostaneme $\langle v_i | v_j \rangle = \begin{cases} 1 & i=j \\ \cos \varphi & i \neq j \end{cases}$

\hfill

Ukážeme, že matice $v_1v_1^T, v_2v_2^T, ..., v_nv_n^T$ jsou lineárně nezávislé. Pak nutně $n \leq \binom{d+1}{2}$, protože dimenze prostoru symetrických matic v $\mathbb{R}^{d \times d}$ je $\binom{d+1}{2}$.

\hfill

Předpokládejme $\sum^n_{i=1}a_iv_iv_i^T = 0$.

Pro každé $j \in 1, ..., n$:

\[ 0 = v_j^T0v_j = v_j^T\left( \sum^n_{i=1}a_iv_iv_i^T \right) v_j \]
\[=  \sum^n_{i=1}a_i v_j^Tv_iv_i^Tv_j = \sum^n_{i=1}a_i \langle v_j | v_i \rangle \langle v_i | v_i \rangle
\]

Protože jsme v $\mathbb{R}$

\[ \sum^n_{i=1}a_i \langle v_j | v_i \rangle^2 = a_j + \cos^2 \varphi \sum_{i \neq j} a_i
\]

To nám dá soustavu
\[
\begin{pmatrix}
1 & \cos^2 \varphi & \cdots &  \cos^2 \varphi \\
\cos^2 \varphi & 1 &  \ddots & \cos^2 \varphi\\
\vdots &  \ddots & \ddots &  \vdots \\
\cos^2 \varphi & \cdots &  \cos^2 \varphi & 1 
\end{pmatrix}\begin{pmatrix}
a_1 \\ a_2 \\ \vdots \\ a_n
\end{pmatrix} = 0
\]

Tato matice je regulární, tedy $a = 0_n$.
Tudíž jsou matice lineárně nezávislé.

\end{proof}
\end{veta}


\end{enumerate}


\part{Přehledové otázky}


(U přehledových otázek uved'te definice, tvrzení, věty, příklady a souvislosti. Důkazy u přehledových otázek nejsou vyžadovány.)
\begin{enumerate}
\item Přehledově sepište, co víte o výpočtu determinantů.
\item Přehledově sepište, co víte o determinantech a jejich geometrickém významu.
\item Přehledově sepište, co víte o počtu koster grafu.
\item Přehledově sepište, co víte o polynomech.
\item Přehledově sepište, co víte o vlastních číslech a vlastních vektorech.
\item Přehledově sepište, co víte o charakteristickém polynomu a jeho koeficientech.
\item Přehledově sepište, co víte o podobných maticích a diagonalizaci.
\item Přehledově sepište, co víte o speciálních komplexních maticích.
\item Přehledově sepište, co víte o skalárním součinu a související normě.
\item Přehledově sepište, co víte o ortogonalitě a kolmé projekci.
\item Přehledově sepište, co víte o ortonormálních bazích.
\item Přehledově sepište, co víte o ortogonálním doplňku.
\item Přehledově sepište, co víte o pozitivně definitních maticích.
\item Přehledově sepište, co víte o bilineárních a kvadratických formách a jejich maticích.
\end{enumerate}


\end{document}