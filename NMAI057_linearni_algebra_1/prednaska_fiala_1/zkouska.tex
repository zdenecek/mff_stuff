\documentclass[10pt,a4paper]{article}

\usepackage{amssymb, amsthm, amsmath, amsfonts}
\usepackage[margin=0.7in]{geometry}
\usepackage{array, xcolor}
\usepackage{enumitem}
\usepackage{graphicx}

\usepackage[czech]{babel}
\usepackage[utf8]{inputenc}

\setlength{\parindent}{0em}

\title{Zpracování vybraných otázek ke zkoušce z LA1}
\date{}
\author{Zdeněk Tomis}

\begin{document}

\maketitle

"Seznam  není  vyčerpávající  ani  povinný  —  občas  může  být  požadována  i  znalost pojmů a skutečností, které nejsou v tomto seznamu uvedeny."

\hfill

V závorce uvádím číslo přednášky, na které se pojem řešil.

\part{Definice (3 otázky)}

\begin{enumerate}
\item Definujte rozšířenou matici soustavy. (1)

\paragraph{Rozšířená matice soustavy}
Pro soustavu $Ax = b$, kde $A$ je matice soustavy, $x$ je vektor neznámých a $b$ je vektor pravých stran, je rozšířená matice soustavy $\left(\begin{array}{c|c}
A & b
\end{array}\right)$

\item Definujte elementární řádkové operace. (1)
\paragraph{Elementární řádkovou úpravou} vznikne z matice $A$ matice $A^\prime$, a to buď

\begin{enumerate}
\item vynásobením i-tého řádku nenulovým reálným číslem $t$
\item přičtením $j$-tého řádku k $i$-tému, $i \neq j$
\end{enumerate}
Z těchto úprav lze odvodit také:	
\begin{enumerate}
\item[(c)] přičtení t-násobku j-tého řádku k i-tému, $j \neq i$
\item[(d)] prohození dvou řádků
\end{enumerate}

\item Definujte odstupňovaný tvar matice. (1)
\paragraph{Odstupňovaný tvar}

Označme $j(i) = min(\{j: a_{i,j} = 0\})$

Matice $A \in \mathbb{R}^{m \times n}$ je v odstupňovaném tvaru právě tehdy, když $\exists r \in \{1, ..., m \}$, takové že

\begin{enumerate}
\item $j(1) < j(2) < ... < j(r) \leq n$  neboli $\forall k,l, k < l \leq r: j(k) < j(l)$ (Nenulové řádky jsou seřazeny podle počtu počátečních nul, který je u každého řádku jiný)
\item $\forall i,j,  m \geq i > r, 1 \leq j \leq n: a_{i, j} = 0$ (nulové řádky jsou až za nenulovými)
\end{enumerate}

\item Napište pseudokód pro Gaussovu eliminaci. (1)

\paragraph{Pseudokód pro Gaussovu eliminaci}

\begin{enumerate}
\item[1.] Seřaď řádky podle počtu počátečních nul.
\item[2.] Pokud mají dva nenulové řádky stejný počet počátečních nul ($i$-tý a $i+1$-ní), tak od $i+1$-ního odečteme $\frac{a_{i+1, j(i)}}{a_{i, j(i)}}$-násobek $i$-tého.
\item[3.] Opakuj, dokud nemají každé dva nenulové řádky různé počty počátečních nul.
\end{enumerate}

Algoritmus je konečný, protože po kroku 2. vždy vzroste celkový počet počátečních nul alespoň o jedna.

\item Definujte volné a bázické proměnné. (2)

\paragraph{Bázické proměnné} soustavy s maticí soustavy $A$ jsou ty proměnné, jimž odpovídají sloupce, ve kterých je v odstupňovaném tvaru $A$ pivot. Volné proměnné jsou všechny ostatní.

\item Definujte hodnost matice. (2)

Zopakovat definici pivotu

\paragraph{Hodnost matice} je počet pivotů v matici v odstupňovaném tvaru, kterou lze z matice získat.

\item Definujte jednotkovou matici. (3)

\paragraph{Jednotková matice} Jednotková matice $I_n \in \mathbb{R}^{n \times n}$ je taková matice pro kterou platí:


\begin{equation*}
(I_n)_{i,j} = \begin{cases} 
1 &\text{pokud }  i = j \\
0 &\text{pokud }  i \neq j
\end{cases}
\end{equation*}

\item Definujte transponovanou matici. (3)

\paragraph{Transponovaná matice} k matici $A \in \mathbb{R}^{m \times n}$ je taková matice $A^T \in \mathbb{R}^{n \times m}$, pro kterou platí:
\begin{equation*}
A^T_{i,j} = A_{j,i}
\end{equation*}
 
\item Definujte symetrickou matici. (3)

\paragraph{Symetrická matice} je taková čtvercová matice $A \in \mathbb{R}^{n \times n}$, pro kterou platí:

\begin{equation*}
A_{j,i} = A_{i, j} \text{, neboli } A = A^T
\end{equation*}

\item Definujte maticový součin. (3)

\paragraph{Maticový součin}: Pro součin matic $A \in \mathbb{R}^{m \times n}$ a $B \in \mathbb{R}^{n \times o}$ platí:

\begin{equation*}
(AB)_{i,j} = \displaystyle \sum_{k = 1}^n a_{i,k} \cdot b_{k,j}
\end{equation*}

\item Definujte inverzní matici. (4)

\paragraph{Inverzní matice} k čtvercové matici $A \in \mathbb{R}^{n \times n}$  je taková matice $A^{-1} \in \mathbb{R}^{n \times n}$, pro kterou platí:

\begin{equation*}
A \times A^{-1} = I_n 
\end{equation*}
 
\item Definujte regulární matici. (4)

\paragraph{Regulární matice} je taková matice, ke které existuje inverzní matice.

Je nutno zopakovat definici inverzní matice.

\item Definujte binární operaci. (4)

\paragraph{Binární operace na množině $X$} je zobrazení $X \times X \to X$. 

Případně lze definovat zobrazení (pomocí relace) a relaci (pomocí kartézského součinu).

\item Definujte komutativní a asociativní binární operace. (4-5)

Binární operace $\circ$ na množině $X$, tj. zobrazení $\circ: X \times X \to X$, je


\paragraph{asociativní,} pokud 
\begin{equation*}
\forall a,b,c \in X: (a \circ b) \circ c = a \circ (b \circ c)
\end{equation*}

\paragraph{komutativní,} pokud 
\begin{equation*}
\forall a,b \in X: a \circ b = b \circ a
\end{equation*}
 
\item Definujte neutrální prvek. (5)

\paragraph{Neutrální prvek} v množině $X$ vzhledem k binární operaci $\circ$ je takové $e \in X$, pro které platí: $\forall x \in X: x \circ e = e \circ x = x$

\item Definujte inverzní prvek. (5) 
\paragraph{Inverzní prvek} k prvku x v množině $X$ vzhledem k binární operaci $\circ$ je takové $x^{-1} \in X$, pro které platí: $x \circ x^{-1} = x^{-1} \circ x = e$, kde $e$ je neutrální prvek v $X$ vzhledem k $\circ$

\item Definujte grupu. (5)

\paragraph{Grupa} je dvojice $(G, \circ)$, kde $G$ je množina a $\circ$ je binární operace na $G$, která splňuje následující:

\begin{enumerate}
\item Neutrální prvek: $\exists e \in G: \forall x \in G: x \circ e = e \circ x= x$
\item Inverzní prvek: $\forall x \in G: \exists x^{-1} \in G: x \circ x^{-1} = x^{-1} \circ x = e$
\item Asociativita: $\forall a,b,c \in G:  (a \circ b) \circ c = a \circ (b \circ c)$
\end{enumerate}

\item Definujte permutaci. (5)

\paragraph{Permutace} je bijekce (injektivní/prosté a surjektivní/na zobrazení) na $\{1, 2, ..., n\}$
 
\item Definujte transpozici. (5)

\paragraph{Transpozice} je permutace na množině velikosti $n$, která má jeden cyklus délky $2$ a $n-2$ pevných bodů.

\item Definujte inverzi v permutaci. (5)

\paragraph{Inverze v permutaci} je taková dvojice prvků $(i, j)$, pro které platí $i < j$, $p(i) > p(j)$.

\item Definujte znaménko permutace.  (5)

\paragraph{Znaménko permutace} $p$ je číslo $sgn(p) = (-1)^{\# \text{ počet inverzí v } p}$.

\item Definujte těleso. (5-6)

\paragraph{Těleso} je taková trojice $(T, \oplus, \otimes)$, pro kterou platí:

\begin{enumerate}
\item $(T, \oplus)$ tvoří Abelovu grupu
\item $(T \setminus \{0\}, \otimes)$, kde $0$ je neutrální prvek k $\oplus$, tvoří Abelovu grupu
\item $0$ komutuje s $\otimes$
\item platí distributivní zákon $\otimes$ na $\oplus$:
$\forall a,b,c \in T: a \otimes (b \oplus c) = (a \otimes b) \oplus (a \otimes c)$ 
(Lze rozepsat i pro násobení zprava)

\end{enumerate}

\item Definujte charakteristiku tělesa. (6)

\paragraph{Charakteristika tělesa} je $\begin{cases} 
\text{nejmenší } n & \text{pro které platí:} \underbrace{1+1+...+1}_{\text{n-krát}} = 0
\\ 0 & \text{pokud takové n neexistuje}
 \end{cases}$

\item Definujte vektorový prostor. (6-7)

\paragraph{Vektorový prostor} nad tělesem $\mathbb{K}$ je množina $V$ spolu se zobrazeními $\oplus: V \times V \to V$ a $\odot: \mathbb{K} \times V \to V$, pokud platí:

\begin{enumerate}
\item $(V, \oplus)$ tvoří Abelovu grupu
\item $\forall u \in V: 1 \odot u = u$, kde 1 je neutrální prvek pro násobení v $\mathbb{K}$
\item asociativita násobení: $\forall a,b \in \mathbb{K}, u \in V: (a \cdot b) \odot u = a \odot (b \odot u)$
\item distributivita na obou sčítání:
\begin{itemize}
\item $\forall a \in \mathbb{K}, u,v \in V: a \odot (u \oplus v) = (a \odot u) \oplus (a \odot v)$
\item $\forall a,b \in \mathbb{K}, w \in V: (a+b) \odot w = (a \odot w) \oplus (b \odot w)$
\end{itemize}
\end{enumerate}

\item Definujte podprostor vektorového prostoru. (7)

Nechť $(V, \oplus, \odot)$ je vektorový prostor nad $\mathbb{K}$ a $U \subseteq V, U \neq \emptyset$ a 
\begin{enumerate}
\item $U$ je uzavřená na $\oplus$,
\item $U$ je uzavřená na $\odot$,
\end{enumerate}

potom se $U$ nazývá podprostor $V$. 

\item Definujte lineární kombinaci. (7)

\paragraph{Lineární kombinace} vektorů $v_1, ..., v_n$ je libovolný vektor, který lze zapsat ve tvaru $a_1 \cdot v_1 + \cdots + a_n \cdot v_n$, kde $a_1, ..., a_n$ jsou prvky tělesa.

\item Definujte lineární obal (podprostor generovaný množinou). (7)

\paragraph{Lineární obal} množiny $X \subseteq V$, kde $V$ je libovolný vektorový prostor nad $\mathbb{K}$, je průnik všech podprostorů $V$, které obsahují $X$.
 
\item Definujte řádkový a sloupcový prostor matice.

Řádkový/sloupcový prostor matice je prostor generovaný jejími řádky/sloupci.

Formálně:
Pro matici $A \in \mathbb{K}^{m \times n}$

\begin{equation*}
\mathfrak{S}(A) = \{u: u \in \mathbb{K}^n: u = Ax: x \in \mathbb{K}^n\}
\end{equation*}
Neboli všechny lineární kombinace sloupců.

\begin{equation*}
\mathfrak{R}(A) = \{u: u \in \mathbb{K}^m: u = A^Tx: x \in \mathbb{K}^m\}
\end{equation*}
Neboli všechny lineární kombinace řádků.

\item Definujte jádro matice.

Jádro matice  $A \in \mathbb{K}^{m \times n}$ je podprostor  $ \mathbb{K}^n$ tvořený řešeními homogenní soustavy $Ax =0$. Značí se $Ker(A)$

\item Definujte lineárně nezávislé vektory. (8)

Množina vektorů $X$ ve vektorovém prostoru $V$ je lineárně nezávislá, pokud nelze z $X$ vybrat $k$ vektorů $v_1, ..., v_k$ a sestavit skaláry $a_1, ..., a_k$ tak, že alespoň jedno $a$ je různé od nuly a $\displaystyle \sum a_i v_i = 0$

Nebo také

\begin{equation*}
\sum_{v_i \in V} a_i v_i = 0 \implies \forall i: a_i = 0
\end{equation*}

\item Definujte bázi vektorového prostoru. (8)

\paragraph{Báze vektorového prostoru} $V$ je libovolná množina $X \subseteq V$, taková že:

\begin{enumerate}
\item $\mathfrak{L}(X) = V$
\item $X$ je lineárně nezávislá
\end{enumerate}

\item Definujte dimenzi vektorového prostoru. 

Nechť má $V$ konečnou bázi. Potom je dimenze $V$ mohutnost jeho báze. Značíme $dim(V)$.

\item Definujte vektor souřadnic.
\paragraph{Vektor souřadnic}
Nechť $X = (v_1, v_2, ..., v_n)$ je konečná uspořádaná báze prostoru $V$ nad tělesem $\mathbb{K}$. Pro libovolný vektor $u \in V$ nazveme koeficienty $(a_1, ..., a_n)^T \in \mathbb{K}^n$ z jednoznačného vyjádření $u = \displaystyle \sum^n_{i=1} a_i v_i$ vektorem souřadnic vektoru $u$ vzhledem k bázi $X$ a značíme jej $[u]_X$.

\item Definujte lineární zobrazení. 
Nechť $V$ a $W$ jsou vektorové prostory nad stejným tělesem $\mathbb{K}$. Potom zobrazení $f: V \to W$ se nazývá lineární zobrazení, pokud platí:
\begin{enumerate}
\item $\forall u,v \in V: f(u+v) = f(u) + f(v)$
\item $\forall u \in V, \forall a \in \mathbb{K}: f(a\cdot v) = a\cdot f(u)$
\end{enumerate}

\item Definujte matici lineárního zobrazení. 

Nechť $V$ a $W$ jsou vektorové prostory nad stejným tělesem $\mathbb{K}$ a $X = (v_1, ..., v_n), Y=(w_1, ..., w_m)$ jsou jejich báze. Potom pro lineární zobrazení $f$ nazveme matici $[f]_{XY} \in \mathbb{K}^{m \times n}$  sestavenou z vektorů souřadnic obrazů vektorů báze X vůči Y maticí f vzhledem k bázím X a Y.

Formálně:

\begin{equation*}
[f]_{XY} = \left( [f(v_1)]_Y, ..., [f(v_n	)]_Y \right)
\end{equation*}

\item Definujte matici přechodu. ??

Nechť V je prostor a X a Y jsou jeho dvě konečné báze. Maticí přechodu od báze X k bázi Y nazveme matici $[id]_{XY}$, kde id je identické zobrazení.
 

\item Definujte izomorfismus vektorových prostorů.

Nechť $V$ a $W$ jsou vektorové prostory nad stejným tělesem $\mathbb{K}$. Potom lineární zobrazení $f: V \to W$, které je prosté a na, nazýváme izomorfismem prostorů V a W.
 
\item Definujte afinní prostor a jeho dimenzi.

Afinním prostorem $A = A(V)$ nad vektorovým prostorem V rozumíme trojici ${A, V, +}$, kde A je množina, jejíž prvky nazýváme body, V je vektorový prostor, + je operace, která bodu a vektoru přířadí bod: $+: A \times V \to A$ splňující následující axiomy:
\begin{enumerate}
\item $\forall a \in A: a + 0 = a$
\item $\forall a \in A, \forall u,w \in V: a + (u + w) = (a + u) + w$
\item Ke každé dvojici bodů $a, b \in A \exists! v \in V: a+v = b$. Tento vektor značíme $b - a$
\end{enumerate}

\end{enumerate}

\newpage
\part{Věty}
\begin{enumerate}


\item Uved’te a dokažte vztah mezi elementárními řádkovými operacemi a soustavami rovnic.


Mějme soustavy $Ax = b$ a $A^\prime x = b^\prime$, takové že $(A|b) \sim \sim (A^\prime|b^\prime)$, potom tyto soustavy mají stejné řešení.

\begin{proof}
Stačí dokázat u úprav (a) a (b), neboť dle definice jsou (c) i (d) od těchto dvou odvozené.

\begin{enumerate}
\item Vynásobení $i$-tého řádku nenulovým skalárem $t$.
\begin{enumerate}
\item $X^\prime \subseteq X$
\item $X \subseteq X^\prime$ 
\end{enumerate}

\item Přičtení $j$-tého řádku k $i$-tému
\begin{enumerate}
\item $X^\prime \subseteq X$
\item $X \subseteq X^\prime$ 
\end{enumerate}

\end{enumerate}

Nutno vypsat úpravy

\end{proof}

\item Vyslovte a dokažte větu o jednoznačnosti volných a bázických proměnných.

Pro každou matici $A$ platí, že sloupce s pivoty v libovolné matici $A^\prime$ v odstupňovaném tvaru, kterou lze získat z A elementárními řádkovými úpravami, jsou určeny jednoznačně.

\begin{proof}
Sporem. Nechť $A^\prime, A^{\prime\prime} \sim \sim A$, platí ale $(A^\prime|0) \sim \sim(A^{\prime\prime}|0) \sim \sim (A|0)$ t. j. tyto soustavy mají shodná řešení. Nechť je bez újmy na obecnosti $x_k$ bázická v $(A^\prime|0)$ a volná v $(A^{\prime \prime}|0)$, a všechny ostatní promměné mají stejný charakter v obou maticích.
Zafixujme hodnoty $x_j$ pro $j > k$, potom je ale \begin{itemize}
\item v $(A^\prime|0)$ $x_k$ určena jednoznačně,
\item v $(A^{\prime \prime}|0)$ může být $x_k$ libovolná
\end{itemize} 

To je spor, neboť obě soustavy mají shodná řešení.
\end{proof}


\item Vyslovte a dokažte Frobeniovu větu.

\paragraph{Frobeniova věta} Soustava $Ax = b$ má řešení právě tehdy, když $rank(A) = rank((A | b))$
\begin{proof}
Vezměme libovolnou $(A^\prime|b^\prime)$  v odstupňovaném tvaru, t. ž. $(A^\prime|b^\prime) \sim \sim (A | b)$. Pak platí:

$Ax = b$ má řešení $\iff (A^\prime|b^\prime)$ nemá pivot v sloupci $b^\prime \iff$ pivoty $A^\prime$ jsou pivoty $(A^\prime|b^\prime) \iff rank(A^\prime) = rank((A|b))$ 
\end{proof}

\item Vyslovte a dokažte větu o vztahu mezi řešeními Ax = b a Ax = 0. 

Nechť $x_0$ je libovolným řešením nehomogenní soustavy $Ax = b$. Potom zobrazení $g: \overline{x} \to \overline{x} + x_0$ je bijekcí  mezi množinami řešení  soustav $Ax = 0$ a $Ax = b$.

\begin{proof} 
Značení: $V \subseteq \mathbb{R}^n$ množina řešení $Ax = 0$

$W \subseteq \mathbb{R}^n$ množina řešení $Ax = b$

Definujeme: $g: V \to \mathbb{R}^n$ předpisem $g(\overline{x})= \overline{x} + x_0$

$h: W \to \mathbb{R}^n$ předpisem $g(x)= x - x_0$

potom $h \circ g$ je identita na $V \implies g$ je prosté.
$g \circ h$ je identita na $W \implies g$ je na. 

$g$ je tedy bijekce.
\end{proof}

\item Uved’te a dokažte větu popisující všechna řešení Ax = b.

Nechť A je matice řádu $m \times n$ a hodnosti $r$. Potom všechna řešení soustavy $Ax = 0$ lze vyjádřit ve tvaru

\begin{equation*}
x = p_1x_1 + p_2x_2 + ... + p_{n - r}x_{n-r}
\end{equation*}

kde $p_i \in \mathbb{R}$ a $x_j$ jsou vhodné řešení soustavy $Ax =0$

Soustava má pouze triviální řešení $x = 0$ právě tehdy, když A je hodnosti $n$.

Důsledek: Obecné řešení soustavy $Ax = b$ lze vyjádřit ve tvaru:

\begin{equation*}
x = x_0 +p_1x_1 + p_2x_2 + ... + p_{n - r}x_{n-r}
\end{equation*}

kde $x_0$ je libovolné řešení soustavy $Ax = b$, $r$ je hodnost A, $p_i \in \mathbb{R}$ a $x_j$ jsou vhodné řešení soustavy $Ax =0$

\begin{proof}
Řešení získané zpětnou substitucí závisí na $n-r$ hodnotách volných proměnných. Označme je parametry $p_1, ..., p_{n-r}$.

Potom řešení x lze vyjádřit 	lineárním výrazem:

\begin{equation*}
\begin{array}{c}
x_1 = \alpha_{11}p_1 + ... + \alpha_{1, n-r}p_{n-r} \\
\vdots \\
x_n = \alpha_{n1}p_1 + ... + \alpha_{n, n-r}p_{n-r} \\
\end{array}
\end{equation*}

Označíme $x_1 = (\alpha_{11}, ..., \alpha_{n,1})^T ... x_{n-r} = (\alpha_{1,n-r}, ..., \alpha_{n,n-r})^T $

Tyto vektory ale zároveň řeší soustavu $Ax = 0$, protože libovolné $x^i$ lze získat volbou parametrů $p_i = 1, p_j = 0, j \neq i$

Druhá část věty: (není nutno dokázat)
$rank(A) = n \iff$ soustava nemá volné proměnné $\iff$ existuje právě jedno řešení soustavy $Ax =0$, a to $x = 0$. 

Důkaz důsledku: Použijeme bijekci mezi řešeními (předchozí věta).	

\end{proof}

\item Vyslovte a dokažte větu o ekvivalentních definicích regulárních matic. 

Pro čtvercovou matici řádu $n$ jnpe: 
\begin{enumerate}[label=\arabic*]
\item A je regulární, tj. $\exists A^{-1}: AA^{-1} = I_n$
\item A má hodnost $n$
\item A lze elementárními úpravami převést na $I_n$
\item Homogenní soustava $Ax = 0$ má pouze triviální řešení.
\end{enumerate}

\begin{proof}
\begin{itemize}
\item $2 \iff 4$: Z věty o řešení homogenních soustav 5.: 
\begin{equation*}
rank(A) = n \iff A^\prime \sim A \text{ neobsahuje volné proměnné } \iff  \text{ existuje právě jedno řešení }
\end{equation*}
\item $2 \implies 3$:
Provedeme Gaussovu eliminaci a potom eliminujeme prvky nad pivoty.
\item $3 \implies 2$: triviální.

\item $1 \iff 2$: \begin{itemize}

\item $2 \implies 1$

Označme $e_1 ... e_n$ sloupce matice $I_n$ a uvažme $n$ soustav pro $Ax^i = e^i$ pro $i = 1 ... n$

$rank(A) = n \implies$ tyto soustavy mají řešení $\implies A^{-1} = (x_1, ..., x_n)$

\item $1 \implies 2$

Sporem. Kdyby $rank(A) < n \implies$ nějaký i-tý řádek lze eliminovat ostatními $\implies$ soustava $Ax_i = e_i$ nemá řešení, protože 1 na i-tém řádku v $e_i$ nelze eliminovat nulami. ... spor s existencí $A^{-1}$.

\end{itemize}

\end{itemize}
\end{proof}

\item Vyslovte a dokažte větu o znaménku složené permutace.

\begin{equation*}
sgn(q \circ p) = sgn(q) \cdot sqn(p)
\end{equation*}

\begin{proof} 
Pro počet inverzí ve složené permutaci platí, že:

\begin{equation*}
\# \text{ inverzí } p \circ q = \# \text{ inverzí } p + \# \text{ inverzí } q - 2 \cdot |\{(i,j): i, j \text{ je inverzí v q}, (q(i), q(j)) \text{ je inverzí v } p \}| 
\end{equation*}

V rámci diagramu: křížení s ostatními hranami neovlivní celkový počet.

Odečtením dvojnásobku, tj. sudého čísla, zachováme paritu.

\end{proof}

\item Uved’te a dokažte větu charakterizující, kdy Zn je těleso.

$\mathbb{Z}_n$ je těleso $\iff$ $n$ je prvočíslo.

\begin{proof}
\begin{itemize}
\item $\impliedby$
Sporem. Nechť $n = p\cdot q, 1 < p,q < n$. 
\begin{equation*}
pq \equiv 0 \mod n \implies p = 0 \vee q = 0
\end{equation*}
Spor. (Proč je to spor, když tam je kongruence a ne rovná se...)

\item $\implies$
Je potřeba ukázat platnost axiomů. Všechny axiomy plynou z celých čísel, kromě existence inverzních prvků:

Ukažme existenci inverzního prvku v násobení $\forall a \exists a^{-1}: a\cdot a^{-1} \equiv 1 \mod n$

Definujeme pro každé $a$ zobrazení $f_a: \{0, ..., n-1 \} \to \{0, ..., n-1 \}$ předpisem $f_a(x) = a \cdot x \mod n$

Ukažme, že $f_a$ je prosté: Sporem. Kdyby nebylo, $\exists b,c, b \neq c: f_a(b) = f_a(c) \implies 0 \equiv ab-ac \implies a (b-c) \equiv 0$. Ale $a \neq 0$ a $b-c \neq 0$. Spor.

$f_a$ je prosté $\implies$ je na $\implies \exists a^{-1}$ splňující $f_a(a^{-1}) = 1$.

\end{itemize}
\end{proof}
\item Vyslovte a dokažte malou Fermatovu větu.

Nechť p je prvočíslo a $0 \neq a \in \mathbb{Z}_p$. Pak $a^{p-1} = 1$ v tělese $\mathbb{Z}_p$.


\begin{proof}
Využijeme zobrazení z přechozího důkazu: 
Definujeme pro každé $a$ zobrazení $f_a: \{0, ..., n-1 \} \to \{0, ..., n-1 \}$ předpisem $f_a(x) = a \cdot x \mod n$

Ukažme, že $f_a$ je prosté: Sporem. Kdyby nebylo, $\exists b,c, b \neq c: f_a(b) = f_a(c) \implies 0 \equiv ab-ac \implies a (b-c) \equiv 0$. Ale $a \neq 0$ a $b-c \neq 0$. Spor.

$f_a$ je prosté $\implies$ je na $\implies$ je bijekcí.

Potom 

\begin{equation*}
\prod_{i=1}^{p-1} i = \prod_{i=1}^{p-1} f_a(a) = \prod_{i=1}^{p-1} a\cdot i = a^{p-1} \prod_{i=1}^{p-1} i  \implies a^{p-1} = 1
\end{equation*}
\end{proof}


\item Vyslovte a dokažte větu o průniku vektorových prostorů.

Nechť $U_i, i \in I$ je systém podprostorů $V$, Potom $\displaystyle \bigcap_{i \in I} U_i$ je také podprostorem V.

\begin{proof}
Označme $W = \displaystyle \bigcap_{i \in I} U_i$.

Ověřme uzavřenost na obou operacích:
\begin{enumerate}
\item Uzavřenost na $\odot$
\begin{equation*}	
\forall a \in \mathbb{K}: u \in W \implies \forall i: u \in U_i \implies \forall i: a \odot u \in U_i \implies a \odot u \in W	
\end{equation*}

\item Uzavřenost na $\oplus$

\begin{equation*}
u, v \in W \implies \forall i: u,v \in U_i \implies \forall i: u\oplus v \in U_i \implies u\oplus v \in W
\end{equation*}
\end{enumerate}

\end{proof}


\item Vyslovte a dokažte větu o ekvivalentních definicích lineárního obalu.

Lineární obal jsme definovali dvěma způsoby, tzv. výstavbami "zevnitř" a "zevnějšku".

Obě definice jsou ekvivalentní:

\begin{enumerate}[label=\arabic*]
\item Lineární obal množiny $X \subseteq V$ je průnik všech podprostorů, které obsahují $X$, kde $V$ je vektorový prostor nad $\mathbb{K}$.
\item Lineární obal množiny $X$ je množina, která obsahuje všechy lineárních kombinace vektorů z $X$.
\end{enumerate}

\begin{proof}
Označme 

$W_1 = \displaystyle \bigcap_{X \subseteq U_i \subseteq V} U_i$ 

$W_2 = \{ \displaystyle \sum^n_{i=1} a_i \cdot v_i: a_i \in \mathbb{K}, v_i \in X, n \in \mathbb{N} \}$

Dokažme $W_1 = W_2$:

\begin{enumerate}

\item $W_2 \subseteq W_1$
\begin{equation*}
u \in W_2 \implies u = \sum_{v_i \in X} a_i v_i \underset{\forall U \text{ je uzavřený na } +, \cdot }{\implies} \forall U: u \in U \implies u \in W_1
\end{equation*}

\item $W_1 \subseteq W_2$

 
$W_2$ je podprostor uzavřený na sčítání a násobení (lze ukázat jednoduše), nutně obsahuje množinu $X$ (stačí zvolit vždy $a_i = 1, a_{j\neq i} = 0$), čili jeden z možných podprostorů U tvořících průnik ($\exists i: U_i = W_2$), tedy definičně $W_1 \subseteq U_i \implies W_1 \subseteq W_2$

\end{enumerate}


\end{proof}

\item Uved’te a dokažte Steinitzovu větu o výměně (včetně lemmatu, pokud jej potřebujete). 

\paragraph{Lemma o výměně} $V$ je vektorový prostor nad $\mathbb{K}$ a $X = {v_1, ..., v_n}$ je systém generátorů $V$.

Potom pro libovolný vektor $u \in V$ platí, že když lze $u$ vyjádřit jako $u = \displaystyle \sum^n_{i=1} a_1v_1$, tak pro každé $i$, kde $a_i \neq 0$ platí, že $X^\prime = (X \setminus v_i) \cup u$ také generuje $V$

\begin{proof}
\begin{equation*}
u = a_1v_1 + ... + a_nv_n \implies v_i = \frac1a_i (u - (a_1v_1 + ... + a_{i-1}v_{i-1}+  a_{i+1}v_{i+1} + ...  a_{n}v_{n}))
\end{equation*}
\begin{equation*}
\forall w \in V: w = \sum_{j=1}^n b_j v_j
\end{equation*}
V tomto součtu můžeme za $v_i$ nahradit vyjádření vyše, což bude lineární kombinace $X^\prime$
\end{proof}

\paragraph{Steinitzova o výměně} $V$ je vektorový prostor nad $\mathbb{K}$ a $X \subseteq V$ je lineárně nezávislá množina a $Y$ je systém generátorů $V$, kde Y je konečná.

Potom platí $|X| \leq |Y|$ a existuje $Z$, taková že: \begin{enumerate}
\item $\mathfrak{L}(Z) = V$
\item $|Z| = |Y|$
\item $X \subseteq Z$
\item $Z \setminus X \subseteq Y$
\end{enumerate}

\begin{proof}
Indukcí dle $|X \setminus Y|$
\begin{itemize}
\item Základní krok $|X \setminus Y| = 0$: Určitě platí $|X| \leq |Y|$, lze použít $Z = Y$
\item Indukční krok $X \setminus Y \neq \emptyset$

Vyberu $u \in X \setminus Y$ libovolné a položíme $X^\prime = X \setminus u$

Použijeme indukční předpoklad pro $X^\prime$ a $Y$ a najdeme $Z^\prime$ splňující předpoklady a-d.

Použijeme  lemma o výměně pro $Z^\prime = \{ v_1, ..., v_n\}$ a $u$ vyměníme za $v_i$, takové že $v_i \in Z^\prime \setminus X^\prime$.

Takové $v_i$ existuje, protože jinak by byla $X^\prime$ lineárně závislá. (je to složité na představu, intuitivně: množina Y umí vygenerovat u, ale množina $X^\prime$ to nemůže umět, jinak by $X^\prime \cup {u}$ nebylo lin. nezavislé...).
\end{itemize}
\end{proof}


\item Vyslovte a dokažte větu o jedinečnosti lineárního zobrazení.

\begin{proof}

\end{proof}

\item Vyslovte a dokažte větu o charakterizaci izomorfismu mezi vektorovými prostory.

\begin{proof}

\end{proof}

\item Vyslovte a dokažte větu o vektorových prostorech souvisejících s maticí A.

\begin{proof}

\end{proof}

\end{enumerate}



\newpage
\part{Přehled}
(U  přehledových  otázek  uved’te  definice,  tvrzení,  věty,  příklady  a  souvislosti.  Důkazy  u přehledových otázek nejsou vyžadovány.)


\begin{enumerate}

\item Přehledově sepište, co víte o elementárních řádkových operacích a Gaussově eliminaci. 

\begin{itemize}[label=$\circ$]
\item Definice řádkových úprav - dvě hlavní a dvě odvozené.
\item Soustava jejiž matici lze vytvořit řádkovými úpravami rozšířené matice sosutavy má stejné řešení jako tato soustava.

\item Řádkové úpravy lze zapsat jako regulární matice. 
\end{itemize}

\item Přehledově sepište, co víte o řešení homogenních a nehomogenních soustav lineárních rovnic.

\begin{itemize}[label=$\circ$]
\item Co to je soustava, matice soustavy, pravých stran, rozšířená matice soustavy a vektor neznámých.
\item Gaussova eliminace a zpětná substituce
\item Charakter proměnné v odstupňovaném tvaru je určen jednoznačně. (Důkaz sporem)
\item Libovolné hodnoty volných proměnných lze doplnit jednoznačně hodnotami bázických proměnných a získat řešení soustavy. (Důkaz indukcí dle řádku odspodu začínaje prvním nenulovým řádkem)
\item Zpětnou substitucí lze získat každé řešení. (Jak to dokázat)
\item Frobeniova věta
\item Každé řešení homogenní soustavy lze zapsat jako: $x = p_1\overline{x}_1 + ... + p_{n-rank}\overline{x}_{n-rank} $
\begin{proof}
Volné proměnné přepíšu jako parametry a provedu zpětné dosazení. (Postup ze cvika)
\end{proof}

\item Jak popsat všechna řešení?
\begin{itemize}
\item Pokud $Ax^0 = b$ a $Ax = b \implies x-x^0 = \overline{x}, A\overline{x} = 0$

Důkaz triviální.
\item $A\overline{x} = 0$ a $Ax^0 = b \implies A(\overline{x} + x^0) = b$ 
\end{itemize}


\item Věta: Nechť $Ax^0 = b$, potom $f: \overline{x} \to \overline{x} + x^0$ je bijekcí množiny řešení $Ax = 0$ 
\begin{proof}
Nechť $g: x \to x - x^0:$ 

$g \circ f = id$ možných řešení homogenní soustavy.

$f \circ g = id$ možných řešení heterogenní soustavy.

$\implies f$ je bijekce.  
\end{proof}

Důsledek: Každé řešení $Ax =b$ lze zapsat jako $x^0 +p_1\overline{x}_1 + ... + p_{n-rank}\overline{x}_{n-rank}$


\end{itemize}

\item Přehledově sepište, co víte o maticových operacích.

\begin{itemize}[label=$\circ$]
\item Definice nulové matice, jednotkové matice.
\item Definice násobení skalárem, sčítání a maticového součinu.
\begin{itemize}
\item Násobení skalárem je komutativní asociativní i ditributivní na sčítání, neutrální prvek je 1.
\item Sčítání je komutativní a asociativní, neutrální prvek je nulová matice.
\item Maticový součin není komutativní, je asociativní $A(BC) = (AB)C$a neutrální prvek je jednotková matice správného rozměru.
\item Různým uzávorkováním maticového součinu můžeme změnit počet nutných operací k jeho výpočtu, to je možno brát v potaz při výpočtu.
\item Maticový součin je distributivní na sčítání.
\end{itemize}
\item Definice transponované matice

\item $(AB)^T = B^T A^T$
\end{itemize}

\item Přehledově sepište, co víte o regulárních a singulárních maticích. 

\begin{itemize}[label=$\circ$]
\item Definice regulární matice (Čtvercová matice, k níž jde sestrojit inverzi)
\item Věta o ekvivalentních definicích regulárních matic
\begin{itemize}
\item $A$ je regulární
\item $rank(A) = n$  
\item $A \sim\sim I_n$
\item $Ax = 0 \implies x = 0$
\end{itemize}
\item platí $A^{-1} A = I_n$
\begin{proof}
Nepřímo, ukažme $A^{-1}$ je regulární:

Sporem: $\exists x, x\neq 0: A^{-1}x = 0$.

$x= I_n x = A A^{-1}x = A0 = 0$ Spor.
\end{proof}
\item Pro R regulární platí $A = B \iff AR = BR \iff RA = RB$
\item $(A^{-1})^{-1} = A$
\item Jak vypočítat inverzi matice?
\begin{proof}
$(A^{-1})^{-1} = I_n (A^{-1})^{-1}  = A A^{-1}(A^{-1})^{-1} = A I_n = A$
\end{proof}
\item součin regulárních matic je regulární
\begin{proof}
$(AB)(B^{-1}A^{-1}) = A(BB^{-1})A^{-1} = I_n$.  Existuje $(AB)^{-1})$, a to $(B^{-1}A^{-1})$.
\end{proof}
\item $(AB)^{-1} = B^{-1}A^{-1}$
Důkaz stejně jako výše.
\item $(A^T)^{-1} =(A^{-1})^T$
\begin{proof}
Využijeme, že $X^T Y^T = (YX)^T$:

$(A^{-1})^T = (A^{-1})^T A^T (A^T)^{-1} =(AA^{-1})^T  (A^T)^{-1} =  I_n (A^T)^{-1} = (A^T)^{-1} $ 
 
\end{proof}
\end{itemize}

\item Přehledově sepište, co víte o binárních operacích a jejich vlastnostech. 

Není toho moc, co by se dalo říct

\begin{itemize}[label=$\circ$]
\item Binární operace na X je zobrazení $X \times X \to X$.
\item Definice zobrazení, relace, kartézského součinu.
\item Možné vlastnosti operací: Asociativita, Komutace, Distribuce na jiné operaci, inverzní a neutrální prvky.

\item Příklady
\item Grupa, Abelova grupa
\end{itemize}


\item Přehledově sepište, co víte o (obecných) grupách.

\begin{itemize}[label=$\circ$]
\item Definice, tři podmínky
\item Abelova grupa
\item Neutrální a inverzní prvky jsou určeny jednoznačně (Důkaz přičtením nuly/ násobením jedničkou)
\item Platí ekvivalentní úpravy $a = b \iff c \circ a = c \circ b \iff a \circ c = b \circ c$
\item Rovnice $a \circ x = b$ a $x \circ a = b$ mají jednoznačná řešení.
\item $(a^{-1})^{-1} = a$
\item $(ab)^{-1} = (b^{-1}a^{-1})$ 
\item Příklady: aditivní grupy, multiplikativní, ostatní (symetrická - množina permutací na 1 až n)

\end{itemize}

\item Přehledově sepište, co víte o permutačních grupách. 

\begin{itemize}[label=$\circ$]
\item
\end{itemize}

\item Přehledově sepište, co víte o tělesech.

\begin{itemize}[label=$\circ$]
\item Definice (4 podbody)
\item Příklady: $\mathbb{R, Q, Z}_p$, racionální lomené funkce
\item Tělesa nejsou $\mathbb{R, Q, Z}_p$ kde p není prvočíslo
\item Vlastnosti: \begin{itemize}
\item $\forall a, a \times 0 = 0$
\item $ab = 0 \implies a = 0 \vee b = 0$ \begin{proof}
Sporem: $\exists a^{-1}, b^{-1}: 1 = a^{-1}abb^{-1}= aba^{-1}b^{-1} = 0a^{-1}b^{-1} = 0$
\end{proof}
\item $a (-1) = -a$ \begin{proof}
$0 = 0a = (1-1)a = 1a + (-1)a \implies -a = (-1)a$
\end{proof}
\item rovnice $a^\prime x = b$ a $a + x = b$ mají jednoznačené řešení, pokud $a^\prime \neq 0$ 
\item lze používat ekvivalentní úpravy (přičtení, násobení)
\item charakteristika (def)
\end{itemize}
\item Věty \begin{itemize}
\item  $\mathbb{Z}_p$ je těleso $\iff p$ je prvočíslo ($\implies:$ Obrácenou implikaci sporem. $\impliedby$: dokázat axiomy, hlavně inverzní prvky přes $f_a$)
\item Malá fermatova věta (důkaz přes $f_a$)
\item Charakteristika tělesa je vždy 0 nebo prvočíslo (důkaz sporem)
\end{itemize}
\end{itemize}

\item Přehledově sepište, co víte o vektorových prostorech a jejich podprostorech. 

\begin{itemize}[label=$\circ$]
\item Definice (sčítání tvoří Abelovu grupu, násobení jedničkou, asociativita, distributivita zprava, zleva)
\item Pojem skalár, vektor
\item Příklady: $\mathbb{K}^n$, posloupnosti, funkce, polynomy
\item $a0 = 0u = 0$
\item $a u = 0 \implies a = 0 \vee u = 0$
\item Podprostor \begin{itemize}
\item Definice
\item Podprostor je prostor
\item Příklady, geometrická interpretace R
\item Věta: Průnik podprostorů je podprostor (ověří se uzavřenost na +,.) 
\item Lineární kombinace, obal (věta o dvojím budování)
\end{itemize} 
\end{itemize}

\item Přehledově sepište, co víte o vektorových prostorech určených s maticí A. 

Pro $A \in \mathbb{K}^{m\times n}$
\begin{itemize}[label=$\circ$]
\item Řádkový prostor $\mathfrak{R}(A)= \{ r \in \mathbb{K}^n: r =A^Tu, u \in \mathbb{K}^m\}$
\item Sloupcový prostor $\mathfrak{S}(A)= \{ s \in \mathbb{K}^m: s =Au, u \in \mathbb{K}^n\}$
\item jádro $Ker(A)= \{ u \in \mathbb{K}^n: Au =  0 \}$
\item Elementární řádkové úpravy zachovávají řádkový prostor, sloupcový zachovávat nemusí. (+ Jádro)
\item $dim(\mathfrak{R}) = rank(A)$,$dim(Ker{R}) = n - rank(A)$
\item Věta o shodnosti dimenzí: $ dim(\mathfrak{R}) = dim(\mathfrak{S})$ (Důkaz pomocí lemmy která říká, že po úpravě (vynásobením regulární maticí) se dimenze sloupcového prostoru zmenší nebo zůstane stejná).
\end{itemize}


\item Přehledově sepište, co víte o lineární závislosti.

\begin{itemize}[label=$\circ$]
\item Definice: Množina je lineárně nezávislá, pokud $\sum = 0 \implies \forall i: a_i = 0$
\item Příklady: \begin{itemize}
\item $|X| = 1 \begin{cases} X = \{0\} & \text{ závislá} \\ \text{ jinak} & \text{ nezávislá} \end{cases}$  
\item $0 \in X \implies X$ je lineárně závislá.
\item Řádky/sloupce diagonální matice jsou lineárně nezávislé.
\item Nenulové řádky v matici v REF jsou lineárně nezávislé
\end{itemize}
\item Y je lineárně nezávislá a $X \subseteq Y \implies X$ je lineárně nezávislá
\item X je lineárně závislá a $X \subseteq Y \implies X$ je lineárně závislá
\item  X je lineárně nezávislá $\iff \forall u \in X: u \notin \mathfrak{L}(X\setminus u)$
\item Asi je možné zmínit báze
\end{itemize}


\item Přehledově sepište, co víte o bázích vektorových prostorů. 

\begin{itemize}[label=$\circ$]
\item Definice 
\item Vektor souřadnic
\item Pro libovolnou bázi platí: (předpoklady neuvádím) $[x]_B + [y]_B = [x+y]_B$, $[a x]_B = a [x]_B$
\item Věta: $\mathfrak{L}(X) = V, \forall Y \subset X: \mathfrak{L}(Y) \neq V \implies X$ je báze. Důsledek: Z každého systému generátorů lze vytvořit bázi.
\item Důsledek: Každý prostor má bázi.
\item Z každého systému generátorů lze vytvořit bázi
\item Steinitzova věta o výměně (+ lemma)
\item Pokud má prostor konečnou bázi, potom mají všechny báze stejnou mohutnost


\end{itemize}

\item Přehledově sepište, co víte o lineárních zobrazeních a jejich maticích.

\begin{itemize}[label=$\circ$]
\item Definice
\item Příklady: nulové, identické 
\item Složení lineárních zobrazení je lineární
\item Matice lineárního zobrazení - definice
\item $[f(u)]_Y = [f]_{XY} \cdot [u]_{X}$
\item Skládání zobrazení vyjádříme součinem matic
\item Matice přechodu - definice, užití
\item Zobrazení je isomorfismus, iff jeho matice je regulární,
\item pak platí inverzní matice je maticí inverzního zobrazení
\item Vektorový prostor dimenze n je isomorfní prostoru nad $\mathbb{K}^n$
	
\end{itemize}

\end{enumerate}



\end{document}