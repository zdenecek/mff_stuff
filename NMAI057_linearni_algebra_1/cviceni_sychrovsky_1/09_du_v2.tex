\documentclass[10pt,a4paper]{article}
\usepackage[utf8]{inputenc}
\usepackage{amsmath}
\usepackage{amsfonts}
\usepackage{amssymb}
\usepackage{array}
\setlength{\parindent}{0em}
\usepackage{xcolor}


\title{Řešení domácího úkolu č. 10}
\date{}

\begin{document}

\maketitle

\section{} 

Přepišme si bázi do souřadnicového vektoru a spočítejme jako soustavu pomocí Gaussovy eliminace:


Vzhledem k tomu, že se báze jeví odhadem jako nezávislé a že jsou tři v trojrozměrném prostoru, bude řešení asi jen jedno.

\begin{equation*}
\color{gray}
\begin{array}{c}
x^2 \\ x \\ 1
\end{array}
\color{black}
\left(
\begin{array}{ccc|c}
1 & 0 & 2 & 1\\
0 & 1 & 1 & 0 \\
1 & -2 & -1 & 2 
\end{array}
\right)
\sim
\left(
\begin{array}{ccc|c}
1 & 0 & 2 & 1\\
0 & 1 & 1 & 0 \\
0 & 0 & -1 & 1 
\end{array}
\right)
\sim
\left(
\begin{array}{ccc|c}
1 & 0 & 0 & 3\\
0 & 1 & 0 & 1 \\
0 & 0 & 1&  -1 
\end{array}
\right)
\end{equation*}

Řešení je opravdu jedno:

\begin{equation*}
[x^2 + 2]_B = (3, 1 , -1)
\end{equation*}


\section{} %2

\textbf{i)} Matici přechodu od báze $B_i$ ke kanonické bázi získáme zapsáním vektorů báze do matice.

\begin{equation*}
P_{B_i \to *}[u]_{B_i} = u
\end{equation*}

Pro nalezení matice přechodu v opačném směru stačí nalézt inverzi matice přechodu.


\begin{equation*}
P_{ B_i \to *}^{-1}P_{ B_i \to *}[u]_{B_i} = P_{ B_i \to *}^{-1}u \implies P_{B_i \to *}^{-1}u = [u]_{B_i} \implies P_{B_i \to *}^{-1} = P_{* \to B_i }
\end{equation*}

Dává smysl, že vektory báze musí být nezávislé a matice tedy regulární. Nemohli bychom přejít z prostoru do prostoru s více dimenzemi, aniž bychom si nějakou souřadnici nevymysleli.

\begin{equation*}
P_{* \to B_1} = P_{B_1 \to *}^{-1} = 
\begin{pmatrix}
1 & 1 & 3 \\
1 & 0 & -1 \\
1 & 2 & 2 
\end{pmatrix}^{-1}
\overset{Wolfram}{= \cdots =} 
\frac15
\begin{pmatrix}
2 & 4 & -1 \\
-3 & -1 & 4\\
2 & -1 & -1\\
\end{pmatrix}
\end{equation*}


\begin{equation*}
P_{* \to B_2} = P_{B_2 \to *}^{-1} = 
\begin{pmatrix}
2 & 1 & -1 \\
2 & 0 & 0 \\
1 & 0 & 2 
\end{pmatrix}^{-1}
\overset{Wolfram}{= \cdots =}
\frac14
\begin{pmatrix}
0 & 2 & 0 \\
4 & -5 & 2 \\
0 & -1 & 2 
\end{pmatrix}
\end{equation*}

\textbf{ii)} Můžeme spočítat jako složení přechodu od $B_2$ ke kanonické bázi a poté z kanonické báze do $B_1$
\begin{equation*}
P_{B_2 \to B_1} = P_{B_2 \to * \to B_1}= P_{* \to B_1} \times  P_{B_2\to *} =
\frac15
\begin{pmatrix}
2 & 4 & -1 \\
-3 & -1 & 4\\
2 & -1 & -1\\
\end{pmatrix}
\times 
\begin{pmatrix}
2 & 1 & -1 \\
2 & 0 & 0 \\
1 & 0 & 2 
\end{pmatrix} =
\end{equation*}

\begin{equation*}
\overset{Photomath}{= \cdots =} \frac15
\begin{pmatrix}
11 & 2 & -4 \\
-4 & -3& 11 \\
1  & 2 & -4
\end{pmatrix}
\end{equation*}

\textbf{iii)} Využijeme naši matici přechodu 

\begin{equation*}
P_{B_2 \to B_1}[v]_{B_2} = [v]_{B_1}
\end{equation*}


\begin{equation*}
[v]_{B_1} = \frac15
\begin{pmatrix}
11 & 2 & -4 \\
-4 & -3& 11 \\
1  & 2 & -4
\end{pmatrix}
\times
\begin{pmatrix}
1 \\ 2 \\3
\end{pmatrix}
\overset{Photomath}{= \cdots =} \frac15
\begin{pmatrix}
3 \\ 23 \\ -7
\end{pmatrix}
\end{equation*}

Pro jistotu jsem výsledek ověřil:

\begin{equation*}
\begin{pmatrix}
1 & 1 & 3 \\
1 & 0 & -1 \\
1 & 2 & 2 
\end{pmatrix} \times \frac15
\begin{pmatrix}
3 \\ 23 \\ -7
\end{pmatrix}\overset{Photomath}{= \cdots =}
\begin{pmatrix}
1 \\ 2 \\ 7
\end{pmatrix} 
\end{equation*}


\begin{equation*}
\begin{pmatrix}
2 & 1 & -1 \\
2 & 0 & 0 \\
1 & 0 & 2 
\end{pmatrix} \times 
\begin{pmatrix}
1 \\ 2 \\ 3
\end{pmatrix}\overset{Photomath}{= \cdots =}
\begin{pmatrix}
1 \\ 2 \\ 7
\end{pmatrix} 
\end{equation*}

\end{document}