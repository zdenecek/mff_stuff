\documentclass[10pt,a4paper]{article}
\usepackage[utf8]{inputenc}
\usepackage{amsmath}
\usepackage{amsfonts}
\usepackage{amssymb}
\usepackage{array}
\setlength{\parindent}{0em}
\begin{document}
1.

Identická permutace:
\begin{equation*}
I =
\begin{pmatrix}
1 & 2 & 3 & 4 \\
1 & 2 & 3 & 4
\end{pmatrix} 
\end{equation*}


Osové symetrie:

Když je osa rovnoběžná na stranu.

\begin{equation*}
O_1 =
\begin{pmatrix}
1 & 2 & 3 & 4 \\
2 & 1 & 4 & 3
\end{pmatrix} = (1,2)(3,4)
\end{equation*}
\begin{equation*}
O_2 =
\begin{pmatrix}
1 & 2 & 3 & 4 \\
3 & 4 & 1 & 2
\end{pmatrix} = (1,3)(2,4)
\end{equation*}

Když je osa diagonála
\begin{equation*}
O_3 =
\begin{pmatrix}
1 & 2 & 3 & 4 \\
1 & 3 & 2 & 4
\end{pmatrix} = (2,3)
\end{equation*}
\begin{equation*}
O_4 =
\begin{pmatrix}
1 & 2 & 3 & 4 \\
4 & 2 & 3 & 1
\end{pmatrix} = (1,4)
\end{equation*}

Středová symetrie:
\begin{equation*}
S =
\begin{pmatrix}
1 & 2 & 3 & 4 \\
4 & 3 & 2 & 1
\end{pmatrix} = O_1 \circ O_2
\end{equation*}

Rotační symetrie:
\begin{equation*}
R_1 =
\begin{pmatrix}
1 & 2 & 3 & 4 \\
4 & 1 & 2 & 3
\end{pmatrix} = (1,4)(2,1)(3,2)(4,3)
\end{equation*}
\begin{equation*}
R_2 =
\begin{pmatrix}
1 & 2 & 3 & 4 \\
3 & 4 & 1 & 2
\end{pmatrix} = R_1^2
\end{equation*}
\begin{equation*}
R_3 =
\begin{pmatrix}
1 & 2 & 3 & 4 \\
2 & 3 & 4 & 1
\end{pmatrix} = R_1^3
\end{equation*}

Neutrální prvek je I stejně jako v $S_4$

Inverzní prvky jsou stejné jako v $S_4$ a pro všechny symetrie kromě rotace jsou inverzemi samotné permutace, pro rotace se pak rotace doplní tak, aby se celkem rotovalo 4-krát.

Asociativita platí, protože se jedná o podmnožinu permutací, na kterých asociativita platí.

Zbývá naznačit uzavřenost: Každou takovouto operací zachováváme tvar čtverce, taky se jedná o všechny možné permutace zachovávající tvar čtverce, proto je podgrupa uzavřená.

\hfill

2.

1)

\hfill

Podívejme se na součin:

$\sum_{k=1}^n P_{i,k} \cdot P^T_{k,j}$

Přičemž $P_{i,k} \cdot P^T_{k,j} = 1$, pokud $P_{i,k} = P^T_{k,j} = 1$

To znamená, že $p(i) = k$, ale taky si uvědomme, že transpozice znamená prohození souřadnic, takže vlastně taky platí $p(j) = k$.

Protože je ale $p$ bijekcí, musí platit $j = i$, tedy součinem bude jednotková matice.

\hfill

Z poznatků výše si můžeme uvědomit, že $P^T$ je permutační matice inverzní permutace. Je pak zřejmé, že složením těchto permutací dostaneme identitu, a permutační maticí identity je jednotková matice.


\hfill

2) Každá permutace $p$ se skládá z cyklů maximální délky $n$, horní mez k bude nejmenší společný násobek všech možných délek cyklů, což se bude rovnat součinu všech prvočísel rovných nebo menších $n$.


\hfill

3) 
Najděme permutaci o cyklech délky dva a tři, například:
$p = (1 2 3)(4 5)$

\begin{equation*}
P =
\begin{pmatrix}
0 & 1 & 0 & 0 & 0 \\
0 & 0 & 1 & 0 & 0 \\
1 & 0 & 0 & 0 & 0 \\
0 & 0 & 0 & 0 & 1 \\
0 & 0 & 0 & 1 & 0
\end{pmatrix}
\end{equation*}

\hfill

4) 

Aby byl součin definován, musí platit $A \in \mathbb{R}^{n \times n}$

Permutaci P nejdříve aplikujeme jako řádkovou úpravu, prohodíme $k$-tý řádek s $(n-k + 1)$-ním řádkem, potom aplikujeme jako sloupcovou úpravu inverzní permutaci (která bude ve skutečnosti stejná jako původní permutace, protože délky cyklů jsou dva) prohodíme $(n-k + 1)$-ní sloupec s $k$-tým.


\begin{equation*}
(PA)_{i,j} = A_{n-i+1,j}
\end{equation*}
\begin{equation*}
(PAP^T)_{i,j} = A_{n-i+1,n-j+1}
\end{equation*}

Představit si to můžeme takto: Nejdříve v matici přehodíme řádky podle osy prostředního(ch) řádku(ů), potom přehodíme sloupce podle osy prostředního(ch) sloupce(ů), takže převrátíme prvky matice podle středu, podobně jako u čtverce v úloze jedna.

\end{document}



